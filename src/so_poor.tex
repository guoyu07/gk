\chapter{你既然这么穷,为什么不去赚钱,来搞什么研究?}

当年到美国留学的时候\cite{sopoor},我下飞机的第二天就去找我的指导教授,我的指导教授是John Hopkins 毕业的,在贝尔实验室作过科学家,后来来到大学任教,很年轻都当了正教授,后来又当了系主任,人到中年的时候,离开大学,自己创业,几年以后以失败告终,然后又回大学任教。

第一次见到老板,又是个老外教授,我当然是毕恭毕敬,说:教授,我是中国来的留学生,来读博士,我对您的研究方向很感兴趣,能不能告诉我现在需要看什么书,我可以马上回去看,以便可以很快上手,开始和你一起搞研究。

教授听完:上下打量了一下我。然后不紧不慢地说:你很有钱吗?最少是百万富翁吗?
我非常吃惊,我想一定得说没有,有钱的话他就不会给我助研,如果这样我就没有钱读书。就得卷铺盖回中国去。所以我说;我是一个中国来的留学生,我没什么钱,但是很想学习东西,特别是对研究有兴趣。
教授又问:你家里很有钱吗?
这回我反正铁定心了,我回答:没有,我是中国来的,中国很穷,我家里也没有钱。
教授听完,说:你既然这么穷,为什么不去赚钱,来搞什么研究?研究是有钱,有闲,吃饱没事干的人干的事,只有有兴趣,又有钱的人,才能真正搞点研究。你说你对我的研究方向感兴趣,我看你根本没有兴趣,不过是你想为我打工赚钱而已,完全不是你想搞什么研究。
我听了满脸通红,说老实话,我根本不知道这个教授搞的研究方向,说感兴趣是因为不干这个廉价的劳工,教授不给我助研,我就没钱读书,所以为了钱,不得不说感兴趣。



教授见的学生多了,什么人对研究感兴趣,什么人对钱感兴趣,一看便知道。当然教授还是让我干了他的助研,我也很努力,虽然没有任何研究的天赋,还是任劳任怨的在实验室干苦力。教授的研究经费一直资助我拿到博士学位。毕业典礼以后,我又去见我的教授,我当然是非常感激,没有这么多年他的助研经费的资助,我不可能拿到我的学位。而且中国人讲究:一日为师,终生为父。所以我去看望他,想得到几句人生真谛的指点。他说:你现在已经拿到博士学位了,你来美国的第一天我就看出来,你来的目的是挣钱,现在你已经有了学位,是该出去挣钱的时候了。Go and make some real money!这就是他给我的人生指南。

多年以后,我逐渐对我的教授说的话有了较深的认识。说到底,研究就是有钱,有闲的人干的事。要搞比较深的研究(在实验室干苦力不是什么研究),一定要有两个条件:
1:要对这个东西感兴趣,非常感兴趣,(就象我没事写这个文章一样)没有任何钱,没有鲜花和美女,没有别人的赞赏,也要不停地钻研。
2:要排开经济的压力,吃喝不愁,衣食无忧的人才能搞研究。

美国的大学的终身教授(tenure)系统,基本就是这个理论。首先一个新科博士(fresh minted PhD)要经过5年的磨练,然后评选终生教授,一旦通过,一辈子就可以无饭碗的忧虑,可以一心一意地搞研究,当然学校也知道:这些终身教授里面绝大多数人是没有什么研究的潜力的。也搞不出什么东西来,就是一些实验室的苦力。不过,只要千千万万终身教授中有几个真有天分的人,加上良好的经济条件,自然可以搞出些东西来。

再举一个例子。70年代的时候复印机行业的老大Xerox担心电脑的兴起,会使得复印机的市场变小,如果大家都用电脑交换信息,当然就没有人买复印机。所以Xerox在Polo Alto建立了一个实验室,请了50-60个世界顶级电脑科学家,排除一切外界的干扰,(包括财务的压力),让他们自由的发挥自己的想象,创造未来能够领导人类的科技,当时的发明有:
1:GUI图形界面,2:以太网.,3: object-oriented programming (不知中文如何说). 这几个划时代的发明奠定了电脑未来的发展,影响到人类的文明的进程。

春天,是每年一度的美国大学录取通知书到达千万中国学子的时候,年复一年,大量的中国的学子盼望着美国大学的研究生院的录取通知书,很多学生为了得到助研的资助,就去申请博士学位,其中绝大部分的人对研究没有兴趣,只是希望去美国挣钱。他们读博士的逻辑是这样的:要去挣钱就得有美国文凭,要有美国文凭就得去美国读书,要去读书就得有奖学金,要有奖学金就得去读博士。所以要去美国就得去读博士。

但是博士本来就是培养来搞研究的能力。搞研究就得有兴趣,除此以外,最关键还要有钱,有闲。很难想象为生活奔波的人能搞出什么研究。研究象音乐一样,是有钱的人的游戏。

还是我教授的那句话:你既然这么穷,为什么不去赚钱,来搞什么研究?

\bibliographystyle{plainnat}
\bibliography{gk}
\clearpage