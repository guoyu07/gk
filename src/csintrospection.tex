\chapter{计算机科学与技术反思录}

计算机科学与技术反思录 [打印本页]
作者: CodeBaby    时间: 2010-7-6 16:55     标题: 计算机科学与技术反思录

计算机科学与技术这一门科学深深的吸引着我们这些同学们,上计算机系已经有近三年了,自己也做了一些思考,我一直认为计算机科学与技术这门专业,在本科阶段是不可能切分成计算机科学和计算机技术的,因为计算机科学需要相当多的实践,而实践需要技术;每一个人(包括非计算机专业),掌握简单的计算机技术都很容易(包括程序设计),但计算机专业的优势就在于,我们掌握许多其他专业并不“深究”的东西,例如,算法,体系结构,等等。非计算机专业的人可以很容易地做一个芯片,写一段程序,但他们做不出计算机专业能够做出来的大型系统。今天我想专门谈一谈计算机科学,并将重点放在计算理论上。

计算机理论的一个核心问题——从数学谈起:
记得当年大一入学,每周六课时高等数学,天天作业不断(那时是六日工作制)。颇有些同学惊呼走错了门:咱们这到底念的是什么系?不错,你没走错门,这就是计算机科学与技术系。我国计算机科学系里的传统是培养做学术研究,尤其是理论研究的人(方向不见得有问题,但是做得不是那么尽如人意)。而计算机的理论研究,说到底了,如网络安全,图形图像学,视频音频处理,哪个方向都与数学有着很大的关系,虽然也许是正统数学家眼里非主流的数学。这里我还想阐明我的一个观点:我们都知道,数学是从实际生活当中抽象出来的理论,人们之所以要将实际抽象成理论,目的就在于想用抽象出来的理论去更好的指导实践,有些数学研究工作者喜欢用一些现存的理论知识去推导若干条推论,殊不知其一:问题考虑不全很可能是个错误的推论,其二:他的推论在现实生活中找不到原型,不能指导实践。严格的说,我并不是一个理想主义者,政治课上学的理论联系实际一直是指导我学习科学文化知识的航标(至少我认为搞计算机科学与技术的应当本着这个方向)。

其实我们计算机系学数学光学高等数学是不够的(典型的工科院校一般都开的是高等数学),我们应该像数学系一样学一下数学分析(清华计算机系开的好像就是数学分析),数学分析这门科学,咱们学计算机的人对它有很复杂的感情。在于它是偏向于证明型的数学课程,这对我们培养良好的分析能力极有帮助。我的软件工程学导师北工大数理学院的王仪华先生就曾经教导过我们,数学系的学生到软件企业中大多作软件设计与分析工作,而计算机系的学生做程序员的居多,原因就在于数学系的学生分析推理能力,从所受训练的角度上要远远在我们之上。当年出现的怪现象是:计算机系学生的高中数学基础在全校数一数二(希望没有冒犯其它系的同学),教学课时数也仅次于数学系,但学完之后的效果却不尽如人意。难道都是学生不努力吗,我看未见得,方向错了也说不一定,其中原因何在,发人深思。

我个人的浅见是:计算机系的学生,对数学的要求固然跟数学系不同,跟物理类差别则更大。通常非数学专业的所谓“高等数学”,无非是把数学分析中较困难的理论部分删去,强调套用公式计算而已。而对计算机系来说,数学分析里用处最大的恰恰是被删去的理论部分。说得难听一点,对计算机系学生而言,追求算来算去的所谓“工程数学”已经彻底地走进了误区。记上一堆曲面积分的公式,难道就能算懂了数学?那倒不如现用现查,何必费事记呢?再不然直接用Mathematics或是Matalab好了。
我在系里最爱做的事情就是给学弟学妹们推荐参考书。中文的数学分析书,一般都认为以北大张筑生老师的“数学分析新讲”为最好。万一你的数学实在太好,那就去看菲赫金哥尔茨的 “微积分学教程”好了--但我认为没什么必要,毕竟你不想转到数学系去。吉米多维奇的“数学分析习题集”也基本上是计算型的东东。书的名气很大,倒不见得适合我们,还是那句话,重要的是数学思想的建立,生活在信息社会里我们求的是高效,计算这玩意还是留给计算机吧。不过现在多用的似乎是复旦大学的《数学分析》也是很好的教材。

中国的所谓高等代数,就等于线性代数加上一点多项式理论。我以为这有好的一面,因为可以让学生较早感觉到代数是一种结构,而非一堆矩阵翻来覆去。这里不得不提南京大学林成森,盛松柏两位老师编的“高等代数”,感觉相当舒服。此书相当全面地包含了关于多项式和线性代数的基本初等结果,同时还提供了一些有用的又比较深刻的内容,如Sturm序列,Shermon-Morrison公式,广义逆矩阵等等。可以说,作为本科生如能吃透此书,就可以算高手。国内较好的高等代数教材还有清华计算机系用的那本,清华出版社出版,书店里多多,一看就知道。从抽象代数的观点来看,高等代数里的结果不过是代数系统性质的一些例子而已。莫宗坚先生的《代数学》里,对此进行了深刻的讨论。然而莫先生的书实在深得很,作为本科生恐怕难以接受,不妨等到自己以后成熟了一些再读。

正如上面所论述的,计算机系的学生学习高等数学:知其然更要知其所以然。你学习的目的应该是:将抽象的理论再应用于实践,不但要掌握题目的解题方法,更要掌握解题思想,对于定理的学习:不是简单的应用,而是掌握证明过程即掌握定理的由来,训练自己的推理能力。只有这样才达到了学习这门科学的目的,同时也缩小了我们与数学系的同学之间思维上的差距。

概率论与数理统计这门课很重要,可惜大多数院校讲授这门课都会少些东西。少了的东西现在看至少有随机过程。到毕业还没有听说过Markov过程,此乃计算机系学生的耻辱。没有随机过程,你怎么分析网络和分布式系统?怎么设计随机化算法和协议?据说清华计算机系开有“随机数学”,早就是必修课。另外,离散概率论对计算机系学生来说有特殊的重要性。而我们国家工程数学讲的都是连续概率。现在,美国已经有些学校开设了单纯的“离散概率论”课程,干脆把连续概率删去,把离散概率讲深些。我们不一定要这么做,但应该更加强调离散概率是没有疑问的。这个工作我看还是尽早的做为好。

计算方法学(有些学校也称为数学分析学)是最后一门由数理学院给我们开的课。一般学生对这门课的重视程度有限,以为没什么用。不就是照套公式嘛!其实,做图形图像可离不开它,密码学搞深了也离不开它。而且,在很多科学工程中的应用计算,都以数值的为主。这门课有两个极端的讲法:一个是古典的“数值分析”,完全讲数学原理和算法;另一个是现在日趋流行的“科学与工程计算”,干脆教学生用软件包编程。我个人认为,计算机系的学生一定要认识清楚我们计算机系的学生为什么要学这门课,我是很偏向于学好理论后用计算机实现的,最好使用C语言或C++编程实现。向这个方向努力的书籍还是挺多的,这里推荐大家高等教育出版社(CHEP)和施普林格出版社(Springer) 联合出版的《计算方法(Computational Methods)》,华中理工大学数学系写的(现华中科技大学),这方面华科大做的工作在国内应算是比较多的,而个人认为以这本最好,至少程序设计方面涉及了:任意数学函数的求值,方程求根,线性方程组求解,插值方法,数值积分,场微分方程数值求解。李庆扬的那本则理论性过强,与实际应用结合得不太紧。

每个学校本系里都会开一门离散数学,涉及集合论,图论,和抽象代数,数理逻辑。不过,这么多内容挤在离散数学一门课里,是否时间太紧了点?另外,计算机系学生不懂组合和数论,也是巨大的缺陷。要做理论,不懂组合或者数论吃亏可就太大了。从理想的状态来看,最好分开六门课:集合,逻辑,图论,组合,代数,数论。这个当然不现实,因为没那么多课时。也许将来可以开三门课:集合与逻辑,图论与组合,代数与数论。(这方面我们学校已经着手开始做了)不管课怎么开,学生总一样要学。下面分别谈谈上面的三组内容。
古典集合论,北师大出过一本《基础集合论》不错。 数理逻辑,中科院软件所陆钟万教授的《面向计算机科学的数理逻辑》就不错。现在可以找到陆钟万教授的讲课录像,http://www.cas.ac.cn /html/Dir/2001/11/06/3391.htm自己去看看吧。总的来说,学集合/逻辑起手不难,普通高中生都能看懂。但越往后越感觉深不可测。

学完以上各书之后,如果你还有精力兴趣进一步深究,那么可以试一下GTM系列中的《Introduction to Axiomatic Set Theory》和《A Course of Mathematical Logic》。这两本都有世界图书出版社的引进版。你如果能搞定这两本,可以说在逻辑方面真正入了门,也就不用再浪费时间听我瞎侃了。

E
据说全中国最多只有三十个人懂图论。此言不虚。图论这东东,技巧性太强,几乎每个问题都有一个独特的方法,让人头痛。不过这也正是它魅力所在:只要你有创造性,它就能给你成就感。我的导师说,图论里面随便揪一块东西就可以写篇论文。大家可以体会里面内容之深广了吧!国内的图论书中,王树禾老师的“图论及其算法”非常成功。一方面,其内容在国内教材里算非常全面的。另一方面,其对算法的强调非常适合计算机系(本来就是科大计算机系教材)。有了这本书为主,再参考几本翻译的,如Bondy \& Murty的《图论及其应用》,人民邮电出版社翻译的《图论和电路网络》等等,就马马虎虎,对本科生足够了。再进一步,世界图书引进有GTM系列的"Modern Graph Theory"。此书确实经典!国内好象还有一家出版了个翻译版。不过,学到这个层次,还是读原版好。搞定这本书,也标志着图论入了门。

离散数学方面我们北京工业大学实验学院有个世界级的专家,叫邵学才,复旦大学概率论毕业的,教过高等数学,线性代数,概率论,最后转向离散数学,出版著作无数,论文集新加坡有一本,堪称经典,大家想学离散数学的真谛不妨找来看看。这老师的课我专门去听过,极为经典。不过你要从他的不经意的话中去挖掘精髓。在同他的交谈当中我又深刻地发现一个问题,虽说邵先生写书无数,但依他自己的说法每本都差不多,我实在觉得诧异,他说主要是有大纲的限制,不便多写。这就难怪了,很少听说国外写书还要依据个什么大纲(就算有,内容也宽泛的多),不敢越雷池半步,这样不是看谁的都一样了。外版的书好就好在这里,最新的科技成果里面都有论述,别的先不说,至少是“紧跟时代的理论知识”。

组合感觉没有太适合的国产书。还是读Graham和Knuth等人合著的经典“具体数学”吧,西安电子科技大学出版社有翻译版。 抽象代数,国内经典为莫宗坚先生的“代数学”。此书是北大数学系教材,深得好评。然而对本科生来说,此书未免太深。可以先学习一些其它的教材,然后再回头来看“代数学”。国际上的经典可就多了,GTM系列里就有一大堆。推荐一本谈不上经典,但却最简
单的,最容易学的:http://www.math.miami.edu/~ec/book/这本“Introduction to Linear and Abstract Algebra"非常通俗易懂,而且把抽象代数和线性代数结合起来,对初学者来说非常理想,我校比较牛的同学都有收藏。

数论方面,国内有经典而且以困难著称的”初等数论“(潘氏兄弟著,北大版)。再追溯一点,还有更加经典(可以算世界级)并且更加困难的”数论导引“(华罗庚先生的名著,科学版,九章书店重印,繁体的看起来可能比较困难)。把基础的几章搞定一个大概,对本科生来讲足够了。但这只是初等数论。本科毕业后要学计算数论,你必须看英文的书,如Bach的"Introduction to Algorithmic Number Theory"。
计算机科学理论的根本,在于算法。现在很多系里给本科生开设算法设计与分析,确实非常正确。环顾西方世界,大约没有一个三流以上计算机系不把算法作为必修的。算法教材目前公认以Corman等著的"Introduction to Algorithms"为最优。对入门而言,这一本已经足够,不需要再参考其它书。

再说说形式语言与自动机。我看过北邮的教材,应该说写的还清楚。但是,有一点要强调:形式语言和自动机的作用主要在作为计算模型,而不是用来做编译。事实上,编译前端已经是死领域,没有任何open problems,北科大的班晓娟博士也曾经说过,编译的技术已相当成熟。如果为了这个,我们完全没必要去学形式语言--用用yacc什么的就完了。北邮的那本在国内还算比较好,但是在深度上,在跟可计算性的联系上都有较大的局限,现代感也不足。所以建议有兴趣的同学去读英文书,不过国内似乎没引进这方面的教材。可以去互动出版网上看一看。入门以后,把形式语言与自动机中定义的模型,和数理逻辑中用递归函数定义的模型比较一番,可以说非常有趣。现在才知道,什么叫“宫室之美,百官之富”!

计算机科学和数学的关系有点奇怪。二三十年以前,计算机科学基本上还是数学的一个分支。而现在,计算机科学拥有广泛的研究领域和众多的研究人员,在很多方面反过来推动数学发展,从某种意义上可以说是孩子长得比妈妈还高了。但不管怎么样,这个孩子身上始终流着母亲的血液。这血液是the mathematical underpinning of computer science(计算机科学的数学基础),也就是理论计算机科学。原来在东方大学城图书馆中曾经看过一本七十年代的译本(书皮都没了,可我就爱关注这种书),大概就叫《计算机数学》。那本书若是放在当时来讲决是一本好书,但现在看来,涵盖的范围还算广,深度则差了许多,不过推荐大一的学生倒可以看一看,至少可以使你的计算数学入入门。

最常和理论计算机科学放在一起的一个词是什么?答:离散数学。这两者的关系是如此密切,以至于它们在不少场合下成为同义词。(这一点在前面的那本书中也有体现)传统上,数学是以分析为中心的。数学系的同学要学习三四个学期的数学分析,然后是复变函数,实变函数,泛函数等等。实变和泛函被很多人认为是现代数学的入门。在物理,化学,工程上应用的,也以分析为主。

随着计算机科学的出现,一些以前不太受到重视的数学分支突然重要起来。人们发现,这些分支处理的数学对象与传统的分析有明显的区别:分析研究的问题解决方案是连续的,因而微分,积分成为基本的运算;而这些分支研究的对象是离散的,因而很少有机会进行此类的计算。人们从而称这些分支为“离散数学”。“离散数学”的名字越来越响亮,最后导致以分析为中心的传统数学分支被相对称为“连续数学”。

离散数学经过几十年发展,基本上稳定下来。一般认为,离散数学包含以下学科:
1) 集合论,数理逻辑与元数学。这是整个数学的基础,也是计算机科学的基础。
2) 图论,算法图论;组合数学,组合算法。计算机科学,尤其是理论计算机科学的核心是
算法,而大量的算法建立在图和组合的基础上。
3) 抽象代数。代数是无所不在的,本来在数学中就非常重要。在计算机科学中,人们惊讶地发现代数竟然有如此之多的应用。

但是,理论计算机科学仅仅就是在数学的上面加上“离散”的帽子这么简单吗?一直到大约十几年前,终于有一位大师告诉我们:不是。D.E.Knuth(他有多伟大,我想不用我废话了)在Stanford开设了一门全新的课程Concrete Mathematics。 Concrete这个词在这里有两层含义:
首先:对abstract而言。Knuth认为,传统数学研究的对象过于抽象,导致对具体的问题关心不够。他抱怨说,在研究中他需要的数学往往并不存在,所以他只能自己去创造一些数学。为了直接面向应用的需要,他要提倡“具体”的数学。在这里我做一点简单的解释。例如在集合论中,数学家关心的都是最根本的问题--公理系统的各种性质之类。而一些具体集合的性质,各种常见集合,关系,映射都是什么样的,数学家觉得并不重要。然而,在计算机科学中应用的,恰恰就是这些具体的东西。Knuth能够首先看到这一点,不愧为当世计算机第一人。其次,Concrete是Continuous(连续)加上 discrete(离散)。不管连续数学还是离散数学,都是有用的数学!

理论与实际的结合——计算机科学研究的范畴
前面主要是从数学角度来看的。从计算机角度来看,理论计算机科学目前主要的研究领域包括:可计算性理论,算法设计与复杂性分析,密码学与信息安全,分布式计算理论,并行计算理论,网络理论,生物信息计算,计算几何学,程序语言理论等等。这些领域互相交叉,而且新的课题在不断提出,所以很难理出一个头绪来。想搞搞这方面的工作,推荐看中国计算机学会的一系列书籍,至少代表了我国的权威。下面随便举一些例子。
由于应用需求的推动,密码学现在成为研究的热点。密码学建立在数论(尤其是计算数论),代数,信息论,概率论和随机过程的基础上,有时也用到图论和组合学等。很多人以为密码学就是加密解密,而加密就是用一个函数把数据打乱。这样的理解太浅显了。
现代密码学至少包含以下层次的内容:
第一,密码学的基础。例如,分解一个大数真的很困难吗?能否有一般的工具证明协议正确?
第二,密码学的基本课题。例如,比以前更好的单向函数,签名协议等。
第三,密码学的高级问题。例如,零知识证明的长度,秘密分享的方法。
第四,密码学的新应用。例如,数字现金,叛徒追踪等。
在分布式系统中,也有很多重要的理论问题。例如,进程之间的同步,互斥协议。一个经典的结果是:在通信信道不可靠时,没有确定型算法能实现进程间协同。所以,改进TCP三次握手几乎没有意义。例如时序问题。常用的一种序是因果序,但因果序直到不久前才有一个理论上的结果....例如,死锁没有实用的方法能完美地对付。例如,......操作系统研究过就自己去举吧!
如果计算机只有理论,那么它不过是数学的一个分支,而不成为一门独立的科学。事实上,在理论之外,计算机科学还有更广阔的天空。

我一直认为,4年根本不够学习计算机的基础知识,因为面太宽了......
这方面我想先说说我们系在各校普遍开设的《计算机基础》。在高等学校开设《计算机基础课程》是我国高教司明文规定的各专业必修课程要求。主要内容是使学生初步掌握计算机的发展历史,学会简单的使用操作系统,文字处理,表格处理功能和初步的网络应用功能。但是在计算机科学系教授此门课程的目标决不能与此一致。在计算机系课程中目标应是:让学生较为全面的了解计算机学科的发展,清晰的把握计算机学科研究的方向,发展的前沿即每一个课程在整个学科体系中所处的地位。搞清各学科的学习目的,学习内容,应用领域。使学生在学科学习初期就对整个学科有一个整体的认识,以做到在今后的学习中清楚要学什么,怎么学。计算机基本应用技能的位置应当放在第二位或更靠后,因为这一点对于本系的学生应当有这个摸索能力。这一点很重要。推荐给大家一本书:机械工业出版社的《计算机文化》(New Perspective of Computer Science),看了这本书我才深刻的体会到自己还是个计算机科学初学者,才比较透彻的了解了什么是计算机科学。另外在厦门大学赵致琢老师的著作《计算科学导论》当中的很多经典理论都是在同类书籍中很难找到的。看看他也许你才会明白一个最基本的问题:为什么计算机科学叫计算科学更为准确。这本书在世界上也可成为精品级的著作。

一个一流计算机系的优秀学生决不该仅仅是一个编程高手,但他一定首先是一个编程高手。我上大学的时候,第一门专业课是C语言程序设计,念计算机的人从某种角度讲相当一部分人是靠写程序吃饭的。在我们北京工业大学实验学院计算机系里一直有这样的争论(时至今日 CSDN上也有),关于第一程序设计语言该用哪一种。我个人认为,用哪种语言属于末节,关键在养成良好的编程习惯。当年老师对我们说,打好基础后学一门新语言只要一个星期。现在我觉得根本不用一个星期,前提是先把基础打好。不要再犹豫了,学了再说,等你抉择好了,别人已经会了几门语言了。

汇编语言和微机原理是两门特烦人的课。你的数学/理论基础再好,也占不到什么便宜。这两门课之间的次序也好比先有鸡还是先有蛋,无论你先学哪门,都会牵扯另一门课里的东西。所以,只能静下来慢慢琢磨。这就是典型的工程课,不需要太多的聪明和顿悟,却需要水滴石穿的渐悟。有关这两门课的书,计算机书店里不难找到。弄几本最新的,对照着看吧。组成原理推荐《计算机组成与结构》清华大学王爱英教授写的。汇编语言大家拿8086/8088入个门,之后一定要学 80x86汇编语言。实用价值大,不落后,结构又好,写写高效病毒,高级语言里嵌一点汇编,进行底层开发,总也离不开他,推荐清华大学沈美明的《IBM— PC汇编语言程序设计》。有些人说不想了解计算机体系结构,也不想制造计算机,所以诸如计算机原理,汇编语言,接口之类的课觉得没必要学,这样合理吗?显然不合理,这些东西迟早得掌握,肯定得接触,而且,这是计算机专业与其他专业学生相比的少有的几项优势。做项目的时候,了解这些是非常重要的,不可能说,仅仅为了技术而技术,只懂技术的人最多做一个编码工人,而永远不可能全面地了解整个系统的设计,而编码工人是越老越不值钱。关于组成原理还有个讲授的问题,在我学这门课程时老师讲授时把CPU工作原理誉微程序设计这一块略掉了,理由是我们国家搞CPU技术不如别的国家,搞了这么长时间好不容易出了个龙芯比Intel的还差个十万八千里,所以建议我们不要学了。我看这在各校也未见得不是个问题吧!若真是如他所说,那中国的计算机科学哪个方向都可以停了,软硬件,应用,有几项搞得过美国,搞不过别人就不搞了,那我们坐在这里干什么?教学的观念需要转变的。

模拟电路这东东,如今不仅计算机系学生搞不定,电子系学生也多半害怕。如果你真想软硬件通吃,那么建议你先看看邱关源的“电路原理”,也许此后再看模拟电路底气会足些。教材:康华光的“电子技术基础”(高等教育出版社)还是不错的(我校电子系在用)。有兴趣也可以参考童诗白的书。

数字电路比模拟电路要好懂得多。推荐大家看一看我们北工大刘英娴教授写的《数字逻辑》业绩人士都说这本书很有参考价值(机械工业出版社的)。原因很明了,实用价值高,能听听她讲授的课程更是有一种“享受科学”的感觉。清华大学阎石的书也算一本好教材,遗憾的一点是集成电路讲少了些。真有兴趣,看一看大规模数字系统设计吧(北航那本用的还比较多)。

计算机系统结构该怎么教,国际上还在争论。国内能找到的较好教材为Stallings的"Computer Organization and Architectureesigning for Performance"(清华影印
本)。国际上最流行的则是“Computer architecture: aquantitative approach", by Patterson \& Hennessy。
操作系统可以随便选用《操作系统的内核设计与实现》和《现代操作系统》两书之一。这两部都可以算经典,唯一缺点就是理论上不够严格。不过这领域属于Hardcore System,所以在理论上马虎一点也情有可原。想看理论方面的就推荐清华大学出版社《操作系统》吧,高教司司长张尧学写的,我们教材用的是那本。 另外推荐一本《Windows操作系统原理》机械工业出版社的,这本书是我国操作系统专家在微软零距离考察半年,写作历时一年多写成的,教操作系统的专家除了清华大学的张尧学(现高教司司长)几乎所有人都参加了。Bill Gates亲自写序。里面不但结合windows2000,xp详述操作系统的内核,而且后面讲了一些windows编程基础,有外版书的味道,而且上面一些内容可以说在国内外只有那本书才有对windows内核细致入微的介绍,
如果先把形式语言学好了,则编译原理中的前端我看只要学四个算法:最容易实现的递归下降;最好的自顶向下算法LL(k);最好的自底向上算法LR(k);LR(1)的简化SLR(也许还有另一简化LALR)。后端完全属于工程性质,自然又是another story。
推荐教材:Kenneth C.Louden写的“Compiler Construction Principles and Practice”即是《编译原理及实践》(机械工业出版社的译本)
学数据库要提醒大家的是,会用VFP,VB, Power builder不等于懂数据库。(这世界上自以为懂数据库的人太多了!)数据库设计既是科学又是艺术,数据库实现则是典型的工程。所以从某种意义上讲,数据库是最典型的一门计算机课程——理工结合,互相渗透。另外推荐大家学完软件工程学后再翻过来看看数据库技术,又会是一番新感觉。推荐教材:Abraham Silberschatz等著的 "Database System Concepts".作为知识的完整性,还推荐大家看一看机械工业出版社的《数据仓库》译本。

计算机网络的标准教材还是来自 Tanenbaum的《Computer Networks》(清华大学有译本)。还有就是推荐谢希仁的《计算机网络教程》(人民邮电出版社)问题讲得比较清楚,参考文献也比较权威。不过,网络也属于Hardcore System,所以光看书是不够的。建议多读RFC,http://www.ietf.org/rfc.htm里可以按编号下载RFC文档。从IP的读起。等到能掌握10种左右常用协议,就没有几个人敢小看你了。再做的工作我看放在网络设计上就比较好了。

数据结构的重要性就不言而喻了,学完数据结构你会对你的编程思想进行一番革命性的洗礼,会对如何建立一个合理高效的算法有一个清楚的认识。对于算法的建立我想大家应当注意以下几点:
当遇到一个算法问题时,首先要知道自己以前有没有处理过这种问题.如果见过,那么你一般会顺利地做出来;如果没见过,那么考虑以下问题:
1. 问题是否是建立在某种已知的熟悉的数据结构(例如,二叉树)上?如果不是,则要自己设计数据结构。
2. 问题所要求编写的算法属于以下哪种类型?(建立数据结构,修改数据结构,遍历,查找,排序...)
3. 分析问题所要求编写的算法的数学性质.是否具备递归特征?(对于递归程序设计,只要设计出合理的参数表以及递归结束的条件,则基本上大功告成.)
4. 继续分析问题的数学本质.根据你以前的编程经验,设想一种可能是可行的解决办法,并证明这种解决办法的正确性.如果题目对算法有时空方面的要求,证明你的设想满足其要求.一般的,时间效率和空间效率难以兼得.有时必须通过建立辅助存储的方法来节省时间.
5. 通过一段时间的分析,你对解决这个问题已经有了自己的一些思路.或者说,你已经可以用自然语言把你的算法简单描述出来.继续验证其正确性,努力发现其中的错误并找出解决办法.在必要的时候(发现了无法解决的矛盾),推翻自己的思路,从头开始构思.
6. 确认你的思路可行以后,开始编写程序.在编写代码的过程中,尽可能把各种问题考虑得详细,周密.程序应该具有良好的结构,并且在关键的地方配有注释.
7. 举一个例子,然后在纸上用笔执行你的程序,进一步验证其正确性.当遇到与你的设想不符的情况时,分析问题产生的原因是编程方面的问题还是算法思想本身有问题.
8. 如果程序通过了上述正确性验证,那么在将其进一步优化或简化。
9. 撰写思路分析,注释.
对于具体的算法思路, 只能靠你自己通过自己的知识和经验来加以获得,没有什么特定的规律(否则程序员全部可以下岗了,用机器自动生成代码就可以了).要有丰富的想象力,就是说当一条路走不通时,不要钻牛角尖,要敢于推翻自己的想法.我也只不过是初学者,说出上面的一些经验,仅供大家参考和讨论。
关于人工智能,我觉得的也是非常值得大家仔细研究的,虽然不能算是刚刚兴起的学科了,但是绝对是非常有发展前途的一门学科。我国人工智能创始人之一,北京科技大学涂序彦教授(这老先生是我的导师李小坚博士的导师)对人工智能这样定义:人工智能是模仿、延伸和扩展人与自然的智能的技术科学。在美国人工智能官方教育网站上对人工智能作了如下定义:Artificial Intelligence, or AI for short, is a combination of computer science, physiology, and philosophy. AI is a broad topic, consisting of different fields, from machine vision to expert systems. The element that the fields of AI have in common is the creation of machines that can "think".
这门学科研究的问题大概说有:
(1)符号主义: 符号计算与程序设计基础,知识表达方法 :知识与思维,产生式规则,语意网络,一阶谓词逻辑问题求解方法:搜索策略,启发式搜寻,搜寻算法,问题规约方法,谓词演算:归结原理,归结过程专家系统:建立专家系统的方法及工具
(2)联接主义(神经网络学派):1988年美国权威机构指出:数据库,网络发展呈直线上升,神经网络可能是解决人工智能的唯一途径。
我想对于人工智能的学习,大家一定不要像学数学似的及一些现成的结论,要学会分析问题,最好能利用程序设计实现,这里推荐给大家ACM最佳博士论文奖获得者涂晓媛博士的著作《人工鱼—计算机动画的人工生命方法》(清华大学出版社)。搞人工生命的同学不会不知道国际知名的涂氏父女吧。关于人工智能的书当然首选《Artificial Intelligence A New Synthesis》Nils J.Nilsson.鼻祖嘛!
关于网络安全我也想在这里说两句,随着计算机技术的发展,整个社会的信息化水平突飞猛进,计算机网络技术日新月异,网络成了当即社会各个工作领域不可缺少的组成部分,只要有网络存在,网络安全问题就是一个必须解决好的问题,学习网络安全不是简简单单的收集一些黑客工具黑一黑别人的网站,而是要学习他的数学原理,实现原理,搞清底层工作机制,这样才能解决大部分的现有问题和新出现的安全问题。

关于计算机科学的一些边缘科学我想谈一谈软件工程技术,对于一个企业,推出软件是不是就是几个程序员坐在一起,你写一段程序,我写一段程序呢?显然不是。软件工程是典型的计算机科学和数学,管理科学,心理学,社会学等学科的综合。它使我们这些搞理论和技术的人进入了一个社会。你所要考虑的不仅仅是程序的优劣,更应该考虑程序与软件的区别,软件与软件产品的区别,软件软件产品的市场前景,如何去更好的与人交流。这方面我还在学习阶段,以后这方面再写文章吧,先推荐给大家几本书:畅销20年不衰的《人月神话》(清华大学中文版,中国电力出版社影印版),《软件工程-实践者研究的方法》(机械工业出版社译本),《人件》(据说每一位微软公司的部门经理都读过这本书,推荐老总们和想当老总的同学都看看,了解一下什么是软件企业中的人)以及微软公司的《软件开发的科学与艺术》和《软件企业的管理与文化》(研究软件企业的制胜之道当然要研究微软的成功经验了!)

关于计算机技术的学习我想是这样的:学校开设的任何一门科学都有其滞后性,不要总认为自己掌握的某门技术就已经是天下无敌手了,虽然现在Java,VB,C,C++用的都很多,怎能保证没有被淘汰的一天,我想.NET平台的诞生和X\#语言的初见端倪完全可以说明问题。换言之,在我们掌握一门新技术的同时就又有更新的技术产生,身为当代的大学生应当有紧跟科学发展的素质。举个例子,就像有些同学总说,我做网页设计就喜欢直接写html,不愿意用什么Frontpage,Dreamweaver。能用语言写网页固然很好,但有高效的手段你为什么不使呢?仅仅是为了显示自己的水平高,unique? 我看真正水平高的是能够以最快的速度接受新事物的人。高级程序设计语言的发展日新月异,今后的程序设计就像人们在说话一样,我想大家从xml中应是有所体会了。难道我们真就写个什么都要用汇编,以显示自己的水平高,真是这样倒不如直接用机器语言写算了。反过来说,想要以最快的速度接受并利用新技术关键还是在于你对计算机科学地把握程度。

计算机技术牵扯的内容更为广泛些,一项一项说恐怕没个一年半载也说不清。我只想提醒大家的还是那句话,技术与科学是不能分家的,学好了科学同时搞技术,这才是上上策。犹如英语,原先人们与老外交流必须要个翻译,现在满马路的人都会说英语。就连21世纪英语演讲比赛的冠军都轮不到英语系的学生了。计算机也是一样的,我们必须面对的一个现实就是:计算机真就只是一个工具,如果不具备其它方面的素养,计算机系的学生虽然不能说找不到工作,不过总有一天当其他专业性人才掌握了计算机技术后将比我们出色许多。原因就在于计算机解决的大都是实际问题,实际问题的知识却是我们少有的。单一的计算机技术没有立足之地。
我想是时候指出:学习每一个课程之前,都要先搞清这一课程的学习目的。这一学科的应用领域。据我自身所了解到的同龄同学和低年级的同学的学习状况:他们之中很少有人知道学一个学科的学习目的,期末考试结束了也不知道学这科做什么用。这就失去了读计算机科学的意义。当然这与现存的教育思想不能说一点关系都没有。
总的来说,从教育角度来讲,国内高校的课程安排不是很合理,强调理论,又不愿意在理论上深入教育,无力接受新技术,想避开新技术又无法避得一干二净。我觉得关键问题就是国内的高校难于突破现状,条条框框限制着怎么求发展。我们虽然认识得到国外教育的优越性,但为什么迟迟不能采取行动?哪怕是去粗取精的取那么一点点。我们需要改变。从我们自身角度来讲,多数人4年下来既没有学习计算机科学的学术水平,也没有学习计算机技术的那种韧劲。在我刚上大一时,我的计算机科学入门导师,淮北煤炭师范学院王爱平教授曾经对我说过这样一番话:“当你选择了计算机这一门科学,就意味着你踏上了一条不归路,就意味着你一生都要为之奋斗……你的身后是悬崖,只有向前走,不能往后退。”
有些同学说按照这样学习学的东西太多,有的未见得有用,我想打个形象的比方:学校学出来的人都是一个球体,方方面面的知识都应具备。可是社会上需要球体的地方很少,反而需要的是砖和瓦,即精通某一行的人才。但是对于同等体积的物体,用球体来改造是最方便最省事的。学校的学生很多,为了能够使更多的学生来适应这个社会,学校也就不得以把所有的学生都打造成一个球体,然后让社会对这些学生进行再加工,成为真正能够有用的人才。即使你非常清楚自己的将来要干什么,并且非常下定决心要走自己的路,这一步你也必须走,世界是在不断变化的,你不能预料未来。想清楚,努力去干吧!
必须结束这篇“胡侃”了,再侃下去非我力所能及。其实计算机还有很多基础课都值得一侃。怎奈我造诣有限,不敢再让内行耻笑。对于博大精深的计算机科学,我只能说我永远都是个Beginner.最后声明:这些只针对本科阶段的学习。即使把这些全弄通了,前面的路还长,计算机科学需要我们为之奋斗......学习计算机科学需要韧性,更需要创新,需要激情。深刻学习理论知识,勇于接受新技术的挑战,这才是我们这一代人应具有的素质。最后送大家一句话“Wake up every day with a feeling of passion for the difference technology will make in people's life!”。

在我大一时无意中找到了南京大学网友sir的帖子“胡侃(理论)计算机学习”,这个帖子对我的大学生活起了至关重要的作用,也因此同他成为了好友,本帖子在原有帖子的基础上改进了其中我认为不太合适的理论,修正了一些观点,在推荐教材方面结合我的学习情况有了较大改变。值得一提的是增加了一些计算机理论的内容,计算机技术的内容结合我国的教学情况和我们学习的实际情况进行了重写。感谢大家的支持,这篇文章才能比较快的完成,这里也只是写下了我在学习计算机科学时的所思所想,很不成熟。与原文相比增加了一些推荐参考书,删去了一些过陈旧的难以找到的材料。并且对一些问题作了更为详细地阐述,也增加了一些新观点。希望大家多多讨论,改进不足,让我们共同努力吧!
作者: 国民    时间: 2010-8-28 14:49

马上就是大二的学生了!目前呢,能够理解的就是:1,每一个人(包括非计算机专业),掌握简单的计算机技术都很容易(包括程序设计),但计算机专业的优势就在于,我们掌握许多其他专业并不“深究”的东西,例如,算法,体系结构,等等。2,数学是从实际生活当中抽象出来的理论,人们之所以要将实际抽象成理论,目的就在于想用抽象出来的理论去更好的指导实践,3,政治课上学的理论联系实际是指导学习科学文化知识的航标4,数学分析,在于它是偏向于证明型的数学课程,对我们培养良好的分析能力极有帮助。5,对计算机系学生而言,追求一堆曲面积分的公式,不能算懂了数学!6,将抽象的理论再应用于实践,不但要掌握题目的解题方法,更要掌握解题思想,对于定理的学习:不是简单的应用,而是掌握证明过程即掌握定理的由来,训练自己的推理能力。只有这样才达到了学习这门科学的目的,同时也缩小了我们与数学系的同学之间思维上的差距。7,计算机科学理论的根本,在于算法.应该说,也就是我所想到的逻辑推理能力吧!8,一个一流计算机系的优秀学生决不该仅仅是一个编程高手,但他一定首先是一个编程高手。基于此,问题又都回到了前面以及后面的说明了!9,真正水平高的是能够以最快的速度接受新事物的人。11,计算机真就只是一个工具,如果不具备其它方面的素养,总有一天当其他专业性人才掌握了计算机技术后将比我们出色许多。原因就在于计算机解决的大都是实际问题,实际问题的知识却是我们少有的。单一的计算机技术没有立足之地。12,国内高校的课程安排不是很合理,强调理论,又不愿意在理论上深入教育,无力接受新技术,想避开新技术又无法避得一干二净。这样的问题,只能是达人知命了!
不能理解的当然包括你所提到的一些专有的名词!除此以外,就是想要了解一下,1,在一个二本的大学里,有什么理由去学习这些课程(当然是指那些教育部规定的,普通大学都会开的课程)?2,在时间的安排上,我们不见的有多少优势?3,在我们大二大三编写的程序中含金量最大的一块是什么?是我们蕴含在其中的一些算法?还是能准确的运用那些编程语法?谢谢了!谢谢你的”胡侃“!!!!!!





\chapter{计算机科学与技术学习反思录}

计算机科学与技术这一门科学深深的吸引着我们这些同学们,上计算机系已经有近三年了,自己也做了一些思考,我一直认为计算机科学与技术这门专业,在本科阶段是不可能切分成计算机科学和计算机技术的,因为计算机科学需要相当多的实践,而实践需要技术;每一个人(包括非计算机专业),掌握简单的计算机技术都很容易(包括程序设计),但计算机专业的优势就在于,我们掌握许多其他专业并不“深究”的东西,例如,算法,体系结构,等等。非计算机专业的人可以很容易地做一个芯片,写一段程序,但他们做不出计算机专业能够做出来的大型系统。(与司徒彦南兄的谈话)今天我想专门谈一谈计算机科学,并将重点放在计算理论上。


计算机理论的一个核心问题--从数学谈起:


记得当年大一入学,每周六课时高等数学,天天作业不断(那时是六日工作制)。颇有些同学惊呼走错了门:咱们这到底念的是什么系?不错,你没走错门,这就是计算机科学与技术系。我国计算机科学系里的传统是培养做学术研究,尤其是理论研究的人(方向不见得有问题,但是做得不是那么尽如人意)。而计算机的理论研究,说到底了,如网络安全,图形图像学,视频音频处理,哪个方向都与数学有着很大的关系,虽然也许是正统数学家眼里非主流的数学。这里我还想阐明我的一个观点:我们都知道,数学是从实际生活当中抽象出来的理论,人们之所以要将实际抽象成理论,目的就在于想用抽象出来的理论去更好的指导实践,有些数学研究工作者喜欢用一些现存的理论知识去推导若干条推论,殊不知其一:问题考虑不全很可能是个错误的推论,其二:他的推论在现实生活中找不到原型,不能指导实践。严格的说,我并不是一个理想主义者,政治课上学的理论联系实际一直是指导我学习科学文化知识的航标(至少我认为搞计算机科学与技术的应当本着这个方向)。


其实我们计算机系学数学光学高等数学是不够的(典型的工科院校一般都开的是高等数学),我们应该像数学系一样学一下数学分析(清华计算机系开的好像就是数学分析),数学分析这门科学,咱们学计算机的人对它有很复杂的感情。在于它是偏向于证明型的数学课程,这对我们培养良好的分析能力极有帮助。我的软件工程学导师北工大数理学院的王仪华先生就曾经教导过我们,数学系的学生到软件企业中大多作软件设计与分析工作,而计算机系的学生做程序员的居多,原因就在于数学系的学生分析推理能力,从所受训练的角度上要远远在我们之上。当年出现的怪现象是:计算机系学生的高中数学基础在全校数一数二(希望没有冒犯其它系的同学),教学课时数也仅次于数学系,但学完之后的效果却不尽如人意。难道都是学生不努力吗,我看未见得,方向错了也说不一定,其中原因何在,发人深思。


我个人的浅见是:计算机系的学生,对数学的要求固然跟数学系不同,跟物理类差别则更大。通常非数学专业的所谓“高等数学”,无非是把数学分析中较困难的理论部分删去,强调套用公式计算而已。而对计算机系来说,数学分析里用处最大的恰恰是被删去的理论部分。说得难听一点,对计算机系学生而言,追求算来算去的所谓“工程数学”已经彻底地走进了误区。记上一堆曲面积分的公式,难道就能算懂了数学?那倒不如现用现查,何必费事记呢?再不然直接用Mathematics或是Matalab好了。


我在系里最爱做的事情就是给学弟学妹们推荐参考书。中文的数学分析书,一般都认为以北大张筑生老师的“数学分析新讲”为最好。万一你的数学实在太好,那就去看菲赫金哥尔茨的“微积分学教程”好了--但我认为没什么必要,毕竟你不想转到数学系去。吉米多维奇的“数学分析习题集”也基本上是计算型的东东。书的名气很大,倒不见得适合我们,还是那句话,重要的是数学思想的建立,生活在信息社会里我们求的是高效,计算这玩意还是留给计算机吧。不过现在多用的似乎是复旦大学的《数学分析》也是很好的教材。


中国的所谓高等代数,就等于线性代数加上一点多项式理论。我以为这有好的一面,因为可以让学生较早感觉到代数是一种结构,而非一堆矩阵翻来覆去。这里不得不提南京大学林成森,盛松柏两位老师编的“高等代数”,感觉相当舒服。此书相当全面地包含了关于多项式和线性代数的基本初等结果,同时还提供了一些有用的又比较深刻的内容,如Sturm序列,Shermon-Morrison公式,广义逆矩阵等等。可以说,作为本科生如能吃透此书,就可以算高手。国内较好的高等代数教材还有清华计算机系用的那本,清华出版社出版,书店里多多,一看就知道。从抽象代数的观点来看,高等代数里的结果不过是代数系统性质的一些例子而已。莫宗坚先生的《代数学》里,对此进行了深刻的讨论。然而莫先生的书实在深得很,作为本科生恐怕难以接受,不妨等到自己以后成熟了一些再读。


正如上面所论述的,计算机系的学生学习高等数学:知其然更要知其所以然。你学习的目的应该是:将抽象的理论再应用于实践,不但要掌握题目的解题方法,更要掌握解题思想,对于定理的学习:不是简单的应用,而是掌握证明过程即掌握定理的由来,训练自己的推理能力。只有这样才达到了学习这门科学的目的,同时也缩小了我们与数学系的同学之间思维上的差距。


概率论与数理统计这门课很重要,可惜大多数院校讲授这门课都会少些东西。少了的东西现在看至少有随机过程。到毕业还没有听说过Markov过程,此乃计算机系学生的耻辱。没有随机过程,你怎么分析网络和分布式系统?怎么设计随机化算法和协议?据说清华计算机系开有“随机数学”,早就是必修课。另外,离散概率论对计算机系学生来说有特殊的重要性。而我们国家工程数学讲的都是连续概率。现在,美国已经有些学校开设了单纯的“离散概率论”课程,干脆把连续概率删去,把离散概率讲深些。我们不一定要这么做,但应该更加强调离散概率是没有疑问的。这个工作我看还是尽早的做为好。


计算方法学(有些学校也称为数学分析学)是最后一门由数理学院给我们开的课。一般学生对这门课的重视程度有限,以为没什么用。不就是照套公式嘛!其实,做图形图像可离不开它,密码学搞深了也离不开它。而且,在很多科学工程中的应用计算,都以数值的为主。这门课有两个极端的讲法:一个是古典的“数值分析”,完全讲数学原理和算法;另一个是现在日趋流行的“科学与工程计算”,干脆教学生用软件包编程。我个人认为,计算机系的学生一定要认识清楚我们计算机系的学生为什么要学这门课,我是很偏向于学好理论后用计算机实现的,最好使用C语言或C++编程实现。向这个方向努力的书籍还是挺多的,这里推荐大家高等教育出版社(CHEP)和施普林格出版社(Springer)联合出版的《计算方法(Computational Methods)》,华中理工大学数学系写的(现华中科技大学),这方面华科大做的工作在国内应算是比较多的,而个人认为以这本最好,至少程序设计方面涉及了:任意数学函数的求值,方程求根,线性方程组求解,插值方法,数值积分,场微分方程数值求解。李庆扬的那本则理论性过强,与实际应用结合得不太紧。 
每个学校本系里都会开一门离散数学,涉及集合论,图论,和抽象代数,数理逻辑。不过,这么多内容挤在离散数学一门课里,是否时间太紧了点?另外,计算机系学生不懂组合和数论,也是巨大的缺陷。要做理论,不懂组合或者数论吃亏可就太大了。从理想的状态来看,最好分开六门课:集合,逻辑,图论,组合,代数,数论。这个当然不现实,因为没那么多课时。也许将来可以开三门课:集合与逻辑,图论与组合,代数与数论。(这方面我们学校已经着手开始做了)不管课怎么开,学生总一样要学。下面分别谈谈上面的三组内容。


古典集合论,北师大出过一本《基础集合论》不错。


数理逻辑,中科院软件所陆钟万教授的《面向计算机科学的数理逻辑》就不错。现在可以找到陆钟万教授的讲课录像,http://www.cas.ac.cn/html/Dir/2001/11/06/3391.htm自己去看看吧。总的来说,学集合/逻辑起手不难,普通高中生都能看懂。但越往后越感觉深不可测。


学完以上各书之后,如果你还有精力兴趣进一步深究,那么可以试一下GTM系列中的《Introduction to Axiomatic Set Theory》和《A Course of Mathematical Logic》。这两本都有世界图书出版社的引进版。你如果能搞定这两本,可以说在逻辑方面真正入了门,也就不用再浪费时间听我瞎侃了。


据说全中国最多只有三十个人懂图论。此言不虚。图论这东东,技巧性太强,几乎每个问题都有一个独特的方法,让人头痛。不过这也正是它魅力所在:只要你有创造性,它就能给你成就感。我的导师说,图论里面随便揪一块东西就可以写篇论文。大家可以体会里面内容之深广了吧!国内的图论书中,王树禾老师的“图论及其算法”非常成功。一方面,其内容在国内教材里算非常全面的。另一方面,其对算法的强调非常适合计算机系(本来就是科大计算机系教材)。有了这本书为主,再参考几本翻译的,如Bondy \& Murty的《图论及其应用》,人民邮电出版社翻译的《图论和电路网络》等等,就马马虎虎,对本科生足够了。再进一步,世界图书引进有GTM系列的"Modern Graph Theory"。此书确实经典!国内好象还有一家出版了个翻译版。不过,学到这个层次,还是读原版好。搞定这本书,也标志着图论入了门。


离散数学方面我们北京工业大学实验学院有个世界级的专家,叫邵学才,复旦大学概率论毕业的,教过高等数学,线性代数,概率论,最后转向离散数学,出版著作无数,论文集新加坡有一本,堪称经典,大家想学离散数学的真谛不妨找来看看。这老师的课我专门去听过,极为经典。不过你要从他的不经意的话中去挖掘精髓。在同他的交谈当中我又深刻地发现一个问题,虽说邵先生写书无数,但依他自己的说法每本都差不多,我实在觉得诧异,他说主要是有大纲的限制,不便多写。这就难怪了,很少听说国外写书还要依据个什么大纲(就算有,内容也宽泛的多),不敢越雷池半步,这样不是看谁的都一样了。外版的书好就好在这里,最新的科技成果里面都有论述,别的先不说,至少是“紧跟时代的理论知识”。


组合感觉没有太适合的国产书。还是读Graham和Knuth等人合著的经典“具体数学”吧,西安电子科技大学出版社有翻译版。


抽象代数,国内经典为莫宗坚先生的“代数学”。此书是北大数学系教材,深得好评。然而对本科生来说,此书未免太深。可以先学习一些其它的教材,然后再回头来看“代数学”。国际上的经典可就多了,GTM系列里就有一大堆。推荐一本谈不上经典,但却最简单的,最容易学的:http://www.math.miami.edu/~ec/book/这本“Introduction to Linear and Abstract Algebra"非常通俗易懂,而且把抽象代数和线性代数结合起来,对初学者来说非常理想,我校比较牛的同学都有收藏。


数论方面,国内有经典而且以困难著称的”初等数论“(潘氏兄弟著,北大版)。再追溯一点,还有更加经典(可以算世界级)并且更加困难的”数论导引“(华罗庚先生的名著,科学版,九章书店重印,繁体的看起来可能比较困难)。把基础的几章搞定一个大概,对本科生来讲足够了。但这只是初等数论。本科毕业后要学计算数论,你必须看英文的书,如Bach的"Introduction to Algorithmic Number Theory"。


计算机科学理论的根本,在于算法。现在很多系里给本科生开设算法设计与分析,确实非常正确。环顾西方世界,大约没有一个三流以上计算机系不把算法作为必修的。算法教材目前公认以Corman等著的"Introduction to Algorithms"为最优。对入门而言,这一本已经足够,不需要再参考其它书。


再说说形式语言与自动机。我看过北邮的教材,应该说写的还清楚。但是,有一点要强调:形式语言和自动机的作用主要在作为计算模型,而不是用来做编译。事实上,编译前端已经是死领域,没有任何open problems,北科大的班晓娟博士也曾经说过,编译的技术已相当成熟。如果为了这个,我们完全没必要去学形式语言--用用yacc什么的就完了。北邮的那本在国内还算比较好,但是在深度上,在跟可计算性的联系上都有较大的局限,现代感也不足。所以建议有兴趣的同学去读英文书,不过国内似乎没引进这方面的教材。可以去互动出版网上看一看。入门以后,把形式语言与自动机中定义的模型,和数理逻辑中用递归函数定义的模型比较一番,可以说非常有趣。现在才知道,什么叫“宫室之美,百官之富”!


计算机科学和数学的关系有点奇怪。二三十年以前,计算机科学基本上还是数学的一个分支。而现在,计算机科学拥有广泛的研究领域和众多的研究人员,在很多方面反过来推动数学发展,从某种意义上可以说是孩子长得比妈妈还高了。但不管怎么样,这个孩子身上始终流着母亲的血液。这血液是the mathematical underpinning of computer science(计算机科学的数学基础),也就是理论计算机科学。原来在东方大学城图书馆中曾经看过一本七十年代的译本(书皮都没了,可我就爱关注这种书),大概就叫《计算机数学》。那本书若是放在当时来讲决是一本好书,但现在看来,涵盖的范围还算广,深度则差了许多,不过推荐大一的学生倒可以看一看,至少可以使你的计算数学入入门。


最常和理论计算机科学放在一起的一个词是什么?答:离散数学。这两者的关系是如此密切,以至于它们在不少场合下成为同义词。(这一点在前面的那本书中也有体现)传统上,数学是以分析为中心的。数学系的同学要学习三四个学期的数学分析,然后是复变函数,实变函数,泛函数等等。实变和泛函被很多人认为是现代数学的入门。在物理,化学,工程上应用的,也以分析为主。


随着计算机科学的出现,一些以前不太受到重视的数学分支突然重要起来。人们发现,这些分支处理的数学对象与传统的分析有明显的区别:分析研究的问题解决方案是连续的,因而微分,积分成为基本的运算;而这些分支研究的对象是离散的,因而很少有机会进行此类的计算。人们从而称这些分支为“离散数学”。“离散数学”的名字越来越响亮,最后导致以分析为中心的传统数学分支被相对称为“连续数学”。


离散数学经过几十年发展,基本上稳定下来。一般认为,离散数学包含以下学科 :

1) 集合论,数理逻辑与元数学。这是整个数学的基础,也是计算机科学的基础。

2) 图论,算法图论;组合数学,组合算法。计算机科学,尤其是理论计算机科学的核心是 
算法,而大量的算法建立在图和组合的基础上。

3) 抽象代数。代数是无所不在的,本来在数学中就非常重要。在计算机科学中,人们惊讶地发现代数竟然有如此之多的应用。


但是,理论计算机科学仅仅就是在数学的上面加上“离散”的帽子这么简单吗?一直到大约十几年前,终于有一位大师告诉我们:不是。D.E.Knuth(他有多伟大,我想不用我废话了)在Stanford开设了一门全新的课程Concrete Mathematics。 Concrete这个词在这里有两层含义:


首先:对abstract而言。Knuth认为,传统数学研究的对象过于抽象,导致对具体的问题关心不够。他抱怨说,在研究中他需要的数学往往并不存在,所以他只能自己去创造一些数学。为了直接面向应用的需要,他要提倡“具体”的数学。在这里我做一点简单的解释。例如在集合论中,数学家关心的都是最根本的问题--公理系统的各种性质之类。而一些具体集合的性质,各种常见集合,关系,映射都是什么样的,数学家觉得并不重要。然而,在计算机科学中应用的,恰恰就是这些具体的东西。Knuth能够首先看到这一点,不愧为当世计算机第一人。其次,Concrete是Continuous(连续)加上discrete(离散)。不管连续数学还是离散数学,都是有用的数学!


理论与实际的结合--计算机科学研究的范畴


前面主要是从数学角度来看的。从计算机角度来看,理论计算机科学目前主要的研究领域包括:可计算性理论,算法设计与复杂性分析,密码学与信息安全,分布式计算理论,并行计算理论,网络理论,生物信息计算,计算几何学,程序语言理论等等。这些领域互相交叉,而且新的课题在不断提出,所以很难理出一个头绪来。想搞搞这方面的工作,推荐看中国计算机学会的一系列书籍,至少代表了我国的权威。下面随便举一些例子。


由于应用需求的推动,密码学现在成为研究的热点。密码学建立在数论(尤其是计算数论),代数,信息论,概率论和随机过程的基础上,有时也用到图论和组合学等。很多人以为密码学就是加密解密,而加密就是用一个函数把数据打乱。这样的理解太浅显了。


现代密码学至少包含以下层次的内容:

第一,密码学的基础。例如,分解一个大数真的很困难吗?能否有一般的工具证明协议正确?

第二,密码学的基本课题。例如,比以前更好的单向函数,签名协议等。

第三,密码学的高级问题。例如,零知识证明的长度,秘密分享的方法。

第四,密码学的新应用。例如,数字现金,叛徒追踪等。


在分布式系统中,也有很多重要的理论问题。例如,进程之间的同步,互斥协议。一个经典的结果是:在通信信道不可靠时,没有确定型算法能实现进程间协同。所以,改进TCP三次握手几乎没有意义。例如时序问题。常用的一种序是因果序,但因果序直到不久前才有一个理论上的结果....例如,死锁没有实用的方法能完美地对付。例如,......操作系统研究过就自己去举吧!


如果计算机只有理论,那么它不过是数学的一个分支,而不成为一门独立的科学。事实上,在理论之外,计算机科学还有更广阔的天空。


我一直认为,4年根本不够学习计算机的基础知识,因为面太宽了,8年,应该差不多了......


这方面我想先说说我们系在各校普遍开设的《计算机基础》。在高等学校开设《计算机基础课程》是我国高教司明文规定的各专业必修课程要求。主要内容是使学生初步掌握计算机的发展历史,学会简单的使用操作系统,文字处理,表格处理功能和初步的网络应用功能。但是在计算机科学系教授此门课程的目标决不能与此一致。在计算机系课程中目标应是:让学生较为全面的了解计算机学科的发展,清晰的把握计算机学科研究的方向,发展的前沿即每一个课程在整个学科体系中所处的地位。搞清各学科的学习目的,学习内容,应用领域。使学生在学科学习初期就对整个学科有一个整体的认识,以做到在今后的学习中清楚要学什么,怎么学。计算机基本应用技能的位置应当放在第二位或更靠后,因为这一点对于本系的学生应当有这个摸索能力。这一点很重要。推荐给大家一本书:机械工业出版社的《计算机文化》(New Perspective of Computer Science),看了这本书我才深刻的体会到自己还是个计算机科学初学者,才比较透彻的了解了什么是计算机科学。


一个一流计算机系的优秀学生决不该仅仅是一个编程高手,但他一定首先是一个编程高手。我上大学的时候,第一门专业课是C语言程序设计,念计算机的人从某种角度讲相当一部分人是靠写程序吃饭的。在我们北京工业大学实验学院计算机系里一直有这样的争论(时至今日CSDN上也有),关于第一程序设计语言该用哪一种。我个人认为,用哪种语言属于末节,关键在养成良好的编程习惯。当年老师对我们说,打好基础后学一门新语言只要一个星期。现在我觉得根本不用一个星期,前提是先把基础打好。不要再犹豫了,学了再说,等你抉择好了,别人已经会了几门语言了。


汇编语言和微机原理是两门特烦人的课。你的数学/理论基础再好,也占不到什么便宜。这两门课之间的次序也好比先有鸡还是先有蛋,无论你先学哪门,都会牵扯另一门课里的东西。所以,只能静下来慢慢琢磨。这就是典型的工程课,不需要太多的聪明和顿悟,却需要水滴石穿的渐悟。有关这两门课的书,计算机书店里不难找到。弄几本最新的,对照着看吧。组成原理推荐《计算机组成与结构》清华大学王爱英教授写的。汇编语言大家拿8086/8088入个门,之后一定要学80x86汇编语言。实用价值大,不落后,结构又好,写写高效病毒,高级语言里嵌一点汇编,进行底层开发,总也离不开他,推荐清华大学沈美明的《IBM-PC汇编语言程序设计》。有些人说不想了解计算机体系结构,也不想制造计算机,所以诸如计算机原理,汇编语言,接口之类的课觉得没必要学,这样合理吗?显然不合理,这些东西迟早得掌握,肯定得接触,而且,这是计算机专业与其他专业学生相比的少有的几项优势。做项目的时候,了解这些是非常重要的,不可能说,仅仅为了技术而技术,只懂技术的人最多做一个编码工人,而永远不可能全面地了解整个系统的设计,而编码工人是越老越不值钱。关于组成原理还有个讲授的问题,在我学这门课程时老师讲授时把CPU工作原理誉微程序设计这一块略掉了,理由是我们国家搞CPU技术不如别的国家,搞了这么长时间好不容易出了个龙芯比Intel的还差个十万八千里,所以建议我们不要学了。我看这在各校也未见得不是个问题吧!若真是如他所说,那中国的计算机科学哪个方向都可以停了,软硬件,应用,有几项搞得过美国,搞不过别人就不搞了,那我们坐在这里干什么?教学的观念需要转变的。


模拟电路这东东,如今不仅计算机系学生搞不定,电子系学生也多半害怕。如果你真想软硬件通吃,那么建议你先看看邱关源的“电路原理”,也许此后再看模拟电路底气会足些。教材:康华光的“电子技术基础”(高等教育出版社)还是不错的(我校电子系在用)。有兴趣也可以参考童诗白的书。


数字电路比模拟电路要好懂得多。推荐大家看一看我们北工大刘英娴教授写的《数字逻辑》业绩人士都说这本书很有参考价值(机械工业出版社的)。原因很明了,实用价值高,能听听她讲授的课程更是有一种“享受科学”的感觉。清华大学阎石的书也算一本好教材,遗憾的一点是集成电路讲少了些。真有兴趣,看一看大规模数字系统设计吧(北航那本用的还比较多)。


计算机系统结构该怎么教,国际上还在争论。国内能找到的较好教材为Stallings的"Computer Organization and Architecture:Designing for Performance"(清华影印 
本)。国际上最流行的则是“Computer architecture: aquantitative approach", by Patterson \& Hennessy。


操作系统可以随便选用《操作系统的内核设计与实现》和《现代操作系统》两书之一。这两部都可以算经典,唯一缺点就是理论上不够严格。不过这领域属于Hardcore System,所以在理论上马虎一点也情有可原。想看理论方面的就推荐清华大学出版社《操作系统》吧,高教司司长张尧学写的,我们教材用的是那本。 另外推荐一本《Windows操作系统原理》机械工业出版社的,这本书是我国操作系统专家在微软零距离考察半年,写作历时一年多写成的,教操作系统的专家除了清华大学的张尧学(现高教司司长)几乎所有人都参加了。Bill Gates亲自写序。里面不但结合windows2000,xp详述操作系统的内核,而且后面讲了一些windows编程基础,有外版书的味道,而且上面一些内容可以说在国内外只有那本书才有对windows内核细致入微的介绍,


如果先把形式语言学好了,则编译原理中的前端我看只要学四个算法:最容易实现的递归下降;最好的自顶向下算法LL(k);最好的自底向上算法LR(k);LR(1)的简化SLR(也许还有另一简化LALR)。后端完全属于工程性质,自然又是another story。


推荐教材:Kenneth C.Louden写的“Compiler Construction Principles and Practice”即是《编译原理及实践》(机械工业出版社的译本)


学数据库要提醒大家的是,会用VFP,VB, Power builder不等于懂数据库。(这世界上自以为懂数据库的人太多了!)数据库设计既是科学又是艺术,数据库实现则是典型的工程。所以从某种意义上讲,数据库是最典型的一门计算机课程--理工结合,互相渗透。另外推荐大家学完软件工程学后再翻过来看看数据库技术,又会是一番新感觉。推荐教材:Abraham Silberschatz等著的 "Database System Concepts".作为知识的完整性,还推荐大家看一看机械工业出版社的《数据仓库》译本。


计算机网络的标准教材还是来自Tanenbaum的《Computer Networks》(清华大学有译本)。还有就是推荐谢希仁的《计算机网络教程》(人民邮电出版社)问题讲得比较清楚,参考文献也比较权威。不过,网络也属于Hardcore System,所以光看书是不够的。建议多读RFC,http://www.ietf.org/rfc.htm里可以按编号下载RFC文档。从IP的读起。等到能掌握10种左右常用协议,就没有几个人敢小看你了。再做的工作我看放在网络设计上就比较好了。


数据结构的重要性就不言而喻了,学完数据结构你会对你的编程思想进行一番革命性的洗礼,会对如何建立一个合理高效的算法有一个清楚的认识。对于算法的建立我想大家应当注意以下几点:


当遇到一个算法问题时,首先要知道自己以前有没有处理过这种问题.如果见过,那么你一般会顺利地做出来;如果没见过,那么考虑以下问题:

1. 问题是否是建立在某种已知的熟悉的数据结构(例如,二叉树)上?如果不是,则要自己设计数据结构。

2. 问题所要求编写的算法属于以下哪种类型?(建立数据结构,修改数据结构,遍历,查找,排序...)

3. 分析问题所要求编写的算法的数学性质.是否具备递归特征?(对于递归程序设计,只要设计出合理的参数表以及递归结束的条件,则基本上大功告成.)

4. 继续分析问题的数学本质.根据你以前的编程经验,设想一种可能是可行的解决办法,并证明这种解决办法的正确性.如果题目对算法有时空方面的要求,证明你的设想满足其要求.一般的,时间效率和空间效率难以兼得.有时必须通过建立辅助存储的方法来节省时间.

5. 通过一段时间的分析,你对解决这个问题已经有了自己的一些思路.或者说,你已经可以用自然语言把你的算法简单描述出来.继续验证其正确性,努力发现其中的错误并找出解决办法.在必要的时候(发现了无法解决的矛盾),推翻自己的思路,从头开始构思.


6. 确认你的思路可行以后,开始编写程序.在编写代码的过程中,尽可能把各种问题考虑得详细,周密.程序应该具有良好的结构,并且在关键的地方配有注释.

7. 举一个例子,然后在纸上用笔执行你的程序,进一步验证其正确性.当遇到与你的设想不符的情况时,分析问题产生的原因是编程方面的问题还是算法思想本身有问题.

8. 如果程序通过了上述正确性验证,那么在将其进一步优化或简化。

9. 撰写思路分析,注释.


对于具体的算法思路,只能靠你自己通过自己的知识和经验来加以获得,没有什么特定的规律(否则程序员全部可以下岗了,用机器自动生成代码就可以了).要有丰富的想象力,就是说当一条路走不通时,不要钻牛角尖,要敢于推翻自己的想法.我也只不过是初学者,说出上面的一些经验,仅供大家参考和讨论。


关于人工智能,我觉得的也是非常值得大家仔细研究的,虽然不能算是刚刚兴起的学科了,但是绝对是非常有发展前途的一门学科。我国人工智能创始人之一,北京科技大学涂序彦教授(这老先生是我的导师李小坚博士的导师)对人工智能这样定义:人工智能是模仿、延伸和扩展人与自然的智能的技术科学。在美国人工智能官方教育网站上对人工智能作了如下定义:Artificial Intelligence, or AI for short, is a combination of computer science, physiology, and philosophy. AI is a broad topic, consisting of different fields, from machine vision to expert systems. The element that the fields of AI have in common is the creation of machines that can "think". 
这门学科研究的问题大概说有:

(1)符号主义: 符号计算与程序设计基础,知识表达方法 :知识与思维,产生式规则,语意网络,一阶谓词逻辑问题求解方法:搜索策略,启发式搜寻,搜寻算法,问题规约方法,谓词演算:归结原理,归结过程专家系统:建立专家系统的方法及工具

(2)联接主义(神经网络学派):1988年美国权威机构指出:数据库,网络发展呈直线上升,神经网络可能是解决人工智能的唯一途径。

我想对于人工智能的学习,大家一定不要像学数学似的及一些现成的结论,要学会分析问题,最好能利用程序设计实现,这里推荐给大家ACM最佳博士论文奖获得者涂晓媛博士的著作《人工鱼-计算机动画的人工生命方法》(清华大学出版社)。搞人工生命的同学不会不知道国际知名的涂氏父女吧。关于人工智能的书当然首选《Artificial Intelligence A New Synthesis》Nils J.Nilsson.鼻祖嘛!


关于网络安全我也想在这里说两句,随着计算机技术的发展,整个社会的信息化水平突飞猛进,计算机网络技术日新月异,网络成了当即社会各个工作领域不可缺少的组成部分,只要有网络存在,网络安全问题就是一个必须解决好的问题,学习网络安全不是简简单单的收集一些黑客工具黑一黑别人的网站,而是要学习他的数学原理,实现原理,搞清底层工作机制,这样才能解决大部分的现有问题和新出现的安全问题。


关于计算机科学的一些边缘科学我想谈一谈软件工程技术,对于一个企业,推出软件是不是就是几个程序员坐在一起,你写一段程序,我写一段程序呢?显然不是。软件工程是典型的计算机科学和数学,管理科学,心理学,社会学等学科的综合。它使我们这些搞理论和技术的人进入了一个社会。你所要考虑的不仅仅是程序的优劣,更应该考虑程序与软件的区别,软件与软件产品的区别,软件软件产品的市场前景,如何去更好的与人交流。这方面我还在学习阶段,以后这方面再写文章吧,先推荐给大家几本书:畅销20年不衰的《人月神话》(清华大学中文版,中国电力出版社影印版),《软件工程-实践者研究的方法》(机械工业出版社译本),《人件》(据说每一位微软公司的部门经理都读过这本书,推荐老总们和想当老总的同学都看看,了解一下什么是软件企业中的人)以及微软公司的《软件开发的科学与艺术》和《软件企业的管理与文化》(研究软件企业的制胜之道当然要研究微软的成功经验了!)


关于计算机技术的学习我想是这样的:学校开设的任何一门科学都有其滞后性,不要总认为自己掌握的某门技术就已经是天下无敌手了,虽然现在Java,VB,C,C++用的都很多,怎能保证没有被淘汰的一天,我想.NET平台的诞生和X\#语言的初见端倪完全可以说明问题。换言之,在我们掌握一门新技术的同时就又有更新的技术产生,身为当代的大学生应当有紧跟科学发展的素质。举个例子,就像有些同学总说,我做网页设计就喜欢直接写html,不愿意用什么Frontpage,Dreamweaver。能用语言写网页固然很好,但有高效的手段你为什么不使呢?仅仅是为了显示自己的水平高,unique? 我看真正水平高的是能够以最快的速度接受新事物的人。高级程序设计语言的发展日新月异,今后的程序设计就像人们在说话一样,我想大家从xml中应是有所体会了。难道我们真就写个什么都要用汇编,以显示自己的水平高,真是这样倒不如直接用机器语言写算了。反过来说,想要以最快的速度接受并利用新技术关键还是在于你对计算机科学地把握程度。


计算机技术牵扯的内容更为广泛些,一项一项说恐怕没个一年半载也说不清。我只想提醒大家的还是那句话,技术与科学是不能分家的,学好了科学同时搞技术,这才是上上策。犹如英语,原先人们与老外交流必须要个翻译,现在满马路的人都会说英语。就连21世纪英语演讲比赛的冠军都轮不到英语系的学生了。计算机也是一样的,我们必须面对的一个现实就是:计算机真就只是一个工具,如果不具备其它方面的素养,计算机系的学生虽然不能说找不到工作,不过总有一天当其他专业性人才掌握了计算机技术后将比我们出色许多。原因就在于计算机解决的大都是实际问题,实际问题的知识却是我们少有的。单一的计算机技术没有立足之地。


我想是时候指出:学习每一个课程之前,都要先搞清这一课程的学习目的。这一学科的应用领域。据我自身所了解到的同龄同学和低年级的同学的学习状况:他们之中很少有人知道学一个学科的学习目的,期末考试结束了也不知道学这科做什么用。这就失去了读计算机科学的意义。当然这与现存的教育思想不能说一点关系都没有。


总的来说,从教育角度来讲,国内高校的课程安排不是很合理,强调理论,又不愿意在理论上深入教育,无力接受新技术,想避开新技术又无法避得一干二净。我觉得关键问题就是国内的高校难于突破现状,条条框框限制着怎么求发展。我们虽然认识得到国外教育的优越性,但为什么迟迟不能采取行动?哪怕是去粗取精的取那么一点点。我们需要改变。从我们自身角度来讲,多数人4年下来既没有学习计算机科学的学术水平,也没有学习计算机技术的那种韧劲。在我刚上大一时,我的计算机科学入门导师,淮北煤炭师范学院王爱平教授曾经对我说过这样一番话:“当你选择了计算机这一门科学,就意味着你踏上了一条不归路,就意味着你一生都要为之奋斗……你的身后是悬崖,只有向前走,不能往后退。”


有些同学说按照这样学习学的东西太多,有的未见得有用,我想打个形象的比方:学校学出来的人都是一个球体,方方面面的知识都应具备。可是社会上需要球体的地方很少,反而需要的是砖和瓦,即精通某一行的人才。但是对于同等体积的物体,用球体来改造是最方便最省事的。学校的学生很多,为了能够使更多的学生来适应这个社会,学校也就不得以把所有的学生都打造成一个球体,然后让社会对这些学生进行再加工,成为真正能够有用的人才。即使你非常清楚自己的将来要干什么,并且非常下定决心要走自己的路,这一步你也必须走,世界是在不断变化的,你不能预料未来。想清楚,努力去干吧!


必须结束这篇“胡侃”了,再侃下去非我力所能及。其实计算机还有很多基础课都值得一侃。怎奈我造诣有限,不敢再让内行耻笑。最后声明:这些只针对本科阶段的学习。即使把这些全弄通了,前面的路还长,计算机科学需要我们为之奋斗......学习计算机科学需要韧性,更需要创新,需要激情。深刻学习理论知识,勇于接受新技术的挑战,这才是我们这一代人应具有的素质。最后送大家一句话“Wake up every day with a feeling of passion for the difference technology will make in people's life!”。

在我大一时无意中找到了南京大学网友sir的帖子“胡侃(理论)计算机学习”,这个帖子对我的大学学习起了至关重要的作用,写这份材料时也引用了其中的不少观点。再有就是每次和司徒彦南兄的交谈,都能从中学到很多东西,在这份材料中也有很多体现。以及每次在放飞技术网上每位同学诚恳的留言。这份材料是我原来在实验学院进行新生入学教育的讲稿之一,原有基础上改进了其中我认为不太合适的理论,修正了一些观点,在推荐教材方面结合我的学习情况有了较大改变。值得一提的是增加了一些计算机理论的内容,计算机技术的内容结合我国的教学情况和我们学习的实际情况进行了重写。这里所作的工作也只是将各位学长和同学们的学习体会以及我在学习计算机科学时的所思所想汇总在一起写了下来,很不成熟。目的就是希望能够给一些刚入学或者是学习计算机科学还没有入门的同学以一些建议。不期能够起到多大的作用,但求能为同学们的学习计算机科学与技术带来微薄的帮助。希望大家批评指正。

评注者:seafrog	2003年5月13日 12:06:00
  恩,
学习计算机理论,贵在有深刻的体会。
评注者:楚云	2003年5月13日 12:09:00
  楼主,我决定把你的贴子给贴到其他地方去,嘿嘿......
评注者:ZoLo	2003年5月13日 12:10:00
  还是学理论好 
不用发愁什么时候被淘汰 
你看有谁敢说高数要过时了
评注者:jpaddle	2003年5月13日 12:15:00
  学理论如果做不到一定的高度,是找不到工作的。
评注者:zengyi820	2003年5月13日 12:16:00
  计算机科学与技术这门专业,在本科阶段是不可能切分成计算机科学和计算机技术的,学技术不接触理论,我的学哥告诉我那只能做编码工人,我们学理论不能不学技术,这一点jpaddle说的一点也没错,但是为了我们能够做得更高,理论不碰是不行的。工程硕士和工学硕士的区别就在这里了。计算机科学所容纳的内涵太多,从事的工作也是五花八门,只是看你是和哪种工作,电脑排版也是计算机工作者,gates也是,只是要看看你要做什么。
评注者:zengyi820	2003年5月13日 12:17:00
  在微软的每一个包上都写着这样一句话:创造人类计算的未来!微软开发团队所能做的事情我们能不能做,一般企业的员工肯定不能,原因:很简单,大家到这里来的没有不用.NET的,用得如何如何好。一个简单的问题谁来写写strcpy这个函数的源代码,怎么实现是最高效的,一个是设计编译环境,一个是使用平台,后者只能跟着前者的屁股后面走,中国软件业为什么只能处在三级水平大家总应该有一些想法吧!一句话,我们应当在知识储备阶段打牢根基,但这并不意味着只学理论,文章一开始就说了,理论脱离了时间就失去了存在的意义,理论要去指导实践才有意义,之所以研究理论是要去更好地实现技术。让我们共同努力,创造中国计算机事业的未来。
评注者:nefu	2003年5月13日 12:18:00
  你的那篇《计算机科学与技术反思录 (完整版)》 我也看了。 
我是一个今年即将毕业的学生,也是学计算机的。你文中提出的学习计算机的方法,确实很好,但是并不适合每一个人,可以说不适合绝大多数人。 你列出的参考书目,每一本都是重量级的(我有幸拜读过其中的一些),这么多本,我想要让一个人8年时间都学完,同时都精通,太难太难。如果你确实是这么做的,我只能说您天赋极高。 看这些书不是看小说,蜻蜓点水不如不看。如果你愿意花8年时间完全学习基础知识,我敢说8年过后,你不一定能够成为中国计算机界的学术泰斗(我这么说你不要生气,毫无恶意)。
评注者:nefu	2003年5月13日 12:18:00
  国外的计算机为什么走在我们的前面,国外的学习方法和我们很不相同,杨振宁先生曾经说过,外国学生能够在很短的时间达到某一个研究领域的最前沿,但是他们的基础可能并不扎实,然而这个时候,他们会回过头来补足。learn what u really need. 
中国计算机发展,需要搞研究的人,但是,不能培养的都是搞研究的人。
评注者:nefu	2003年5月13日 12:19:00
  很大程度上,需要培养一大批能够把理论加以实现的人。天才只有那么一些,如果让每一个人都按照培养天才的方法去培养,只能毁掉科学。世界上最好的程序员如果让他去做数论研究,我想不一定能做出什么东西来。而让最好的学者去写一个程序,可能还不如一个中级程序员。但是,如果让最好的程序员加上最好的学者,中国的计算机科学才能有极大的进步。
以上是我个人的一些看法。很希望能够和你交流:)
评注者:zengyi820	2003年5月13日 12:20:00
  你好!首先我想我必须说:感谢您百忙之中抽出时间能够阅读我写的文章,以及做了很多的思考,在你的信中的若干观点我完全赞同。实际上我写这篇文章的初衷并不是让所有搞计算机的人都照着去做,第一我想和你的想法是一致的:没这个必要。任何桓龉叶疾恍枰饷炊喔阊醯娜恕8慰鋈缒闼担绻敢饣?年时间完全学习基础知识,8年过后不一定能够成为中国计算机界的学术泰斗!这点我是完全赞同的。
评注者:zengyi820	2003年5月13日 12:21:00
  实际上这篇文章是我在我们学校做新生入学教育时整理出来的文稿之一,每年都会用恍2豢煞袢希?0至80年代我们的大学教育是精英式的教育,而进入90年代,大学教育只能算是普及性的教育,原因很简单:扩招。现在研究生教育和本科教育一个显著的区别就是:精与不精的问题。learn what u really need。这句话我也经常对我的学弟学妹们说,但是紧接着一个很现实的问题摆在他们面前:What do you really need?他们并不知道。
评注者:zengyi820	2003年5月13日 12:22:00
   就像很多数学系的同学在入学时或者是在毕业那一刻从没想过去搞计算机,但是毕业后坐在各个世界知名企业当中的软件设计者大多都具有数学背景,而计算机系的同学很大程度上要去受他们的领导。微软集团的李开富博士我想你应当很熟悉吧,他在微软公司的成名之作就是以概率论及统计学原理解决了语音识别问题。计算机系的学生很多在上学时就有在公司里实习的机会,就能得到老板的好评,不瞒你说,我也有。但是就我所接触到的人来看,问题是他们现在能够做得很好,但是他们能做到多高,这是个问题。
评注者:zengyi820	2003年5月13日 12:22:00
   专科教育和本科教育的最大区别也在于此。我们学校一位非常有名的数学系计算机软件工程专家王仪华教授曾经说过:“想做一个高级程序员根本没必要去读本科,专科里多的是”何谓专科:答,专业化。何谓职高:答,职业化。他们都很早的作了定位:learn what u really need。但是这就是和本科最不同的地方。本科的课程设置有些地方看起来极不合理,一句话:没用。但是就像你初中,高中学的很多知识一样,这是个思维训练的过程。数学系的学生之所以做得好就是因为或者说至少是他们具有严谨的思维,有严格的推理思想。
评注者:zengyi820	2003年5月13日 12:23:00
   考虑问题会比我们多想一些。这些我们不是做不到的,而是我们一味的强调学习你需要的。而从来不去想你到底需要什么。What do you really need?我想只有你在做一项具体的工作或职业时你才有这个能力去说。就像我的一个网友,也是看到这篇文章以后认识的。他是联想集团IT研发部的。我问他你现在觉得在大学阶段你学得最有用的一门课是什么:答,不好说,为什么,答:职业不同需要的东西相差太远。即使职业相同,不同的工作中需要的东西有时千差万别。
评注者:zengyi820	2003年5月13日 12:24:00
   我问他你觉得自己学习自己感兴趣的东西怎么样。因为我周围的同学经常是逃课去自己学自己所谓有用的。他的回答是:OK,那你上大学来干什么了?以为自己很了不起,你觉得有用就有用了!国家高教司都不如你?天真!(呵呵,我经常被这样骂的。)在给你举个例子吧:在网络安全领域绿盟应当算是家喻户晓了,他们的主要工作人员一水职高学生。月薪比levono和microsoft也低不了多少,但是他们只能去用现成的技术,真正能搞网络安全解决方案的还是本科生:为什么:抽象能力有差别,思维角度有差别。
评注者:zengyi820	2003年5月13日 12:24:00
   我不是说大专生和职高生的能力有限。而是说所接受教育的方向不同。不同的教育的目的就是早就不同的人。不同的人从事不同的职业。现在问题已经很清楚了。我想你可能已经工作了吧,或者是像我的同学一样很多时间都是自己学自己认为有用的东西。如果一个本科生决定我今后就是去做个普通程序员,ok,那时另外一回事了。想做程序员还不容易呢!我们学校曾经接过一个公司的项目,这个项目说白了就是个MIS系统,但是竞标一年没人赶接,为什么?原因就在于里面用到了人工神经网络的很多原理。一般的公司不可能做。后来我们学校接了,也如期完成了。这是谁做的?不瞒你,数学系的建数学模型。我们系按照抽象出的模型要求编程实现的。
评注者:zengyi820	2003年5月13日 12:25:00
   我之所以说8年时间去学基础知识正是因为除了搞这方面的专家和高校教师不会有人真得这么做。只是告诉大家应当踏踏实实的做事,因为我们懂得太少。只是想安一安像我这样浮躁的年轻同学的心。我绝对不否认你的观点:需要培养一大批能够把理论加以实现的人。之所以有科学,或者说科学的意义就是将理论应用到实践当中去指导实践。这一点我是极力赞成的。在学术浮躁的今天,我怕的是今后国家能搞尖端的人太少,甚至没有。这篇文章说实话,不是写给所有人看的,只是想正正学术风气。鼓动大家踏踏实实做事。
评注者:zengyi820	2003年5月13日 12:26:00
   我这篇文章叫做《计算机科学与技术反思录》,里面有一言:“并将重点放在计算机科学理论上”主要是给计算机科学入门的学生尤其是大一的学生一个方向。仅将重点放在计算机科学上。如你说的很多计算机技术方面的问题在我的其他文章当中会有综合的论述。不在这篇文章当中阐述清的原因就在于:I am just a beginner.有些问题现在我自己还在体会,思考。我现在还是学生,还需要不断的学习。哪天我觉得自己有能力看计算机技术了我会第一时间和你探讨的。
评注者:zengyi820	2003年5月13日 12:26:00
   信中本着对问题的探讨讲,绝无其他意思,得罪之处望见谅。希望我们能成为好朋友。另外,我想把咱们的谈话贴到论坛里不知你是否同意。还有就是你是在哪里看到我的这个帖子的,因为我的这个帖子被贴得到处都是,我自己都不知道有几处。希望你今后能够多多指点我,还是那句话I am just a beginner。懂的东西还太少。最后如果你是学生,祝你学有所成!如果你是已经工作的哥哥,那就祝你工作顺利!
评注者:noah	2003年5月13日 12:29:00
   个人以为数学对计算机的影响算法固然是一个方面。更重要的是对计算机科学的发展有决定性的影响。虽然长时间来计算机的发展似乎是软硬件的互相协调,可是程序存储式计算机的思想是数学家提出来的。问题是是否会有更好的计算机设计思想,这个可能依赖于数学的进步与数学家的努力。计算机从技术到科学的转变很大程度上依赖于数学工作者和计算机工作者共同努力。所以如果从科学的高度来看计算机的发展,数学的作用就很明显了
评注者:zengyi820	2003年5月13日 12:32:00
   上面贴的是在CSDN上转载此文时各个学校的学生的一些讨论,精简了一些贴过来,作为对文章的补充。
评注者:zengyi820	2003年5月13日 12:54:00
   上面贴的是文章被转载到CSDN上面时各个大学的同学的一些评论,精简了一些贴过来,算是对文章的补充,希望大家一起探讨,发表自己的意见。
评注者:gucheng	2003年5月15日 22:17:00
   经典,经典.....
评注者:pengtao82711	2003年5月29日 1:16:00
   zengyi820,你好,一个偶然的机会,我看了你这篇《计算机科学与技术学习反思录 》,看后深有感触。谢谢,诚挚的谢意。我现在就读于中国人民解放军后勤工程学院,计算机系专科二年级,明年就要毕业了。曾经我以为学几种技术能找到工作就够了,但现在我已经有了更深更远的打算,正如王爱平教授曾经说过的一番话:“当你选择了计算机这一门科学,就意味着你踏上了一条不归路,就意味着你一生都要为之奋斗……你的身后是悬崖,只有向前走,不能往后退。”虽然还有一年的时间就毕业了,但一生的时间还长得很,我将踏上这条不归路,并为之奋斗。谢谢。
评注者:pengtao82711	2003年5月29日 1:31:00
   再次谢谢你,我准备把你的这篇文章发给我所有的计算机系的同学,尽管我们是专科生,但如果我们自己都放弃了,那谁来拯救我们?不过至少我自己已经拯救了自己,应该说叫醒悟了,原来我对计算机的理解从来都很肤浅,你在文章中提到的很多好笑的做法,我都有过,为此我感到抱歉。但我想以后不会再有了,永远也不会再有了。计算机科学和技术的领域太广了,每个计算机系的学生确实都应该先把计算机理论学好,打下坚实的基础。
希望各位在看了这篇文章之后向更多的人发表这篇文章,为了中国的计算机业,为了整个计算机领域的进步,希望大家都能努力做好,并希望所有计算机领域或即将踏入计算机领域的朋友都能在计算机领域取得巨大成功。
评注者:pengtao82711	2003年5月29日 1:37:00
   zengyi820,在我最初看你那篇文章的时候,我还以为你是位计算机领域的老前辈,没想到。。。。。。你也是82出生的吧?很高兴认识你,我叫彭陶,重庆人,现就读于后勤工程学院。我很佩服你,才大三就有这么高的造诣,希望能和你交个朋友。我的qq号:39672630,Email:pengtao82711@163.com。希望你能尽快回复。
评注者:zengyi820	2003年5月29日 3:12:00
   呦,谢谢大哥夸奖了,但是我要说个事实,计算机科学博大精深,现在有放飞网的同学和学长领我入门,我也充其量是个beginner.可折我的寿了,让我的学长看见了要挨骂的。
评注者:Youngwoo	2003年6月1日 22:34:00
   看了之后,启发不少
评注者:ui	2003年6月5日 16:13:00
   uiu
评注者:西门吹雪	2003年6月5日 16:17:00
   我隔断时间来这儿逛逛,发现北工大藏龙卧虎的人不少啊,这位小弟,这么年轻,但是对计算机科学的认识很深刻。不过文章的观点,我也认为只是适合一部分学生,适合于有志于基础理论研究的人才。计算机科学与技术博大精深,人的精力时间都有限,很难精通所有的基础理论。大部分人还是去学习自己技术所需要的部分理论吧。
评注者:	2003年6月19日 15:38:00
   理论联系实际,学以致用!
评注者:Microsoft-lee	2003年6月21日 15:47:00
   就是不一样!
所以还是厉害!
回去要想想看!
THANKS !
评注者:yangyang303	2003年6月23日 12:59:00
   小弟也是实验学院的,02级电信。拜读学长此文,心潮澎湃。“非典”时期,同学们的学习生活发生了极大的变化。最突出的是教学的组织形式的变化。对于离校学生而言,一反传统的面对面的教学,主要是通过网络等方式了解教学内容和教学要求进行自主学习。因为有了更多的自主性,多数同学应当是能够有计划的坚持学习,也有的学生学“疯”了,也有的学生玩“疯”了。而阁下通过这篇激扬文字,对自己的专业、定位进行积极的反思,无疑是这段时期学习的有心人和佼佼者。洋洋洒洒成一篇,这就是位有心人、佼佼者。值得我们这些新同学学习。鄙人这段时间,没有学长这么有创意的成果,但也坚持看看书什么的,相比起来,我差的还很远!


