\chapter{学习理论物理的途径}

又名《物理与数学的崩溃关系》


但凡爱看武侠的人都知道练武功有内功和招式,其实学物理也是大同小异.

物理所对应的内功就是数学.想必物理系二年级正在学"电动力学"的小弟弟
小妹妹们已经从王那领教了(对了也许上学期王不在,算你们走运).从纯粹
物理学的角度讲,一旦建立了MAXWELL方程组,里面的物理就少得可怜了.但
是就是为了那么一点点最精粹的物理,我们需要实用大量的数学工具,包括
物理系的四门数学基础课:高等数学,复变函数,数理方程和线性代数.这些
都是相当基础的课程,重要性自不必说.但是仅仅是这些课程学好了对于物
理来讲是不够的.我建议想学物理的人应当学一些更加高等的课程.

高等数学由于教学时间的限制对很多"古典分析"中的基础问题没有涉及.我
建议大家看看北大的张筑生写的<<数学分析新讲>>.当年我收集过各种版本
的"数学分析",比来比去还是张的这套好,内容充实适合自学.当然不要忘了
北大的<<数学分析习题集>>,虽然此书是给林源渠的<<数学分析>>配套的,但
是里面的题多而且好,可以补充张的书的习题不足的毛病.我建议大家花一年
到一年半的时间好好读读这套书.
复变函数.我建议大家着重于它的应用,也就是要会算.复变函数中有许多定
理在数学分析中有对应,并不困难.我建议大家去学复变函数中"古典分析"
之外的理论,比如共形映射,作为进一步学习的基础.我推荐北大庄钦泰的<<
复变函数>>,也许前面的内容和钟玉泉的类似,但是后面就不一样了.这本书
我也没看完.

线性代数.我建议大家看看王萼芳和丁石孙的<<高等代数>>.这是以前清华高
等代数课程的教材.这本书以古典的方法讲授了"古典代数"的全部内容,而且
习题丰富,仔细学下来很有好处.

数学物理方程.我建议大家看看希尔伯特和柯朗的<<数学物理方法>>.这套书
写得很精粹和全面.对于掌握了"古典分析"和"古典代数"的同学,一方面可以
以此来复习已经学到的几乎全部内容,另一方面这套书可以说是学物理的人的
看家本领,学到此为止可以说是"小成",更重要的是这本书中的许多内容已经
涉及现代数学的内容.相比之下  昆淼,郭敦仁和王竹溪的书虽然各有所长,但
是境界已经是纯粹应用了.当然如果精通这三位的书中的一本也算"小成".

我看能在短短的四年中有此"小成"已经很不容易,就算以前上五年有此小成的
人也不多.往往有许多人还没有"小成"就开始想"大成",结果是一事无成.

如果你不想做数学物理,"小成"已经是足够了.关键是学得要扎实,比如你可以
不知道许多定理,但是一定要知道所学的脉络,要知道"根",这样才能举一反三.


上面所说的只是内功修为,要学物理还有招式呀.

学物理应当从普通物理入手,这无可争辩.通过普通物理,可以慢慢感受什么是
物理,从而真正入门.力学就可以选物理系的教材,那套绿皮的<<力学与热学>>
的上.热学选<<力学与热学>>的下.这套书浅显易懂,内容全面,是初学的好书.
电磁学可以选赵凯华的<<电磁学>>.这套书很经典,而且内容也很丰富,是学习
电动力学的良好前导.光学可以选赵凯华的<<光学>>,这本书的部份内容已经
超出了普通物理的水平,应当属于中级物理的范畴,而且是光学专业的同学的看
家书.至于量子物理,我很难找出满意的书,因为量子现象几乎没有简单而正确
的解释,所以普通物理中很难含盖.

至于四大力学,虽然是物理的一个核心,但是我不建议初学物理的人要在四年之
内学完它们,因为这四大力学可以说是高深莫测,而且就算勉强学完了也不会精通.
对于物理的学士而言,我认为精通经典力学和电动力学之一已经是很不容易的事
了.经典力学可以选朗道的<<经典力学>>.这本书很薄,但是是朗道一套书中最好
的.从朗道对拉氏量的讨论,你可以发现,理论物理完全不是你以前所认为的理论
物理.电动力学可以选郭硕鸿的<<电动力学>>就可以了,看JACKSON的书需要很好
的数学基础,关键是对位势形偏微分方程有相当的了解.至于量子力学和统计力学
我认为不以物理为职业的人没有必要学.电动力学学好了学习电子工程类的电磁
场理论并不困难;经典力学学好了,学习机械类的振动理论也很轻松.而量子力学
和统计力学的物理以外的用处就不大了.所以对于以后并不一定干物理的本科生
而言,这种既学不会又"没用"的课,最好还是不学.

学过普通物理,经典力学和电动力学,作为一个本科生已经足够了.如果不打算继续
学物理了,那么可以学学其它的东西.你会惊讶的发现,由于你学了足够多的数学,
其它学科是那样的容易,而且它们细致和精巧的程度不会超过经典力学和电动力学.
如果打算继续学物理,那么就得学习物理学中最困难的量子力学和统计力学了.这两
门(实际是一门)学问可以说是高深莫测.就是对于一个内功小成的人而言,它们的数
学也是你所不掌握的.实际上,曾经有许多人试图把量子力学变成经典力学和电动力
学那样的"形式物理",但是这种努力总是以失败高终.这两门学问的深度远远超过我
们今天的数学所能达到的范畴.

量子力学实际上是一种量子理论.它所包含的内容极广,从大学三年级学生学的一维
无穷神势井,到超弦可以说都是量子理论.量子力学大致分两个层次,非相对论的量子
力学以及量子场论和量子规范场论.对于前者P.A.M DIRAC在1937年写过著名的<<量
子力学的原理>>.无论如何要从这本书学起.这本书会告诉你,量子力学不仅仅是薛
定锷方程,而是一组原理.从原理出发,而不是从具体问题出发,这正是真正的高手的
做法.但是DIRAC的书的练习太少,不妨参考曾谨言的<<量子力学I,II>>和<<量子力学
习题集>>.曾先生过于强调量子力学的丰富内容,而忽视了量子力学首先是一组基本
原理,这是曾先生书的不足.但是通过看DIRAC的书"顿悟"也好还是看曾先生的书"渐
悟"也好,最终是殊途同归.但是我以为还是要先看曾先生的书,多做习题为妙.不然
如果悟性不够那么光看DIRAC的书,你一点收获都得不到,而先看曾先生的书至少可以
照猫画虎打打基础,等到表面上的东西学得差不多了,再看DIRAC的书才会有"顿悟"之
感.但是你要明白,你所学的量子力学从数学角度讲是"形式的"和"未经证明的",并不
可以和经典力学和电动力学相提并论.实际上,很少有学物理的人关心这个问题,但是
有一本<<Quantum Physics>>对此详细地进行了讨论.此书虽然叫<<Quantum Physics>>
但是里面的内容是量子力学的数学基础.但是里面的许多概念是是现代数学的内容,
看起来很艰难.

量子场论的数学基础并不完善,但是作为一种"形式"理论近几年的物理学中用得越来
越多.搞物理,尤其是理论的人,应当学学.经典的教材是卢里的<<粒子与场>>.这本书
从DIRAC方程起手,容易为初学者接受,而且此书写得比较早,有许多现在流行的量子
场论的书中没有的内容.这可以使初学者体会到,我们是在某种原理下进行尝试和探索
,许多东西并不是天经地义的.

量子规范场论在学李群和李代数之前,是不能学的.

学到量子场论为止,那么也算是学理论物理有了"根".接下来的事情就要看你的兴趣了.

如果对凝聚态理论感兴趣,你可以学统计力学.这方面的书以朗道的书为上.朗道在这方
面可是得过诺贝尔奖.朗道在两册统计力学中,以俄国人惯有的繁琐(他的<<经典力学>>
是例外)将统计物理的原理和方法讲得清清楚楚.当然朗道讲的不全,你可以参考雷克老
太太的<<现代统计物理教程>>.这书几乎含盖了统计物理的所有内容,但是言之不详,好
在有参考文献.学凝聚态不能不学固体物理,我选的是黄昆的<<固体物理>>,这本书很好
理解.当年黄老爷子在文化大革命时还说"学(我的)固体物理不用学量子力学"呢!不过
那时候正在批判量子力学,黄老爷子可是为了固体物理不受牵连才说的这句话.不过黄老
爷子的<<固体物理>>确实写的容易懂,是初学者的良师.作为学凝聚态的人,群论是必修
了.不过我们学的是群表示论.学群论,孙洪洲(不是鲤鱼洲)的<<群论>>就足够了.群论的
内容大致是有限群和连续群两部份,前一部份和晶体的对称性直接相关,后一部份和角动
量理论有关,学凝聚态的人做含有d或f电子的紧束缚方法时自然会用到.如果想做点
FANCY
的凝聚态理论,那么就得看点FANCY的书了.比如马汉的<<多粒子问题>>(该有中译本了)
或者北大的<<固体物理中格林函数方法>>.不过读这些书之前最好读过量子场论,否则比
较艰难.而且作为过渡,最好先看过卡拉威的<<固体理论>>.不过能懂<<固体理论>>已经是

不简单了,清华没几个.

如果对光学感兴趣,那么除了赵凯华的<<光学>>作为基础外还要看看光学的名著.本人当
年

对光学深恶痛绝,没看过什么光学的书,总是考试之前背三天公式.如果想做量子光学那么

量子场论就有用了.量子光学的麻烦在于边界条件,一般量子场论的边界很简单,而量子光

学就不是了.一个有限体系的量子光学性质是很有意思的问题.比如微腔中的光吸收和发
射

以及由此引申出的光子晶体中的若干问题.这里要分清光子晶体和人工电介质.光子晶体
中存在量子效应,而人工电介质中没有.所以一个有三维人工周期机构工作在微波波段的
陶瓷算不上光子晶体,只是人工电介质.

如果对核物理感兴趣,那我建议你多看看角动量理论或者群论的书.这算是量子力学的一
部份.但是搞核理论的要求对这些东西极其熟悉,能够拿来就用.同样这些东西对搞量子
化学和能带论的人也很重要.不过做核理论是很辛苦的,不如凝聚态和光学那么轻松.

对物理学理论本身感兴趣的人恐怕内功"小成"就不够了.他们需要进一步学习数学.可以
从实变函数和泛函分析学起.学习实变函数,有利于你建立现代数学的一些基本观念(如函

数类)掌握一些基本方法以及积累一些素材.学过实变函数就可以进入现代数学的基础,泛

函分析了.只有学过泛函分析,你才能对(非相对论)量子力学有清楚的认识.这时量子力学

才不是形式的而是严格的.实变函数和泛函分析的书最好的当属<<REAL AND ABSTRACT
ANALYSIS>>

为了准备学微分几何,还要学一些拓朴和代数.这只是准备概念,不必费太多时间.代数
可以看蓝以中的<<高等代数教程>>,这书用近式代数的语言将古典的矩阵和线性空间的
理论加以重复,对于理解抽象的代数概念很有好处.拓朴可以看<<拓朴学基础>>.这书上
的习题狂多,不过只要第一章会了其它章节很简单.

学过泛函分析和拓朴就可以学真正在发展物理理论中有用的微分几何了.微分几何内容
十分庞杂,从最基础的导数的值等于切线斜率,一直到函数空间中的几何学.这些东西要
在短时间内学会很不容易,不过也有迹可寻.首选的入门书是陈维桓的<<微分几何基础>>
这书不需要高深的基础,但是却是微分几何的入门.学过之后就可以看陈省身的<<微分
几何>>了.这两本书读过以后再回头读<<数学物理中的微分形式>>,学习如何应用这些数
学.<<数学物理中的微分形式>>算不上严格的数学书,但是里面对如何使用数学却讲得很
好.如果觉得李群和李代数有用,还可以专门看看这方面的书.不过我建议找一本以特殊
函数为工具,介绍李群的书.看过以后你就知道Bessel函数等那些在数理方法中学过的东
西是何等重要.它们直接是对称性的反映,只不过那时你还小并没有认识这一点.学过这以

后你知道量子力学真正关心的是什么了.原来量子力学做来做去是一种关于对称的理论.
在这一理论中作为群的表示的基的波函数是次要的,而群本身和代表它的特征值才重要,
而这些被物理量正是特征值.

再往下就得听天由命了,也许你走运,发现了融合量子论和广义相对论的方法,也许不走运

什么也没发现.这可就是天数了,看再多的书也没用. 