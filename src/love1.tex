\chapter{究竟是喜欢一个人本身,还是喜欢一种预期}

出处:网络(豆瓣)\cite{love}


\begin{center}
原名:别向这个操蛋的世界投降\\ 作者:陈轩 
\end{center}

我妈常常喜欢念叨:人家又不喜欢你,你干嘛还要去喜欢人家。以前我一直想不出什么话反驳,只好简单粗暴地回应:一边去,你一老娘儿们你懂什么你。

我见过很多人,换男女朋友比换内裤还勤快的那种自不必说,还有像我们宿舍的闷骚青年,追女生,人家不睬他,他郁闷一阵子,提枪掉马就直奔下一目标而去了。我在旁边看的目瞪口呆。你要问他,他保准振振有词:人家又不屌我,我喜欢她有什么用。是的,有什么用。然后还会反过头来劝我:没用的,我跟你说$\cdots\cdots$这仿佛是如此的天经地义,如此的不证自明。 

昨天,我仔细地想了想,终于想通了这个问题。其中的关键就在于,你究竟是喜欢一个人本身,还是喜欢一种预期,一种前景,喜欢一种未来对方有可能和你上床睡觉结婚生子的可能性? 

这个年龄很多人都急吼吼地寻找另一半抱团取暖。要我说,其中有多少是真的喜欢对方本身,这很难说。我这么说可能一来打击面太广,二来没有调查取证,所以显得不那么令人信服。其实这很好判断,那就是扪心自问:换一个人行不行? 

这样多少有点神经质。对大多数人来说,并不存在一个绝对不可替代的the one。否则的话,这个世界会麻烦许多。小的时候,小到我才第一次思考爱情这回事的时候,我就对一个问题百思不得其解:你喜欢一个人,而这个人在茫茫人海中又恰巧喜欢你,这是多么大的一个巧合啊!而幼小的我放眼望去,这个世界上充斥着不可胜数的一对对巧合。 

要解释这样一件事,只能说明,在大多数人眼里,另一半绝不是不可替代的。而每一个个体的特质,很大程度上是相异的。换句话说,要追溯这种可替代性的载体,那可能就是每个个体作为伴侣所能为对方提供的“服务”了。 

比如说,深夜陪你聊天,闲暇陪你娱乐,工作学习相互鼓励,人情冷暖相互慰藉,生理需要相互解决。然后买房结婚,构筑家庭,生儿育女,传宗接代。老了之后相互扶持,终了一生。这些都只是些伴侣给你带来的效用而已。这个过程中,肯定会产生感情,不过这个感情的基础来自于这些过程当中一点一滴的积累,而不是来自于对方本身。换句话说,换一个人,你照样可以和他(她)积累起深厚的感情。而关键就看谁最开始和你开启这段旅程。 

所以,少不更事的时候,我们总以为只有某个特定的对象才能给我们带来这一切,只有他们才能给我们幸福感。而后长大了我们知道并不是这么回事。“好女人多的是的,何必呢。”我无数次地听见这句话。这就是所谓的成熟吧。 

这一切,也很美好。但这不是我想象中的爱情。 

就像我那个倔强的困惑,如果不存在将就凑合的心理考量,如果每个人都是固执的完美主义者,那么怎么可能你喜欢的人也正好喜欢你呢?但是,一旦喜欢,那便是雷打不动的定格。爱情所投射的对象本身基本不会产生多少重大的变化,除非她人品突变,性格突变,样貌突变,而这一切绝对是小概率事件。爱情对象在那,那么爱情本身便随之恒定。她不喜欢我,那么我也就不喜欢她了,这作何道理?我喜欢的是她这个人,而不是“她可能喜欢我”“我们可以像情侣一样生活”这种期盼。 

所以真正着眼于对象本身的爱情——我不敢说这是真正的爱情,但这是我理解的爱情——是这样的:她不认识我,我会喜欢她;我们点头相交,我会喜欢她;她拒绝我,我会喜欢她;她反复拒绝我,我还是喜欢她;她不回我信,不听我电话,不回我短信,我还是喜欢她;她和别的男人谈恋爱,我还是喜欢她;她和别的男人上床,我还是喜欢她;她和别的男人结婚,我还是喜欢她;她死了,我还是喜欢她。 

因为我喜欢的是她本人,她本人不变,感情就不会也没有理由变。这一切都不会随着她对我的态度,她自身的选择而变化。 

但是,这有什么用呢? 

“我想学哲学,我想学艺术。”“学这些有什么用呢,能当饭吃吗?” 

“我就是喜欢她”“她又不喜欢你,有什么用呢?” 

“这个社会为什么这么不公平?”“这个社会就这样,你说这些又有什么用呢?” 

是的,有什么用呢。我们每当面临内心的召唤的时候,这个问句都会鬼魅般如影随形。有时甚至不用父母亲友耳提面命,我们自己就习惯性地自问自责:有什么用呢?有什么用呢? 

那要是追问到底,我们生于世间,百年来往,又有什么用呢? 

如果生命是有意义的,那么我们内心的召唤就是有意义的。午夜梦回想到她时那满心酸楚难言的悸动,铺开信纸秉笔夜书时那字斟句酌的计较,经年再见面对佳人时那喷薄欲出的情意,这一切都是爱情原本的意义所在,这一切都是生命本身赋予的。 

这有什么用?这本身就是最大的意义。 

我们年轻时那些美丽的梦,它们往往敌不过这坚硬的世界,我们要将就,我们要放弃,我们要隐忍。比如爱情,谁年少时没有些洁白的向往。但我们敌不过现实的无奈,父母的唠叨,亲朋的压力,甚至敌不过我们自己本身内心的虚弱和不耐烦。然后我们就将其掩埋,扭头它寻,只有等到回首前尘时才泪满衣襟。 

我知道,很多人笑我幼稚。就连身边很好的朋友也常常对我说:“我保证,XX年之后你就不这样想了。”当然了,他们一再看着我过了XX年,还是一如既往地这么幼稚。这算幼稚吗?我只是觉得大家的理解不同罢了。 

当然,我并不是说我不会放弃。“也许有一天我会放弃,但是我绝不会像那些自以为看透了的人那样,等到将来自己的儿孙后辈面临这种类似的境遇的时候,傻逼哄哄地嘲讽他们,说一些“别犯傻了,爱情这东西,就是$\cdots\cdots$”之类的屁话。我会对他说,儿子,老爸当年也这样过,但是老爸比较没种,没有坚持到底,就向这个急躁的世界缴械投降了。希望你比老爸有出息。去吧,坚持你自己的内心,老爸支持你。” 

是的,虽然自知终须一败,但请再多坚持一会,别向这个操蛋的世界投降。


出处:网络

\bibliographystyle{plainnat}
\bibliography{gk}
\clearpage

