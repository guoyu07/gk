\chapter{解密中国互联网}



\begin{center}
来源:\href{http://hi.baidu.com/ncaoz}{caoz的和谐blog}\\
作者:4399曹政\cite{caoz}
\end{center}

\zihao{5}

出处:\url{http://hi.baidu.com/ncaoz/item/6895b089a6cc71ded1f8cd4e}

前言

    修改一下前言,那个,本来呢,写这样一篇文章,刻意避开微博,很想测试一下朋友圈的传播效应;but,冯大辉坏了我的好事,如果是别人转到微博,我还可以认为是噪音数据,@fenng的威力太大,我现在已经无法有效甄别朋友圈和微博的来访比例了。不知道百度的童鞋是否可以帮我分析一下。

    另外,修改错别字及部分勘误,增补一项内容。


\section{中国互联网的构成}


如之前冯大辉总结,中国互联网分三个层面;第一层面是媒体上的互联网,也就是大众容易识别和认识的互联网;第二层面是草根互联网,这是中国互联网巨大的组成部分,却极少在公众面前出现;第三层面是黑暗互联网,其实它一直以来,非常巨大,非常恐怖,以至于,往往因为某些疏漏造成了全国性的事件,人们才能窥到冰山一角。


第一种,媒体上的互联网,主要的思路是,覆盖尽可能多的用户,生怕别人不知道自己;搞个发布会,要给记者塞车马费,各种软文公关铺天盖地。


第二种,很多年以前,我一直以为是他们不掌握媒体资源,所以被忽视;后来和这些人接触多了,才理解,其实草根互联网,很多是怕媒体的,怕被精英和同行了解,原因很简单,他们都很担心,如果巨头理解了他们的业务构成,理解了他们的用户获取方式,恐怕很快,他们就会失去一切归零; 还记得风风火火的开心网么?各种人给开心网的衰败找了无数理由,我只陈述一个简单的事实,QQ农场上线的时间,就是开心网由盛转衰的转折点。

草根互联网,生存壮大于巨头看不起的环境,并依赖于特定的受众群发展,他们的思路是,我照顾好我的用户就得了,精英们最好别知道。

当然,壮大后的草根互联网,往往也会转入媒体上的互联网,比如最近,forgame上市,多少媒体如梦方醒,多少媒体人开始疯狂补课,这公司哪里冒出来的?

草根互联网的典范有,2004年之前的hao123;2012年之前的4399,各种地方社区如化龙巷,小鱼社区,西子湖畔;8684公交查询,9158等等。

其实,在2002年之前,QQ也是草根互联网的典范。有谁记得,当年南非电讯投资QQ的时候,多少业内专家笑话南非人SB,事实证明,谁是SB?


第三种,黑暗互联网,他们隐藏的更深,只有在特定的时间,特定的事件,才会一不小心暴露在媒体面前;还记得六省断网么?还记得前几天突然半夜里 .cn域名解析全部挂掉了么?这就是黑暗互联网擦枪走火的事情,这个领域包括但不限于私服(百亿+市场规模),外挂,组织性盗号,地下账号交易及漏洞黑市,网络诈骗,DDOS攻击产业(与私服产业密切相关),黑卡;单纯的孩子可能会认为,这事交给警察叔叔不就好了,中国那么多网警;这个,据我粗陋的了解,这个,我是不敢在公开文字里披露的。

只说一个小例子,当年Xfocus论坛有个热帖,两个黑产的代表人物因分赃不均在论坛骂战,互揭老底,辗转翻了几百页,成为神贴,后被有关部门勒令锁帖,至于内容,很黄很暴力就是了。

盛大最后与私服行业全面和解,成为中国特色的合法私服产业,这个背景,不说了。


\section{中国互联网的发展逻辑}

第一,用户比客户重要

    最早一些商业精英有一个思路,说是生意离钱越近,赚钱就越近。

    但是在互联网,这个逻辑是错的;不论中国还是美国,这个逻辑都是错的;前段时间周鸿祎借用了毛泽东的说法“人在地失,人地皆存;人亡地在,人地皆失”,人就是用户,地就是收益;说的是对的。

    范例1:最早推出竞价排名的公司,叫做overture,这个生意模式很好,也发展了足够的客户,依赖于与雅虎和谷歌的合作,一度成为市场上最受资本追捧的公司,但是问题是,他只有商业模式和客户,却没有属于自己的用户;突然有一天,google宣布,不再和overture合作,自己建立广告系统,一夜之间,这家公司的业绩下降2/3;祸不单行的是,雅虎也找了过来,要不卖给我,要不我们也学google自建广告系统;overture连还价的机会都没有;只好委身变卖。 有最优质的客户,有最牛b的商业模式,没有用户基础。  此外,DoubleClick 同理。有兴趣的童鞋可以查一下,doubleclick,全球最大的广告中介平台,拥有最强大的广告发布算法,覆盖全球的优质客户基础,因为没有自己的用户群,是怎样股价狂跌,最后被迫卖给google的。

    范例2:263免费电子邮局,曾经市场第一,为了追求收入;强制升级到全面付费版本;他们的逻辑是,邮件地址类似于手机号码,高端人群不会随意变更邮件地址;结果,可笑的是,不但他们丢失了免费用户,付费用户也流失殆尽,中国互联网的奇葩案例。

    范例3:QQ 马化腾最初做QQ并没有自己做运营平台的想法,只是想把系统卖给运营商;结果运营商从软件工程的思路来考核,这个东西多少人月做出来的? 这么一算,QQ连100万人民币都卖不掉! 100万人民币,你没看错!!当时马化腾几乎80万人民币就卖掉了QQ,可这时恰好看到了AOL收购ICQ的新闻,1亿多美金好像,是按照一个用户多少钱算的,pony眼睛一亮,原来互联网上,用户=钱!然后他按照这个估值重新估价,结果中国的各种互联网精英嘲笑不已,新浪各种白领用户还可以算点钱,QQ那些小p孩也值钱?别开玩笑了!IDG当真了,南非人当真了,那些精英们就说,看,SB非洲人,被马化腾忽悠了吧。 今天还会有人质疑QQ的用户不值钱么? 但是就在最近两年,还有不少人质疑4399的用户不值钱,这个,我就只能呵呵了。

    范例4:百度, 谁还记得当年,百度不过是一个技术服务商,那时候流行一个词叫ASP (application Service Provider),投资圈的故事是,美国掘金,卖裤子的发财了,百度当时走的就是这个路线,给门户提供技术引擎,但后来为了发展自己的用户平台,得罪了最大的客户新浪。当时媒体一股脑认为,霸主新浪分分钟捏死创业公司百度。so,今天你看到robin li 意义风发的讲商业模式多么重要,我只提醒大家一句,当年他颠覆的就是以客户为中心的模式,才有了百度后来的辉煌。

    范例5:360,周鸿祎前段时间分享的文章提到的案例,免费杀毒,取悦用户,一年1.8亿杀毒软件分成不要了,得罪了自己最大的客户。后来的回报,是当初的10倍。 是的,我知道这条很多人会有争议,我知道有不少朋友一提360必然要讲出一堆七七八八的问题;我只陈述这个事实,其他的,大家自由发挥。


第二,草根比精英重要

       最初,投资圈好说一句话,80\%的财富集中在20\%的用户身上,所以,服务好这些人,就可以赚到大钱,事实证明,在中国互联网,服务好草根用户,才是王道

    范例1:网址站的奇迹;我知道很多人还是没搞清楚360怎么赚的钱;我告诉你们,他们最大的收入来源,其实就是360的网址导航;各位知道么? 百度收购hao123后,一直是低调处理,闷声赚钱;但是今天,你去百度再看看,hao123已经迅速扩充为独立事业部门,并且拥有了自己的联盟渠道业务,以及非常宽松的预算,为什么?百度和360的对抗重心,在流量入口,而这个流量入口,绝大部分,集中在网址导航。网址导航,草根用户的上网入口,多少精英不屑一顾。

    范例2:还是QQ,中国互联网的大哥大,一度被认为是低端用户的产品毫无价值,前些年,有一种风气,商业人士用MSN,小p孩才用QQ,我跟身边的朋友说,想不脱离中国互联网,就别放弃QQ,事实证明,我是对的;当然,今天你有了放弃的理由,因为微信出来了。

    范例3:唯品会,中国已上市的电商公司里,貌似表现最好的就是唯品会;谁还记得,当年唯品会创业,信誓旦旦的认为,中国奢侈品消费进入爆发期,赚有钱人的钱,才是王道,结果烧光了多少美金?一路亏钱,后来痛定思痛,决心转型,主打二三线品牌促销,降低用户消费层次,一下子爆发了,钱也赚到了。这个例子最典型不过!

   范例4:域名生意, 1997年,我在北京读书,开始给互联网公司打工,那时候的互联网公司,和现在不能比,就是注册域名做企业网站的,当时,我们认为,好的域名,就是英文域名,数字的、汉语拼音的,弱爆了,谁会去用。当时的互联网,是精英互联网。而英文域名,基本上老外都注册光了,所以,我们认为,1997年,没什么好域名可以买了。 2001年还是2002年,蔡文胜先生才进入域名行业,汉语拼音,是中国人熟悉的;而数字域名,是输入难度系数最低的。草根需求远大于精英需求。说自己没有眼光,就是当时一直没有意识到,草根需求才是互联网王道。


第三,跨界优势及资源副作用。

   caoz做过几年传统的IT行业,一直以为资源是决定成败的关键因素;但是在互联网接触了几年,越来越发现,资源优势方,往往因为资源优势,忽视了用户体验和用户诉求,在竞争中,动作迟缓,拼劲不足,往往落败。

    越有资源越不行,几乎成为互联网铁律;而目前包括百度,腾讯,也出现了这样的反思,他们内部叫做“富二代思维”,百度,腾讯的内部产品,往往有富二代的思路,仰仗资源,反而缺乏竞争力。

   先说几个资源副作用范例

    范例1:微信是腾讯爆发的重要产品,但是,微信却并非腾讯嫡系团队的战果,腾讯移动部门几百人,在移动互联网领域屡屡错失良机,广州的电子邮局团队,反而爆发了巨大的冲击力。

    范例2:新浪刚出来火的时候,有一家新闻网站高调出世,就是千龙新闻网;当时千龙新闻网是传统媒体集团的产物,有各大传统媒体的合法授权,简单说,可以认为是官二代;当时一群评论家认为,千龙新闻网的资源优势远胜新浪网,新浪将会很快被终结;而事实是,这种衔着金钥匙出生的网站,注定没有竞争力,居然一度沦落为链接农场,成为搜索引擎要格外注意的垃圾链接来源网站。

   再说跨界竞争案例,跨界竞争者,不受行业思维局限,敢于求变,一动手就颠覆你的商业模式,往往出其不意。

   范例1:史玉柱搞游戏

     当时认为搞保健品的弄游戏纯粹是乱来,多少资深游戏人都给史玉柱的游戏下了一定不行的结论;结果呢?虽然今天我们说巨人似乎后续的产品也不见得多好;但是游戏行业公认的一点是,征途颠覆了游戏的传统商业模式,这个模式已经被人称为中国模式。而后续中国的页游,手游,都延续了这一模式,从按时间付费转为免费游戏,道具付费。

   范例2:360搞杀毒

      360瑞星大战一开始,我就认为瑞星输定了,我的判断依据是,互联网模式必将击碎传统软件模式,事实正如我预料。所有传统的IT公司,都应该从此吸取教训。

    范例3:小米搞手机

      虽然我一直还算是比较看的开跨界竞争的,当初我还是认为雷军越界太大了,用互联网的思路逆袭传统生产领域似乎不太可能,但事实击碎了我的判断。

   范例4:新上市的 forgame

      这个公司的高管,创始人,没有一个传统游戏行业的人!在页游初起,火爆的时候,传统的游戏公司在干什么?看不见,看不起,看不透,做不来,追不上,就这五个步骤。没有游戏行业背景,反而没有包袱和思维定势,更敢放手一搏。

    当然,要说页游这个领域,这些年的新贵都是跨界高手,比如心动游戏,比如恺英网络,比如游族,等等等等。


第四,视野比勤奋更重要

   勤奋当然重要,但正确的视野,会让你的勤奋,以n倍增值。

  范例1: 我有个认识超过10年的朋友叫苏光升,他以前做了一个智能手机社区,是关于塞班的,但是做的规模并不大,大概市场第三的样子,苦巴巴的坚持着;后来因缘际会认识蔡文胜先生,蔡老板和当时的市场第一谈了谈,建议对方转型安卓,对方表示塞班市场如日中天,没有转型的必要;蔡先生后来和苏光升聊了聊,那时候安卓的市场占有率好像还不足5\%,苏光升对安卓的前途也是半信半疑,后来又请教了创新工场的汪华先生,两位牛人的一致判断让他有了主心骨,坚决转型,结果在很短的时间内,接连做出了极为有影响力的安卓产品,并且得到了巨头公司的认可和资本合作,公司的估值在两年时间增值了几十倍;原来被认为遥不可及的竞争对手,现在,嗯,在后面遥不可及的位置。


 范例2:2004年,我第一次见俞军,听他讲搜索引擎,他说,搜索引擎是改变人类知识获取能力的一种革命,与造纸术,活字印刷并列。这些年我反思,为什么当初那么多公司有做搜索引擎,却只有百度脱颖而出,因为很多人,包括我们熟知的很多巨头,也包括当时的周鸿祎,张朝阳,只是把搜索引擎当做一种工具,一种获利手段,一种模式;只有足够视野的人,才会意识到,搜索引擎所带来的冲击和变革,是多么的巨大和深远!2004年,谁会相信,一个搜索引擎公司,可以颠覆如日中天的门户呢,实际上,2001年,俞军就已经预见到了。事实证明,他的远见,成就了百度,也成就了他自己。


第五,免费的是最贵的

这个真的是中国特色的,好像是史玉柱最早说的?不太确定,但是史玉柱绝对是一个典型的代表。

巨人集团的游戏,不但免费玩,还给玩家付工资?传统游戏人会觉得不可思议,但是最后算下来,收益率却高的惊人。这一模式已经成为中国游戏领域的黄金法则。

植物大战僵尸2,在全球都是付费下载,只有中国是免费下载,但是、只有中国市场,付费道具最贵!这也算是本土化的一个范例了。 当然,举这个例子并不代表我认可这种行为。 勘误,感谢网友提醒,说目前中国和全球均采用免费+内置道具模式,而且价格在诟病后已经调整统一,我没有仔细研究,我依稀记得最初全球是付费的,中国是免费的,中间是否发生了变化?

360,免费杀毒后,收益已经超过了之前杀毒行业总和的n倍。

淘宝 Vs Ebay ,用免费开店模式+草根网站推广  秒杀了 付费开店模式+门户网站排他广告。

但是今天我们看最新披露的数据,淘宝的利润水平远超ebay,不是远超ebay中国,是远超ebay全球。神奇不?


纯正的中国特色,免费是最成功的商业模式。

周鸿祎说,也许有一天,硬件会免费。

这一天有多远不知道,但是极路由已经用低于成本价销售了,我理解为,这是一个信号。


第六,唯一不变的,是变化;

这是马云说的吧,我有时候和一些创业者聊天,他们会有一些想法,然后纠结于对还是不对,然后各种思考,各种咨询,就是不动手;我会说,对不对,其实我也不知道,但是如果不去行动,那就永远都不对;做错总好过不做。

还记得某些大佬信誓旦旦,“我们很专注,永远不会去碰什么什么”,若干年后,你再去看,他们大碰而特碰;当然,你可以认为此一时,彼一时,当时不做也不能说是错,但是我想说的是,变化是一直存在的。

QQ最初是为了给电讯行业做增值服务,当时不要说各种精英评论家不知道它的前途,马化腾自己能预测么?没人能够!2001年的时候,互联网泡沫破灭,各种低潮,中国评论家们天天说的是,什么时候中国才会出一个10亿美金市值的互联网公司呢?今天我们看到QQ市值已经是千亿美金;当时不要说千亿美金,百亿美金听上去都是神话。如此巨大的市场契机,准确洞察者自然能捕获先机,但是即便做不到准确洞察,先人一步,能快速应变者依然可以脱颖而出。

百度最初是给门户做技术服务商,后来看到google崛起,才断然转型,李彦宏回国时,也并没有一步就找到方向。

在转型和变化中,最具代表性的是网易,我一直认为,丁磊是中国最早、最成功的个人站长,从虚拟社区,到免费邮局,跟风门户,然后转战SP移动增值,又逆袭游戏领域,网易从辉煌上市,到出现财务丑闻,跌成垃圾股几乎退市,又成为三大门户里第一个翻身的公司。这家公司的变化,最有代表性;而且网易今天仍保持着草根互联网的特质,在传统的门户三巨头里,现在的网易是账面上最有钱的,又是最低调的,媒体上最少提及的。

周鸿祎创建qihu,360只是尝试中的副产品,但是副产品成就了这家公司。谁还记得qihu最早是主推社区搜索的呢?


中国互联网的巨头,几乎没有谁一上来就找对了方向。除了如上的例子,陈天桥,小米,等等,都是在巨大转型后获得了巨大的成功。所以,远见卓识固然重要,但快速应变依然是互联网生存的必然法门,那些大佬们说,他们的巨头公司离倒闭只有几个月云云,真不是危言耸听,在移动互联网大潮来临时,腾讯靠微信躲过一劫,在股市上这一变化也有清晰的体现。360布局移动,失之苹果,收之安卓;总算有一席之地。百度危急之下抛出19亿美金大单,保持市场地位;这一切一切,都是巨头在市场变化面前,如临大敌,不敢放松的体现,而最新的,是阿里不遗余力的热推来往,一样的,没有谁有铁打的江山,即便你有市场顶级的地位,一轮变局就会覆灭。诺基亚、黑莓的故事,就在近前。


很久没有写这么长的文章了,如果您觉得这个文章还不错,请分享到微信朋友圈。

微博我就不发了,您也别发了,实话说,氛围越来越差。

\bibliographystyle{plainnat}
\bibliography{gk}
\clearpage
