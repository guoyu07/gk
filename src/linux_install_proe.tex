\chapter{Ubuntu 7.10 安装 Pro.Engineer.Wildfire.v3.0 M080 for Linux}



发信人: oicu (午夜阳光), 信区: Linux

标  题: Ubuntu 7.10 安装 ProE.Wildfire.v3.0 M080

发信站: BBS 蓝色星空站 (Wed Mar 26 01:55:47 2008), 站内

Ubuntu 7.10 安装 Pro.Engineer.Wildfire.v3.0 M080 for Linux\cite{linux_install_proe}

声明:

只供个人学习研究,不得用于商业行为。

PTC公司已不再开发新的Pro/E for Linux版,Pro/E Wildfire 3.0是目前已知最后一个Linux版本,Pro/E 4.0之后只有Windows版。

安装方法参考 http://forum.ubuntu.org.cn/ 论坛里的相关帖子,本文主要讲一下配置。

个人用户没有必要使用浮动授权,因此只介绍单机版(锁定的授权)的安装方法。

硬件要求:

网卡一张,内存推荐1G以上,显示器推荐使用19寸的,需要硬盘空间3G以上。

本文测试使用的硬件平台:

CPU:   P3E 850MHz

内存:  128*3 SDR

显卡:  GF2 MX400

显示器:17 CRT

操作系统:

Ubuntu 7.10 中文版(32bit)

下载软件(两个文件共1.2G左右):

http://www.verycd.com/groups/software/123786.topic

TLF-SOFT-PTC.PRO.ENGINEER.WILDFIRE.v3.0.M080.LINUX-SHooTERS.nfo

TLF-SOFT-PTC.PRO.ENGINEER.WILDFIRE.v3.0.M080.LINUX-SHooTERS.img


这个光盘镜像只有英、法、日、德、意语言,没有中文,而且不带Pro/Mechanica,唯一一个与Mechanica有关的设置就是“定位其他安装目录(可选)”Locate Other Installation Locations (Optional)


一些约定:

根据实际情况修改,用过ProE的都知道默认的文件存放很凌乱的,什么都往用户目录里面放,因此建立好一系列目录便于管理及使用,目录尽量简短、不带空格。

安装目录:

/opt/PTC

ProE程序文件安装目录:

/opt/PTC/wf3

授权文件存放目录:

/opt/PTC/license

默认工作目录:

/opt/PTC/start

trail.txt跟踪文件存放目录:

/opt/PTC/start/trail

用户配置存放目录:

/opt/PTC/text

虚拟光驱mount目录:

/opt/CDROM

用户名及组:

oicu:oicu

开始安装:

安装前关掉所有桌面特效,ProE和compiz貌似有冲突,必须关了compiz才能用。如果是DELL的本子的话,建议先装个风扇控制的,要不过热会死机的。

建立目录及安装需要的软件包:

sudo mkdir -p /opt/\{PTC/\{license,start/trail,text,wf3\},CDROM\}

sudo chown -R oicu:oicu /opt/PTC

sudo chown -R oicu:oicu /opt/CDROM

sudo apt-get -y install csh portmap

sudo apt-get -y install libmotif3 libstdc++2.10-glibc2.2

sudo apt-get -y install libgtk1.2 libgtk1.2-common

安装Gmount-iso方便加载光盘镜像img文件:

sudo apt-get -y install gmountiso

安装后在Application-system-Gmountiso打开程序,默认只能挂载iso文件,打开文件类型选择为“所有文件”即可挂载img文件(附图)。

运行安装程序:

LANG=C

sudo /opt/CDROM/setup

暂时不要选择任何安装,把左下角显示的网卡MAC地址记下来。

如果无法启动安装程序,试试export LANG=EN

修改授权文件:

cp /opt/CDROM/SHooTERS/ptc\_licfile.dat /opt/PTC/license/wf3\_licfile.dat

chmod +w /opt/PTC/license/wf3\_licfile.dat

gedit /opt/PTC/license/wf3\_licfile.dat

将wf3\_licfile.dat文件中所有的 <HOSTID> 字符(包括<>符号在内)替换为网卡MAC地址(MAC地址的格式为xx-xx-xx-xx-xx),然后保存。

返回到setup界面选择 "Pro/Engineer" 进行安装英文版,安装目录选择为:

/opt/PTC/wf3

如果选择的是 "Custom Installer", 是自定义安装程序,仅仅是将所选择的文件复制到硬盘。

在 license-config 界面,选择 "nodelocked license" 后指向刚才编辑的授权文件:

/opt/PTC/license/wf3\_licfile.dat

之后什么都不用再选择,一路Next即可。

如果顺利的话,很快就安装完的。

不要急着启动,还需要做一些设置才能启动ProE。

sudo cp -rf /opt/CDROM/SHooTERS/wf3/* /opt/PTC/wf3/i486\_linux/obj

制作启动脚本 "Wildfire3.0" 放到桌面或者 \~{}/.gnome2/nautilus-scripts/ 目录,下面为脚本内容:

\#!/bin/sh

cd /opt/PTC/start

LANG=C

cp /opt/PTC/text/config.pro /opt/PTC/wf3/text

/opt/PTC/wf3/bin/proe1

加上执行权限(附图):

chmod 755 \~{}/Desktop/Wildfire3.0

脚本内容的解释:

cd /opt/PTC/start 的作用是进入默认工作目录,根据情况修改。由于工作内容的不同,使用的配置可能不一样,cp那行的作用是切换不同的
config设置,如果工作目录固定不变,可以不使用该行。

使用中发现,上面的脚本中的环境变量LANG不能设置为en\_US.UTF-8以及"sudo dpkg-reconfigure locales"命令能查看到的任何语言,但是随便设置一个不存在的语言都可以启动ProE,不可以不设置。

如果脚本放到 ~{}/.gnome2/nautilus-scripts/ 目录,可以在桌面点击右键,选择 "scripts->Wildfire3.0" 来启动ProE。

安装到此结束。

ProE的配置:

建立一个 config.pro 文件,放在 /opt/PTC/text 目录中(如果上面的启动脚本没有使用cp那行,那就放在 /opt/PTC/start 目录中),内容如下(!号开头的行是注释掉的,这是我个人用的部分配置,请根据情况修改,注意Linux系统和Windows不同,文件及目录是区分大小写的):

! Made by oicu\#lsxk.org 2008.03.25, Rev ProE.Wildfire3.0 for Linux

! Modified: 2008-04-05, oicu\#lsxk.org

!== Menus display language: English ==

menu\_translation no

msg\_translation no

dialog\_translation no

help\_translation no

!== Menus display location language ==

!menu\_translation both

!msg\_translation yes

!dialog\_translation yes

!help\_translation yes

!== Colors ==

!pro\_colormap\_path /opt/PTC/text

system\_colors\_file /opt/PTC/text/Pre-Wildfire-scheme-syscol.scl

!== Drawing ==

!drawing\_setup\_file /opt/PTC/text/jis.dtl

!drawing\_view\_origin\_csys PRT\_CSYS\_DEF

!format\_setup\_file /opt/PTC/text/jis.dtl

!pro\_symbol\_dir /opt/PTC/text/symbol

!save\_drawing\_picture\_file embed

!== Environment ==

!prehighlight no

pro\_unit\_length unit\_mm

pro\_unit\_mass unit\_kilogram

pro\_unit\_sys mmks

spin\_center\_display no

!mass\_property\_calculate automatic

!== Features ==

allow\_anatomic\_features yes

!== File Storage \& Retrieval

file\_open\_default\_folder working\_directory

!lang\_propagate yes

!== search\_path ==

!search\_path /opt/PTC/start

template\_designasm /opt/PTC/text/startassy.asm

template\_solidpart /opt/PTC/text/startpart.prt

!== Model Display ==

show\_shaded\_edges yes

spin\_with\_orientation\_center no

spin\_with\_part\_entities yes

tangent\_edge\_display dimmed

!== System ==

set\_trail\_single\_step yes

trail\_delay 0.9

trail\_dir /opt/PTC/start/trail

!== User Interface ==

mdl\_tree\_cfg\_file /opt/PTC/text/tree-none.cfg

visible\_message\_lines 3

open\_window\_maximized yes

!== Mapkeys ==

mapkey purge @MAPKEY\_NAMEpurge;@MAPKEY\_LABELpurge;\

mapkey(continued) @SYSTEM/opt/PTC/wf3/bin/purge;


配置的说明:

前面是设置显示语言为英文版,本文安装的就是英文版,可省略。

allow\_anatomic\_features这行一定要加,可以增加一些高级命令。

open\_window\_maximized是启动窗口最大化。

file\_open\_default\_folder的作用是,点open打开文件时,默认打开的目录设置为工作目录(启动脚本中cd的目录)。

trail\_dir是设置trail.txt文件存放目录,这样就不显得文件凌乱了,关于

trail文件的作用可另行查阅资料。

mapkey purge @MAPKEY\_NAMEpurge;@MAPKEY\_LABELpurge;\

mapkey(continued) @SYSTEM/opt/PTC/wf3/bin/purge;

这两行是设置了一个清理工作目录下所有旧版本文件的快捷键,打开ProE后,在自定义窗口把快捷键拖到工具栏即可使用。

pro\_unit\_length 和 pro\_unit\_mass 是设置默认单位。

show\_shaded\_edges 和 tangent\_edge\_display 是线的显示设置。

不显示旋转中心:

spin\_center\_display no

ProE重定向时候的显示设置(旋转实体时),后一个是旋转时候显示基准轴和基准面的:

spin\_with\_orientation\_center no

spin\_with\_part\_entities yes

提示消息的行数:

visible\_message\_lines 3

其他的都是一些配置文件的路径设置和显示设置。

save\_drawing\_picture\_file embed是设置预览文件嵌入到工程图中,点open打开文件时候就可以预览工程图了,使用该配置后,工程图的文件大小会增大很多。建议不使用。

卸载ProE:

sudo rm -rf /opt/PTC/*

sudo rm -rf /opt/PTC/.* > /dev/null 2>\&1

sudo -y rmdir /opt/PTC

然后再检查一下用户目录 ~{}/ 内有没有残留的配置文件。

可能存在的问题:

运行ProE后可能会有如下的提示:

Pro/Engineer could not initialize the specified locale "en\_US" and has reverted to running in English. Please check the locale settings before re-launching Pro/ENGINEER. 

前面制作脚本那里提及的,环境变量的LANG语言随便设置都可以,因为安装的就是英文版的,这个提示不影响使用。中文系统用LANG=C时候不出现提示。


※ 修改:·oicu 于 Sep  5 17:51:54 2010 修改本文·[FROM: 蓝色☆空]

※ 来源:·四川大学蓝色星空站 http://lsxk.org·[FROM: 蓝色☆空]

\bibliographystyle{plainnat}
\bibliography{gk}
\clearpage



