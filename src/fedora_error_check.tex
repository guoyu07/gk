\chapter{Fedora排错}

系统日志查看方法\cite{fedora_error_check}

cat

tail -f

日志文件说明

/var/log/message~系统启动后的信息和错误日志,是Red Hat Linux中最常用的日志之一

/var/log/secure~与安全相关的日志信息

/var/log/maillog~与邮件相关的日志信息

/var/log/cron~与定时任务相关的日志信息

/var/log/spooler~与UUCP和news设备相关的日志信息

/var/log/boot.log~守护进程启动和停止相关的日志消息

系统:

\# uname -a \# 查看内核/操作系统/CPU信息

\# cat /etc/issue

\# cat /etc/redhat-release \# 查看操作系统版本 Enterprise Linux Enterprise Linux Server release 5.1 (Carthage)企业Linux服务器版本迦太基

\# cat /proc/cpuinfo \# 查看CPU信息

\$ file /bin/ls 查看linux os是32位还是64位的简单方法

\# hostname \# 查看计算机名

\$ indent c代码格式化工具,使用很广泛

\$ lsb\_release -a 查看linux内核版本和发行版的版本

\$ ls /proc/sys/fs/inotify 

如果显示为\verb|max_queued_events max_user_instances max_user_watches|

那么说明内核支持inotfiy

\# lspci -tv \# 列出所有PCI设备

\# lsusb -tv \# 列出所有USB设备

\# lsmod \# 列出加载的内核模块

\# env \# 查看环境变量

资源:

\# free -m \# 查看内存使用量和交换区使用量

\# df -h \# 查看各分区使用情况

\# du -sh <目录名> \# 查看指定目录的大小

\# grep MemTotal /proc/meminfo \# 查看内存总量

\# grep MemFree /proc/meminfo \# 查看空闲内存量

\# uptime \# 查看系统运行时间、用户数、负载

\# cat /proc/loadavg \# 查看系统负载

磁盘和分区:

\# mount | column -t \# 查看挂接的分区状态

\# fdisk -l \# 查看所有分区

\# swapon -s \# 查看所有交换分区

\# hdparm -i /dev/hda \# 查看磁盘参数(仅适用于IDE设备)

\# dmesg | grep IDE \# 查看启动时IDE设备检测状况

网络:

\# ifconfig \# 查看所有网络接口的属性

\# iptables -L \# 查看防火墙设置

\# route -n \# 查看路由表

\# netstat -lntp \# 查看所有监听端口

\# netstat -antp \# 查看所有已经建立的连接

\# netstat -s \# 查看网络统计信息

\# iptables -I INPUT -p tcp -s 要封的IP --dport 22 -j DROP

封杀访问22端口的恶意IP

\# ssh-copy-id -i ~/.ssh/id\_rsa.pub "-p 12345 user@server"

ssh-keygen打通Linux登录,可以快速把公钥传送到目标服务器上,避免粘贴复制中出错

ssh key无法登录Linux服务器\cite{linux_wiki}

\begin{compactenum}
\item 检查目标服务器/etc/ssh/sshd\_config中的RSAAuthentication/PubkeyAuthentication配置是否打开
\item 注意看/etc/ssh/sshd/sshd\_config中AuthorizedKeysFile是如何定义的,约定俗成的定义是 .ssh/authorized\_keys,但是有些奇葩管理员会修改这个配置,导致无论如何编辑authorizedkeys文件都不会成功
\item 反复查看登录命令中的ip是否填对,这是一种常见的低级错误
\end{compactenum}

进程:

\# ps -ef \# 查看所有进程

\# top \# 实时显示进程状态(另一篇文章里面有详细的介绍)

用户:

\# w \# 查看活动用户

\# id <用户名> \# 查看指定用户信息

\# last \# 查看用户登录日志

\# cut -d: -f1 /etc/passwd \# 查看系统所有用户

\# cut -d: -f1 /etc/group \# 查看系统所有组

\# crontab -l \# 查看当前用户的计划任务

服务:

\# chkconfig –list \# 列出所有系统服务

\# chkconfig –list | grep on \# 列出所有启动的系统服务

程序:

\# rpm -qa \# 查看所有安装的软件包

\# update-alternatives --config editor


\bibliographystyle{plainnat}
\bibliography{gk}
\clearpage

