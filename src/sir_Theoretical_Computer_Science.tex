\chapter{理论计算机科学漫谈}


******************************************************************
  
版权声明:本文作者sir系旅美学人、南京大学校友。 

为了学术或 教育的(非营利)目的,在保留本版权

声明的情况下,您可以自由 转载本文的电子版。

如果您要在传统媒体上转载此文,请与南京大学

小百合BBS站上的网友sir联系。 

****************************************************************** 




\begin{center}\textbf{理论计算机科学漫谈}\end{center}




早就答应russel的,今天有点时间,把欠债还上。 

计算机科学和数学的关系有点奇怪。二三十年以前,计算机科学基本上还是数学的一个分支。而现在,计算机科学拥有广泛的研究领域和众多的研究人员,在很多方面反过来推动数学发展,从某种意义上可以说是孩子长得比妈妈还高了。 

但不管怎么样,这个孩子身上始终流着母亲的血液。这血液是the mathematical underpinning of computer science(计算机科学的数学基础),-- 也就是理论计算机科学。 

现代计算机科学和数学的另一个交叉是计算数学/数值分析/科学计算,传统上不包含在理论计算机科学以内。所以本文对计算数学全部予以忽略。 

最常和理论计算机科学放在一起的一个词是什么? 答:离散数学。这两者的关系是如此密切,以至于它们在不少场合下成为同义词。 

传统上,数学是以分析为中心的。数学系的同学要学习三四个学期的数学分析,然后是复变,实变,泛函等等。实变和泛函被很多人认为是现代数学的入门。在物理,化学,工程上应用的,也以分析为主。 

随着计算机科学的出现,一些以前不太受到重视的数学分支突然重要起来。人们发现,这些分支处理的数学对象与传统的分析有明显的区别:分析研究的对象是连续的,因而微分,积分成为基本的运算;而这些分支研究的对象是离散的,因而很少有机会进行此类的计算。人们从而称这些分支为“离散数学”。“离散数学”的名字越来越响亮,最后导致以分析为中心的传统数学分支被相对称为“连续数学”。 

离散数学经过几十年发展,基本上稳定下来。一般认为,离散数学包含以下学科: 

\begin{compactenum}
\item 集合论,数理逻辑与元数学。这是整个数学的基础,也是计算机科学的基础。 
\item 图论,算法图论;组合数学,组合算法。计算机科学,尤其是理论计算机科学的核心是算法,而大量的算法建立在图和组合的基础上 
\item 抽象代数。代数是无所不在的,本来在数学中就非常重要。在计算机科学中,人们惊讶地发现代数竟然有如此之多的应用。 
\end{compactenum}



但是,理论计算机科学仅仅就是在数学的上面加上“离散”的帽子这么简单吗?一直到大约十几年前,终于有一位大师告诉我们:不是。 

D.E.Knuth(他有多伟大,我想不用我废话了)在Stanford开设了一门全新的课程Concrete Mathematics。 Concrete这个词在这里有两层含义: 

第一,针对abstract而言。Knuth认为,传统数学研究的对象过于抽象,导致对具体的问题关心不够。他抱怨说,在研究中他需要的数学往往并不存在,所以他只能自己去创造一些数学。为了直接面向应用的需要,他要提倡“具体”的数学。 

在这里我做一点简单的解释。例如在集合论中,数学家关心的都是最根本的问题--公理系统的各种性质之类。而一些具体集合的性质,各种常见集合,关系,映射都是什么样的,数学家觉得并不重要。然而,在计算机科学中应用的,恰恰就是这些具体的东西。Knuth能够首先看到这一点,不愧为当世计算机第一人。 

第二,Concrete是Continuous(连续)加上discrete (离散)。不管连续数学还是离散数学,都是有用的数学! 

前面主要是从数学角度来看的。从计算机角度来看,理论计算机科学目前主要的研究领域包括:可计算性理论,算法设计与复杂性分析,密码学与信息安全,分布式计算理论,并行计算理论,网络理论,生物信息计算,计算几何学,程序语言理论等等。这些领域互相交叉,而且新的课题在不断提出,所以很难理出一个头绪来。下面随便举一些例子。 

由于应用需求的推动,密码学现在成为研究的热点。密码学建立在数论(尤其是计算数论),代数,信息论,概率论和随机过程的基础上,有时也用到图论和组合学等。 

很多人以为密码学就是加密解密,而加密就是用一个函数把数据打乱。这就大错特错了。现代密码学至少包含以下层次的内容: 

第一,密码学的基础。例如,分解一个大数真的很困难吗?能否有一般的工具证明协议正确? 

第二,密码学的基本课题。例如,比以前更好的单向函数,签名协议等。 

第三,密码学的高级问题。例如,零知识证明的长度,秘密分享的方法。 

第四,密码学的新应用。例如,数字现金,叛徒追踪等。 

在分布式系统中,也有很多重要的理论问题。例如,进程之间的同步,互斥协议。一个经典的结果是:在通信信道不可靠时,没有确定型算法能实现进程间协同。所以,改进TCP三次握手几乎没有意义。例如时序问题。常用的一种序是因果序,但因果序直到不久前才有一个理论上的结果.... 

例如,死锁没有实用的方法能完美地对付。  

例如,...... 

  

关于死锁 Re: 理论计算机科学漫谈(6) 
  

我简单地觉得与“熵”这个东西有关. 没有这么复杂。关键在效率:对付死锁的方法,例如死锁检测,都非常严重地减低效率,以至于得不尝失,因为死锁并不是一种经常出现的现象。所以在全局上,一般都用所谓“鸵鸟算法”,也就是假装什么都不会发生。在局部上,例如你要设计一个访问共享数据的算法,那么你就要证明你的算法在局部上是deadlock free。至于它会不会导致全局的死锁,就烦不了许多了。 





发信人: sir (sir), 信区: Mathematics. 本篇人气: 8984

标  题: 理论计算机科学漫谈(1)\cite{tcs1}

发信站: 南大小百合 (Thu Nov 30 11:08:08 2000) , 转信

\textbf{理论计算机科学漫谈(1)}



早就答应russel的,今天有点时间,把欠债还上。

计算机科学和数学的关系有点奇怪。二三十年以前,计算机科学基本上还是数学的一个分支。而现在,计算机科学拥有广泛的研究领域和众多的研究人员,在很多方面反过来推动数学发展,从某种意义上可以说是孩子长得比妈妈还高了。

但不管怎么样,这个孩子身上始终流着母亲的血液。这血液是the mathematical underpinning of computer science(计算机科学的数学基础),-- 也就是理论计算机科学。

现代计算机科学和数学的另一个交叉是计算数学/数值分析/科学计算,传统上不包含在理论计算机科学以内。所以本文对计算数学全部予以忽略。

\textbf{理论计算机科学漫谈(2)}

发信人: sir (sir), 信区: Mathematics. 本篇人气: 1950

标  题: 理论计算机科学漫谈(2)\cite{tcs2}

发信站: 南大小百合 (Thu Nov 30 11:23:19 2000) , 转信

最常和理论计算机科学放在一起的一个词是什么?答:离散数学。这两者的关系是如此密切,以至于它们在不少场合下成为同义词。

传统上,数学是以分析为中心的。数学系的同学要学习三四个学期的数学分析,然后是复变,实变,泛函等等。实变和泛函被很多人认为是现代数学的入门。在物理,化学,工程上应用的,也以分析为主。

随着计算机科学的出现,一些以前不太受到重视的数学分支突然重要起来。人们发现,这些分支处理的数学对象与传统的分析有明显的区别:分析研究的对象是连续的,因而微分,积分成为基本的运算;而这些分支研究的对象是离散的,因而很少有机会进行此类的计算。人们从而称这些分支为“离散数学”。“离散数学”的名字越来越响亮,最后导致以分析为中心的传统数学分支被相对称为“连续数学”。

\textbf{理论计算机科学漫谈(3)}


发信人: sir (sir), 信区: Mathematics. 本篇人气: 1549

标  题: 理论计算机科学漫谈(3)\cite{tcs3}

发信站: 南大小百合 (Thu Nov 30 11:30:37 2000) , 转信


离散数学经过几十年发展,基本上稳定下来。一般认为,离散数学包含以下学科:

1) 集合论,数理逻辑与元数学。这是整个数学的基础,也是计算机科学的基础。

2) 图论,算法图论;组合数学,组合算法。计算机科学,尤其是理论计算机科学的核心是算法,而大量的算法建立在图和组合的基础上。

3) 抽象代数。代数是无所不在的,本来在数学中就非常重要。在计算机科学中,人们惊讶地发现代数竟然有如此之多的应用。


\textbf{理论计算机科学漫谈(4)}

发信人: sir (sir), 信区: Mathematics. 本篇人气: 1381

标  题: 理论计算机科学漫谈(4)\cite{tcs4}

发信站: 南大小百合 (Thu Nov 30 11:44:35 2000) , 转信


但是,理论计算机科学仅仅就是在数学的上面加上“离散”的帽子这么简单吗?一直到大约十几年前,终于有一位大师告诉我们:不是。

D.E.Knuth(他有多伟大,我想不用我废话了)在Stanford开设了一门全新的课程Concrete Mathematics。 Concrete这个词在这里有两层含义:

第一,针对abstract而言。Knuth认为,传统数学研究的对象过于抽象,导致对具体的问题关心不够。他抱怨说,在研究中他需要的数学往往并不存在,所以他只能自己去创造一些数学。为了直接面向应用的需要,他要提倡“具体”的数学。

在这里我做一点简单的解释。例如在集合论中,数学家关心的都是最根本的问题--公理系统的各种性质之类。而一些具体集合的性质,各种常见集合,关系,映射都是什么样的,数学家觉得并不重要。然而,在计算机科学中应用的,恰恰就是这些具体的东西。Knuth能够首先看到这一点,不愧为当世计算机第一人。

第二,Concrete是Continuous(连续)加上discrete(离散)。不管连续数学还是离散数学,都是有用的数学!


\textbf{理论计算机科学漫谈(5)}



发信人: sir (sir), 信区: Mathematics. 本篇人气: 1269

标  题: 理论计算机科学漫谈(5)\cite{tcs5}

发信站: 南大小百合 (Thu Nov 30 12:09:50 2000) , 转信


前面主要是从数学角度来看的。从计算机角度来看,理论计算机科学目前主要的研究领域包括:可计算性理论,算法设计与复杂性分析,密码学与信息安全,分布式计算理论,并行计算理论,网络理论,生物信息计算,计算几何学,程序语言理论等等。这些领域互相交叉,而且新的课题在不断提出,所以很难理出一个头绪来。

下面随便举一些例子。

由于应用需求的推动,密码学现在成为研究的热点。密码学建立在数论(尤其是计算数论),代数,信息论,概率论和随机过程的基础上,有时也用到图论和组合学等。

很多人以为密码学就是加密解密,而加密就是用一个函数把数据打乱。这就大错特错了。

现代密码学至少包含以下层次的内容:

第一,密码学的基础。例如,分解一个大数真的很困难吗?能否有一般的工具证明协议正确?

第二,密码学的基本课题。例如,比以前更好的单向函数,签名协议等。

第三,密码学的高级问题。例如,零知识证明的长度,秘密分享的方法。

第四,密码学的新应用。例如,数字现金,叛徒追踪等。

\textbf{理论计算机科学漫谈(6)}

发信人: sir (sir), 信区: Mathematics. 本篇人气: 1292

标  题: 理论计算机科学漫谈(6)\cite{tcs6}

发信站: 南大小百合 (Thu Nov 30 12:18:32 2000) , 转信



在分布式系统中,也有很多重要的理论问题。

例如,进程之间的同步,互斥协议。一个经典的结果是:在通信信道不可靠时,没有确定型算法能实现进程间协同。所以,改进TCP三次握手几乎没有意义。

例如时序问题。常用的一种序是因果序,但因果序直到不久前才有一个理论上的结果....
..

例如,死锁没有实用的方法能完美地对付。

例如,......

【 在 pie (燃烧吧,小宇宙!) 的大作中提到: 】

: 【 在 probe (农民) 的大作中提到: 】

: 我简单地觉得与“熵”这个东西有关

没有这么复杂。关键在效率:对付死锁的方法,例如死锁检测,都非常严重地减低效率,以至于得不尝失,因为死锁并不是一种经常出现的现象。所以在全局上,一般都用所谓“鸵鸟算法”,也就是假装什么都不会发生。在局部上,例如你要设计一个访问共享数据的算法,那么你就要证明你的算法在局部上是deadlock free。至于它会不会导致全局的死锁,就烦不了许多了。





\textbf{关于计算机学习}

发信人: sir (sir), 信区: Mathematics. 本篇人气: 763

标  题: 回答pie关于计算机学习

发信站: 南大小百合 (Fri Dec  1 07:28:49 2000) , 转信

其实,我也是从那个迷茫的年代里走过来的。计算机是一个新兴学科,亦理亦工,其教学在国际上都不成熟,更不用说与国际水平差距很大的国内教学了。

我个人的看法:

首先,如果要做research,就要把基础打好。数学是计算机系学生的内功,学好数学,无论做什么都胆子壮一点。我们以前学习的那一点数学远远不够。至少,下面列举的数学是重要的:代数(群环域,布尔代数和多项式理论),数论和计算数论,概率论,数理统计和随机过程,图论和算法图论,组合分析,数值分析,数理逻辑和集合论,数学基础,信息论,博弈论,线性和非线性规划。

随便举个例子:学数字通信时,会学到很多编码。你感到很枯燥,也难以记住,更不知道它有什么用处。但事实上,这些编码全部都有深刻的代数背景。如果你懂得其原理,就会感到学术的美妙。

其次,大量使用hack是工程学科的特点。无论你是否做研究,你都算是一个工程师。你必须理解,hack总是在理论不起作用时让我们避免麻烦。hack的背景是经验,所以你刚接触时很难理解它,即使你聪明过人。对策是多读相关的背景材料和多动手实践,尤其是动手。

举例说:你能理解为什么IP路由一定要搞成这样?看看RFC,再自己动手编点程序试验一下。我认识的师兄中里就有超级大牛,他如何成为大牛的?不就是这样一点点积累出来的?
你如果捧着课本想,永远也想不通。

第三,天才的想法总是少数,大部份工作是平凡的。但是,天才的想法往往是在大量平凡工作的基础上产生的。我看别人的文章,总觉得naive。但三篇naive的文章积累起来,其进步就不是你随便能想到的。做工作要从最基本的着手。

我的看法不见得正确。欢迎列位高手指正。



: 这里我有一点感慨:

: 虽然读了几年大学,我却不知道应该学什么。

: 因为我所见的所谓knowledge实在是一系列的technique和trick

: Sience是什么?我渴求而又看不见.

: 阿Sir学长,Is naivete necessary in the research for science?

: I am quite curious.

\chapter{胡侃学习(理论)计算机}


******************************************************************  

版权声明:本文作者sir系旅美学人、南京大学校友。 

为了学术或 教育的(非营利)目的,在保留本版权

声明的情况下,您可以自由 转载本文的电子版。

如果您要在传统媒体上转载此文,请与南京大学

小百合BBS站上的网友sir联系。 

****************************************************************** 

出处:http://bbs.sjtu.edu.cn/bbstcon?board=CS\&reid=1178579072


我也来冒充一回高手,谈谈学习计算机的一点个人体会。由于我是做理论的,所以先着重谈谈理论。 

记得当年大一,刚上本科的时候,每周六课时数学分析,六课时高等代数,天天作业不断(那时是六日工作制)。颇有些同学惊呼走错了门:咱们这到底念的是什么系?不错,你没走错门,这就是(当时的)南大计算机系。系里的传统是培养做学术研究,尤其是理论研究的人。而计算机的理论研究,说到底了就是数学,虽然也许是正统数学家眼里非主流的数学。 

数学分析这个东东,咱们学计算机的人对它有很复杂的感情。爱它在于它是第一门,也是学分最多的一门数学课,又长期为考研课程--94以前可以选考数学分析与高等代数,以后则并轨到著名的所谓“工科数学一”。其重要性可见一斑。恨它则在于它好象难得有用到的机会,而且思维跟咱们平常做的这些离散/有限的工作截然不同。当年出现的怪现象是:计算机系学生的高中数学基础在全校数一数二(希望没有冒犯其它系的同学),教学课时数也仅次于数学系,但学完之后的效果却几乎是倒数第一。其中原因何在,发人深思。 

我个人的浅见是:计算机类的学生,对数学的要求固然跟数学系不同,跟物理类差别则更大。通常非数学专业的所谓“高等数学”,无非是把数学分析中较困难的理论部分删去,强调套用公式计算而已。而对计算机系来说,数学分析里用处最大的恰恰是被删去的理论部分。说得难听一点,对计算机系学生而言,追求算来算去的所谓“工科数学一”已经彻底地走进了魔道。记上一堆曲面积分的公式,难道就能算懂了数学分析? 

中文的数学分析书,一般都认为以北大张筑生老师的“数学分析新讲”为最好。我个人认为南大数学系的“数学分析教程”也还不错,至少属于典型的南大风格,咱们看着亲切。随便学通哪一本都行。万一你的数学实在太好,这两本书都吃不饱,那就去看菲赫金哥尔茨的“微积分学教程”好了--但我认为没什么必要,毕竟你不想转到数学系去。 

吉米多维奇的“数学分析习题集”也基本上是计算型的东东。如果你打算去考那个什么“工科数学一”,可以做一做。否则,不做也罢。 

中国的所谓高等代数,就等于线性代数加上一点多项式理论。我以为这有好的一面,因为可以让学生较早感觉到代数是一种结构,而非一堆矩阵翻来覆去。当年我们用林成森,盛松柏两位老师编的“高等代数”,感觉相当舒服,我直到现在还保留着教材。此书相当全面地包含了关于多项式和线性代数的基本初等结果,同时还提供了一些有用的比较深的内容,如Sturm序列,Shermon-Morrison公式,广义逆矩阵等等。可以说,作为本科生如能吃透此书,就可以算高手。后来它得以在南大出版社出版,可惜好象并轨以后就没有再用了。 

国内较好的高等代数教材还有清华计算机系用的那本,清华出版社出版,书店里多多,一看就知道。特点嘛,跟南大那本差不太多。 

但以上两本书也不能说完美无缺。从抽象代数的观点来看,高等代数里的结果不过是代数系统性质的一些例子而已。莫宗坚先生的“代数学”里,对此进行了深刻的讨论。然而莫先生的书实在深得很,作为本科生恐怕难以接受,不妨等到自己以后成熟了一些再读。 

概率论与数理统计这门课很重要,可惜少了些东西。 

少了的东西是随机过程。到毕业还没有听说过Markov过程,此乃计算机系学生的耻辱。没有随机过程,你怎么分析网络和分布式系统?怎么设计随机化算法和协议?据说清华计算机系开有“随机数学”,早就是必修课。人家可是工科学校,作为自以为“理科计算机系”出身的人,我感到惭愧。 

另外,离散概率对计算机系学生来说有特殊的重要性。现在,美国已经有些学校开设了单纯的“离散概率论”课程,干脆把连续概率删去,把离散概率讲深些。我们不一定要这么做,但应该更加强调离散概率是没有疑问的。 

计算方法是最后一门由数学系给我们开的课。一般学生对这门课的重视程度有限,以为没什么用。其实,做图形图像可离不开它。而且,在很多科学工程中的应用计算,都以数值的为主。 

这门课有两个极端的讲法:一个是古典的“数值分析”,完全讲数学原理和算法;另一个是现在日趋流行的“科学与工程计算”,干脆教学生用软件包编程。南大数学系的几位老师做了件大好事,把前者的一本极为经典的教材翻译出版了:德国Stoer的“数值分析引论”。如果你能学会此书中最浅显的三分之一,就算没有白上过计算方法这门课!而后一种讲法似乎国内还没有跟上潮流?不过,只要你有机会在自己的电脑上装个matlab之类,完全可以无师自通。 

本系里,通常开一门离散数学,包括集合论,图论,和抽象代数,另外再单开一门数理逻辑。这样安排,主要由于南大的逻辑传统:系里很多老师都算莫先生的门人,就连孙先生都是逻辑专业出身(见孙先生自述)。 

不过,这么多内容挤在离散数学一门课里,是否时间太紧了点?另外,计算机系学生不懂组合和数论,也是巨大的缺陷。要做理论,不懂组合或者数论吃亏可就太大了。 

从理想的状态来看,最好分开六门课:集合,逻辑,图论,组合,代数,数论。这个当然不现实,因为没那么多课时。也许将来可以开三门课:集合与逻辑,图论与组合,代数与数论。 

不管课怎么开,学生总一样要学。下面分别谈谈上面的三组内容。 

古典集合论,北师大出过一本“基础集合论”不错。南大出版朱梧(木贾)老师的“集合论导引”也许观点更高些,但他的书形式化得太厉害,念起来吃力。 

数理逻辑,莫先生的书自然是经典。然而我们也不得不承认,此书年代久远,光读它恐怕不够。尤其是命题/谓词演算本身有好多种不同的讲法,多看几家能大大开阔自己的视野。例如陆钟万老师的“面向计算机科学的数理逻辑”就不错。朱老师也著有“数理逻辑教程”一书,但也同样读起来费力些。 

总的来说,学集合/逻辑起手不难,但越往后越感觉深不可测。建议有兴趣的同学读读朱老师的“数学基础引论”--此书有点时间简史的风格,讲到精彩处,所谓“天花乱坠,妙雨缤纷”,令人目不暇接。读完以后,你对这些数学/哲学中最根本的问题有了个大概了解,也知道了山有多高,海有多深。 

学完以上各书之后,如果你还有精力兴趣进一步深究,那么可以试一下GTM系列中的"Introduction to Axiomatic Set Theory"和"A Course of Mathematical Logic"。这两本都有世界图书的引进版。你如果能搞定这两本,可以说在逻辑方面真正入了门,也就不用再浪费时间听我瞎侃了。:) 

据说全中国最多只有三十个人懂图论(当年上课时陈道蓄老师转引张克民老师的话)。此言不虚。图论这东东,技巧性太强,几乎每题都有一个独特的方法,让人头痛。不过这也正是它魅力所在:只要你有创造性,它就能给你成就感。所以学图论没什么好说的,做题吧。 

国内的图论书中,王树禾老师的“图论及其算法”非常成功。一方面,其内容在国内教材里算非常全面的。另一方面,其对算法的强调非常适合计算机系(本来就是科大计算机系教材)。有了这本书为主,再参考几本翻译的,如Bondy\&Murty的“图论及其应用”,邮电出版社翻译的“图论和电路网络”等等,就马马虎虎,对本科生足够了。 

再进一步,世界图书引进有GTM系列的"ModernGraph Theory"。此书确实经典!国内好象还有一家出版了个翻译版。不过,学到这个层次,还是读原版好。搞定这本书,也标志着图论入了门,呵呵。组合感觉没有太适合的国产书。还是读Graham和Knuth 等人合著的经典“具体数学”吧,有翻译版,西电出的。 

抽象代数,国内经典为莫宗坚先生的“代数学”。此书是北大数学系教材,深得好评。然而对本科生来说,此书未免太深。可以先学习一些其它的教材,然后再回头来看“代数学”。国际上的经典可就多了,GTM系列里就有一大堆。推荐一本谈不上经典,但却最简单的,最容易学的:\href{http://www.math.miami.edu/~ec/book/}{http://www.math.miami.edu/\~{}ec/book/}

这本“Introduction to Linear and Abstract Algebra"非常通俗易懂,而且把抽象代数和线性代数结合起来,对初学者来说非常理想。不过请注意版权问题,不要违反法律噢。 

数论方面,国内有经典而且以困难著称的”初等数论“(潘氏兄弟著,北大版)。再追溯一点,还有更加经典(可以算世界级)并且更加困难的”数论导引“(华罗庚先生的名著,科学版,九章书店重印)。把基础的几章搞定一个大概,对本科生来讲足够了。但这只是初等数论。本科毕业后要学计算数论,你必须看英文的书,如Bach的"Introduction to Algorithmic Number Theory"。理论计算机的根本,在于算法。现在系里给本科生 

开设算法设计与分析,确实非常正确。环顾西方世界,大约没有一个三流以上计算机系不把算法作为必修的。 

算法教材目前公认以Corman等著的"Introduction to Algorithms"为最优。对入门而言,这一本已经足够,不需要再参考其它书。南大曾翻译出版此书,中文名为”现代计算机常用数据结构与算法“。pie好象提供了网上课程的link,我也就不用废话。 

最后说说形式语言与自动机。我们用过北邮的教材,应该说写的还清楚。但是,有一点要强调:形式语言和自动机的作用主要在作为计算模型,而不是用来做编译。事实上,编译前端已经是死领域,没有任何open problem。如果为了这个,我们完全没必要去学形式语言--用用yacc什么的就完了。北邮的那本,在深度上,在跟可计算性的联系上都有较大的局限,现代感也不足。所以建议有兴趣的同学去读英文书......不过英文书中好的也不多,而且国内似乎没引进这方面的教材。 

入门以后,把形式语言与自动机中定义的模型,和数理逻辑中用递归函数定义的模型比较一番,可以说非常有趣。现在才知道,什么叫”宫室之美,百官之富“! 





\textbf{理论计算机科学漫谈(0)}

发信人: sir (阿涩), 信区: Mathematics. 本篇人气: 5761

标  题: 胡侃学习(理论)计算机(0)\cite{sir0}

发信站: 南京大学小百合站 (Mon Oct  8 03:57:41 2001), 站内信件

我也来冒充一回高手,谈谈学习计算机的一点个人体会。

由于我是做理论的,所以先着重谈谈理论。

记得当年大一,刚上本科的时候,每周六课时数学分析,六课时高等代数,天天作业不断(那时是六日工作制)。颇有些同学惊呼走错了门:咱们这到底念的是什么系?
不错,你没走错门,这就是(当时的)南大计算机系。系里的传统是培养做学术研究,尤其是理论研究的人。而计算机的理论研究,说到底了就是数学,虽然也许是正统数学家眼里非主流的数学。


\textbf{理论计算机科学漫谈(1)}

发信人: sir (阿涩), 信区: Mathematics. 本篇人气: 1452

标  题: 胡侃学习(理论)计算机(1)\cite{sir1}

发信站: 南京大学小百合站 (Mon Oct  8 03:58:55 2001), 站内信件



数学分析这个东东,咱们学计算机的人对它有很复杂的感情。爱它在于它是第一门,也是学分最多的一门数学课,又长期为考研课程--94以前可以选考数学分析与高等代数,以后则并轨到著名的所谓“工科数学一”。
其重要性可见一斑。恨它则在于它好象难得有用到的机会,而且思维跟咱们平常做的这些离散/有限的工作截然不同。当年出现的怪现象是:计算机系学生的高中数学基础在全校数一数二(希望没有冒犯其它系的同学),
教学课时数也仅次于数学系,但学完之后的效果却几乎是倒数第一。其中原因何在,发人深思。

我个人的浅见是:计算机类的学生,对数学的要求固然跟数学系不同,跟物理类差别则更大。通常非数学专业的所谓“高等数学”,无非是把数学分析中较困难的理论部分删去,强调套用公式计算而已。而对计算机系来说,
数学分析里用处最大的恰恰是被删去的理论部分。说得难听一点,对计算机系学生而言,追求算来算去的所谓“工科数学一”已经彻底地走进了魔道。记上一堆曲面积分的公式,难道就能算懂了数学分析?

中文的数学分析书,一般都认为以北大张筑生老师的“数学分析新讲”为最好。我个人认为南大数学系的“数学分析教程”也还不错,至少属于典型的南大风格,咱们看着亲切。随便学通哪一本都行。万一你的数学实在
太好,这两本书都吃不饱,那就去看菲赫金哥尔茨的“微积分学教程”好了--但我认为没什么必要,毕竟你不想转到数学系去。

吉米多维奇的“数学分析习题集”也基本上是计算型的东东。如果你打算去考那个什么“工科数学一”,可以做一做。否则,不做也罢。


\textbf{胡侃学习(理论)计算机(2)}

发信人: sir (阿涩), 信区: Mathematics. 本篇人气: 1157

标  题: 胡侃学习(理论)计算机(2)\cite{sir2}

发信站: 南京大学小百合站 (Mon Oct  8 04:01:46 2001), 站内信件

中国的所谓高等代数,就等于线性代数加上一点多项式理论。我以为这有好的一面,因为可以让学生较早感觉到代数是一种结构,而非一堆矩阵翻来覆去。当年我们用林成森,盛松柏两位老师编的“高等代数”,感觉相当
舒服,我直到现在还保留着教材。此书相当全面地包含了关于多项式和线性代数的基本初等结果,同时还提供了一些有用的比较深的内容,如Sturm序列,Shermon-Morrison公式,广义逆矩阵等等。可以说,作为本科
生如能吃透此书,就可以算高手。后来它得以在南大出版社出版,可惜好象并轨以后就没有再用了。

国内较好的高等代数教材还有清华计算机系用的那本,清华出版社出版,书店里多多,一看就知道。特点嘛,跟南大那本差不太多。

但以上两本书也不能说完美无缺。从抽象代数的观点来看,高等代数里的结果不过是代数系统性质的一些例子而已。莫宗坚先生的“代数学”里,对此进行了深刻的讨论。然而莫先生的书实在深得很,作为本科生恐怕难以接受,不妨等到自己以后成熟了一些再读。

\textbf{胡侃学习(理论)计算机(3)}

发信人: sir (阿涩), 信区: Mathematics. 本篇人气: 1058

标  题: 胡侃学习(理论)计算机(3)\cite{sir3}

发信站: 南京大学小百合站 (Mon Oct  8 04:02:32 2001), 站内信件

概率论与数理统计这门课很重要,可惜少了些东西。

少了的东西是随机过程。到毕业还没有听说过Markov过程,此乃计算机系学生的耻辱。没有随机过程,你怎么分析网络和分布式系统?怎么设计随机化算法和协议?据说清华计算机系开有“随机数学”,早就是必修课。人家可是工科学校,作为自以为“理科计算机系”出身的人,我感到惭愧。

另外,离散概率对计算机系学生来说有特殊的重要性。现在,美国已经有些学校开设了单纯的“离散概率论”课程,干脆把连续概率删去,把离散概率讲深些。我们不一定要这么做,但应该更加强调离散概率是没有疑问的。

\textbf{胡侃学习(理论)计算机(4)}

发信人: sir (阿涩), 信区: Mathematics. 本篇人气: 1002

标  题: 胡侃学习(理论)计算机(4)\cite{sir4}

发信站: 南京大学小百合站 (Mon Oct  8 04:03:13 2001), 站内信件


计算方法是最后一门由数学系给我们开的课。一般学生对这门课的重视程度有限,以为没什么用。其实,做图形图像可离不开它。而且,在很多科学工程中的应用计算,都以数值的为主。

这门课有两个极端的讲法:一个是古典的“数值分析”,完全讲数学原理和算法;另一个是现在日趋流行的“科学与工程计算”,干脆教学生用软件包编程。南大数学系的几位老师做了件大好事,把前者的一本极为经典的教材翻译出版了:德国Stoer的“数值分析引论”。如果你能学会此书中最浅显的三分之一,就算没有白上过计算方法这门课!而后一种讲法似乎国内还没有跟上潮流?不过,只要你有机会在自己的电脑上装个matlab之类,完全可以无师自通。


\textbf{胡侃学习(理论)计算机(5)}


发信人: sir (阿涩), 信区: Mathematics. 本篇人气: 954

标  题: 胡侃学习(理论)计算机(5)\cite{sir5}

发信站: 南京大学小百合站 (Mon Oct  8 04:03:48 2001), 站内信件


本系里,通常开一门离散数学,包括集合论,图论,和抽象代数,另外再单开一门数理逻辑。这样安排,主要由于南大的逻辑传统:系里很多老师都算莫先生的门人,就连孙先生都是逻辑专业出身(见孙先生自述)。

不过,这么多内容挤在离散数学一门课里,是否时间太紧了点?另外,计算机系学生不懂组合和数论,也是巨大的缺陷。要做理论,不懂组合或者数论吃亏可就太大了。

从理想的状态来看,最好分开六门课:集合,逻辑,图论,组合,代数,数论。这个当然不现实,因为没那么多课时。也许将来可以开三门课:集合与逻辑,图论与组合,代数与数论。

不管课怎么开,学生总一样要学。下面分别谈谈上面的三组内容。


\textbf{胡侃学习(理论)计算机(6)}


发信人: sir (阿涩), 信区: Mathematics. 本篇人气: 906

标  题: 胡侃学习(理论)计算机(6)\cite{sir6}

发信站: 南京大学小百合站 (Mon Oct  8 04:04:39 2001), 站内信件


古典集合论,北师大出过一本“基础集合论”不错。南大出版朱梧(木贾)老师的“集合论导引”也许观点更高些,但他的书形式化得太厉害,念起来吃力。

数理逻辑,莫先生的书自然是经典。然而我们也不得不承认,此书年代久远,光读它恐怕不够。尤其是命题/谓词演算本身有好多种不同的讲法,多看几家能大大开阔自己的视野。例如陆钟万老师的“面向计算机科学的数理逻辑”就不错。朱老师也著有“数理逻辑教程”一书,但也同样读起来费力些。

总的来说,学集合/逻辑起手不难,但越往后越感觉深不可测。建议有兴趣的同学读读朱老师的“数学基础引论”--此书有点时间简史的风格,讲到精彩处,所谓“天花乱坠,妙雨缤纷”,令人目不暇接。读完以后,你对这些数学/哲学中最根本的问题有了个大概了解,也知道了山有多高,海有多深。

学完以上各书之后,如果你还有精力兴趣进一步深究,那么可以试一下GTM系列中的"Introduction to Axiomatic Set Theory"和"A Course of Mathematical Logic"。这两本都有世界图书的引进版。你如果能搞定这两本,可以说在逻辑方面真正入了门,也就不用再浪费时间听我瞎侃了。:)

\textbf{胡侃学习(理论)计算机(7)}

发信人: sir (阿涩), 信区: Mathematics. 本篇人气: 862

标  题: 胡侃学习(理论)计算机(7)

发信站: 南京大学小百合站 (Mon Oct  8 04:05:20 2001), 站内信件

据说全中国最多只有三十个人懂图论(当年上课时陈道蓄老师转引张克民老师的话)。此言不虚。图论这东东,技巧性太强,几乎每题都有一个独特的方法,让人头痛。不过这也正是它魅力所在:只要你有创造性,它就能给你成就感。所以学图论没什么好说的,做题吧。

国内的图论书中,王树禾老师的“图论及其算法”非常成功。一方面,其内容在国内教材里算非常全面的。另一方面,其对算法的强调非常适合计算机系(本来就是科大计算机系教材)。有了这本书为主,再参考几本翻译的,如Bondy\&Murty的“图论及其应用”,邮电出版社翻译的“图论和电路网络”等等,就马马虎虎,对本科生足够了。

再进一步,世界图书引进有GTM系列的"Modern Graph Theory"。此书确实经典!国内好象还有一家出版了个翻译版。不过,学到这个层次,还是读原版好。搞定这本书,也标志着图论入了门,呵呵。

组合感觉没有太适合的国产书。还是读Graham和Knuth等人合著的经典“具体数学”吧,有翻译版,西电出的。


\textbf{胡侃学习(理论)计算机(8)}

发信人: sir (阿涩), 信区: Mathematics. 本篇人气: 847

标  题: 胡侃学习(理论)计算机(8)

发信站: 南京大学小百合站 (Mon Oct  8 04:05:52 2001), 站内信件

抽象代数,国内经典为莫宗坚先生的“代数学”。此书是北大数学系教材,深得好评。然而对本科生来说,此书未免太深。可以先学习一些其它的教材,然后再回头来看“代数学”。国际上的经典可就多了,GTM系列里就有一大堆。推荐一本谈不上经典,但却最简单的,最容易学的:http://www.math.miami.edu/~ec/book/ 这本“Introduction to Linear and Abstract Algebra"非常通俗易懂,而且把抽象代数和线性代数结合起来,对初学者来说非常理想。不过请注意版权问题,不要违反法律噢。

数论方面,国内有经典而且以困难著称的”初等数论“(潘氏兄弟著,北大版)。再追溯一点,还有更加经典(可以算世界级)并且更加困难的”数论导引“(华罗庚先生的名著,科学版,九章书店重印)。把基础的几章搞定一个大概,对本科生来讲足够了。但这只是初等数论。本科毕业后要学计算数论,你必须看英文的书,如Bach的"Introduction to Algorithmic Number Theory"。


\textbf{胡侃学习(理论)计算机(9)}


发信人: sir (阿涩), 信区: Mathematics. 本篇人气: 836

标  题: 胡侃学习(理论)计算机(9)\cite{sir9}

发信站: 南京大学小百合站 (Mon Oct  8 04:06:42 2001), 站内信件

理论计算机的根本,在于算法。现在系里给本科生开设算法设计与分析,确实非常正确。环顾西方世界,大约没有一个三流以上计算机系不把算法作为必修的。

算法教材目前公认以Corman等著的"Introduction to Algorithms"为最优。对入门而言,这一本已经足够,不需要再参考其它书。南大曾翻译出版此书,中文名为”现代计算机常用数据结构与算法“。pie好象提供了网上课程的link,我也就不用废话。


\textbf{胡侃学习(理论)计算机(10)}


发信人: sir (阿涩), 信区: Mathematics. 本篇人气: 948

标  题: 胡侃学习(理论)计算机(10)\cite{sir10}

发信站: 南京大学小百合站 (Mon Oct  8 04:07:10 2001), 站内信件


最后说说形式语言与自动机。我们用过北邮的教材,应该说写的还清楚。但是,有一点要强调:形式语言和自动机的作用主要在作为计算模型,而不是用来做编译。事实上,编译前端已经是死领域,没有任何open problem。如果为了这个,我们完全没必要去学形式语言--用用yacc什么的就完了。北邮的那本,在深度上,在跟可计算性的联系上都有较大的局限,现代感也不足。所以建议有兴趣的同学去读英文书......不过英文书中好的也不多,而且国内似乎没引进这方面的教材。

入门以后,把形式语言与自动机中定义的模型,和数理逻辑中用递归函数定义的模型比较一番,可以说非常有趣。现在才知道,什么叫”宫室之美,百官之富“!





\chapter{胡侃学习计算机--理论之外}

******************************************************************  

版权声明:本文作者sir系旅美学人、南京大学校友。 

为了学术或 教育的(非营利)目的,在保留本版权

声明的情况下,您可以自由 转载本文的电子版。

如果您要在传统媒体上转载此文,请与南京大学

小百合BBS站上的网友sir联系。 

****************************************************************** 


如果计算机只有理论,那么它不过是数学的一个分支,而不成为一门独立的科学。事实上,在理论之外,计算机科学还有更广阔的天空。我一直认为,4年根本不够学习计算机的基础知识,因为面太宽了...... 一个一流计算机系的优秀学生决不该仅仅是一个编程高手,但他一定首先是一个编程高手。 

我上大学的时候,第一门专业课时程序设计,现在好象改成了计算机科学导论?不管叫什么名字,总之,念计算机的人就是靠程序吃饭。 

去年在计算机系版有过一场争论,关于第一程序设计语言该用哪一种。我个人认为,用哪种语言属于末节,关键在养成良好的编程习惯。当年老师对我们说,打好基础后学一门新语言只要一个星期。现在我觉得根本不用一个星期--前提是先把基础打好。 

数据结构有两种不同的上法:一种把它当成降低要求的初级算法课,另一种把它当成高级的程序设计课。现在国内的课程好象介乎两者之间,而稍偏向前者。我个人认为,假如已经另有必修的算法课,恐怕后一个目的更重要些。 

国内流行的数据结构书也有两种:北大的红皮书(许卓群等著,高教版)和清华的绿皮书(严蔚敏等著,清华版)。两书差距不大。红皮书在理论上稍深一些,当然离严格的算法书还差好远。绿皮书更易接受些,而且佩有一本不错的习题集,但我觉得它让学生用伪代码写作业恐怕不见得太好。最好还是把算法都code以后debug一番,才能锻炼编程能力。 

汇编预言和微机原理是两门特烦人的课。你的数学/理论基础再好,也占不到什么便宜。这两门课之间的次序也好比先有鸡还是先有蛋,无论你先学哪门,都会牵扯另一门课里的东西。所以,只能静下来慢慢琢磨。这就是典型的工程课,不需要太多的聪明和顿悟,却需要水滴石穿的渐悟。 

有关这两门课的书,电脑书店里不难找到。弄几本最新的,对照着看吧。 

模拟电路这东东,如今不仅计算机系学生搞不定,电子系学生也多半害怕。如果你真想软硬件通吃,那么建议你先看看邱关源的“电路原理”,也许此后再看模拟电路底气会足些。 

教材:康华光的“电子技术基础”还是不错的。有兴趣也可以参考童诗白的书。 

数字电路比模拟电路要好懂得多。阎石的书也算一本好教材,遗憾的一点是集成电路讲少了些。真有兴趣,到东南无线电系去旁听他们的课。 

计算机系统结构该怎么教,国际上还在争论。国内能找到的较好教材为Stallings的"Computer Organization and Architecture:Designing for Performance"(清华影印本)。国际上最流行的则是“Computer architecture: a quantitative approach", by Patterson \& Hennessy。 

操作系统可以随便选用Tanenbaum的"Operating System Design and Implementation"和"Modern Operating  System" 两书之一。这两部都可以算经典,唯一缺点 就是理论上不够严格。不过这领域属于Hardcore System, 所以在理论上马虎一点也情有可原。 

如果先把形式语言学好了,则编译原理中的前端我看只要学四个算法:最容易实现的递归下降;最好的自顶向下算法LL(k);最好的自底向上算法LR(k);LR(1)的简化SLR(也许还有另一简化LALR?)。后端完全属于工程性质,自然又是another story。 


推荐教材: Aho等人的著名的Dragon Book: "Compilers: Principles, Techniques and Tools". 或者Appel的"Modern Compiler Implementation in C". 

学数据库的第一意义是告诉你,会用VFP编程不等于懂数据库。(这世界上自以为懂数据库的人太多了!)数据库设计既是科学又是艺术,数据库实现则是典型的工程。 

所以从某种意义上讲,数据库是最典型的一门计算机课--理工结合,互相渗透。 

推荐教材:Silberschatz, et al., "Database System Concepts". 
网络的标准教材还是来自Tanenbaum:”Computer Networks"(清华影印本)。不过,网络也属于Hardcore System,所以光看书是不够的。建议多读RFC,从IP的读起。等到能掌握10种左右常用协议,就没有几个人敢小看你了。 

必须结束这篇“胡侃”了,再侃下去非我力所能及。其实计算机还有很多基础课都值得一侃,如程序设计语言原理,图形图像处理,人工智能等等。怎奈我造诣有限,不敢再让内行耻笑。 

最后声明:前后的两篇“胡侃”只针对本科阶段的学习。即使把这些全弄通了,前面的路还长.....


\chapter{胡侃理论计算机}


声明\cite{kinglear}: 

1.本文集众前辈及恩师之经验于一文,由我执笔总结前辈所感而已。并非尽我所言,特别说明基于南京大学网友sir《胡侃理论计算机》一文并融入我的若干观点。 

2.本文虽经多次修订,仍有诸多不妥之处,有待笔者进一步学习之后修订此文,文章侧重理论学习兼谈实践,望读者各取所需。 

3. 本文早期版本曾流传于其它网站,本文会不断融入作者最新的学习感受,最终版本将只在此处保持最新更新,请读者注意此文修改中的若干重要思想变动。 


计算机科学与技术这一门科学深深的吸引着我们这些同学们,上计算机系已经有近三年了,自己也做了一些思考,原先不管是国内还是国外都喜欢把这个系分为计算机软件理论、计算机系统、计算机技术与应用。后来又合到一起,变成了现在的计算机科学与技术。我一直认为计算机科学与技术这门专业,在本科阶段是不可能切分成计算机科学和计算机技术的,因为计算机科学需要相当多的实践,而实践需要技术;每一个人(包括非计算机专业),掌握简单的计算机技术都很容易(包括原先Major们自以为得意的程序设计),但计算机专业的优势是:我们掌握许多其他专业并不"深究"的东西,例如,算法,体系结构,等等。非计算机专业的人可以很容易地做一个芯片,写一段程序,但他们做不出计算机专业能够做出来的大型系统。今天我想专门谈一谈计算机科学,并将重点放在计算理论上。 

在我大一时无意中找到了南京大学网友sir的帖子"胡侃(理论)计算机学习",这个帖子对我大学学习起到了至关重要的指导作用,我在这篇文章成文的时候正是基于sir的文章做得必要的补充和修改,并得到了sir的支持。再有就是每次和本系司徒彦南兄的交谈,都能从中学到很多东西,在这份材料中也有很多体现。这份材料是我原来给学弟学妹们入学教育的讲稿之一,原有基础上改进了其中我认为不太合适的理论,修正了一些观点,在推荐教材方面结合我的学习情况有了较大改变。值得一提的是增加了一些计算机理论的内容,计算机技术的内容结合我国的教学情况和我们学习的实际情况进行了重写。这里所作的工作也只是将各位学长和同学们的学习体会以及我在学习计算机科学时的所思所想汇总在一起写了下来,很不成熟。目的就是希望能够给一些刚入学或者是学习计算机科学还没有入门的同学以一些建议。不期能够起到多大的作用,但求能为同学们的学习计算机科学与技术带来微薄的帮助。还是那句话,计算机科学博大精深,我只是个初学者,不当之处希望大家批评指正。 

\section{计算机理论的一个核心问题--从数学谈起}

  [1]高等数学Vs数学分析 

  记得当年大一入学,每周四课时高等数学,天天作业不断(那时是七天工作制)。颇有些同学惊呼走错了门:咱们这到底念的是什么系?不错,你没走错门,这就是计算机科学与技术系。我国计算机科学系里的传统是培养做学术研究,尤其是理论研究的人(方向不见得有多大的问题,但是做得不是那么尽如人意)。而计算机的理论研究,说到底了,如网络安全学,图形图像学,视频音频处理,哪个方向都与数学有着很大的关系,虽然也许是正统数学家眼里非主流的数学。这里我还想阐明我的一个观点:我们都知道,数学是从实际生活当中抽象出来的理论,人们之所以要将实际抽象成理论,目的就在于想用抽象出来的理论去更好的指导实践,有些数学研究工作者喜欢用一些现存的理论知识去推导若干条推论,殊不知其一:问题考虑不全很可能是个错误的推论,其二:他的推论在现实生活中找不到原型,不能指导实践。严格的说,我并不是一个理想主义者,政治课上学的理论联系实际一直是指导我学习科学文化知识的航标 (至少我认为搞计算机科学与技术的应当本着这个方向)。 

  其实我们计算机系学数学仅学习高等数学是不够的 (典型的工科院校一般都开的是高等数学),我们应该像数学系一样学一下数学分析(清华计算机系开的好像就是数学分析,我们学校计算机学院开的也是,不过老师讲起来好像还是按照高等数学讲),数学分析这门科学,咱们学计算机的人对它有很复杂的感情。在于它是偏向于证明型的数学课程,这对我们培养良好的分析能力和推理能力极有帮助。我的软件工程学导师北工大数理学院的王仪华先生就曾经教导过我们,数学系的学生到软件企业中大多作软件设计与分析工作,而计算机系的学生做程序员的居多,原因就在于数学系的学生分析推理能力,从所受训练的角度上要远远在我们平均水平之上。当年出现的怪现象是:计算机系学生的高中数学基础在全校数一数二(希望没有冒犯其它系的同学),教学课时数也仅次于数学系,但学完之后的效果却不尽如人意。难道都是学生不努力吗,我看未见得,方向错了也说不一定,其中原因何在,发人深思。 

  我个人的浅见是:计算机系的学生,对数学的要求固然跟数学系不同,跟物理类差别则更大。通常非数学专业的所谓“高等数学”,无非是把数学分析中较困难的理论部分删去,强调套用公式计算而已。而对计算机系来说,数学分析里用处最大的恰恰是被删去的理论部分。说得难听一点,对计算机系学生而言,追求算来算去的所谓"工程数学"已经彻底地走进了误区。记上一堆曲面积分的公式,难道就能算懂了数学?那倒不如现用现查,何必费事记呢?再不然直接用Mathematica或是Matlab好了。退一万步讲,即使是学高等数学我想大家看看华罗庚先生的《高等数学导论》也是比一般的教材好得多。华罗庚在数学上的造诣不用我去多说,但是他这光辉的一生做得我认为对我们来说,最重要的几件事情:首先是它筹建了中国科学院计算技术研究所,这是我们国家计算机科学的摇篮。在有就是他把很多的高等数学理论都交给了做工业生产的技术人员,推动了中国工业的进步。第三件就是他一生写过很多书,但是对高校师生价值更大的就是他在病期间在病床上和他的爱徒王元写了《高等数学引论》(王元与其说是他的爱徒不如说是他的同事,是中科院数学所的老一辈研究员,对歌德巴赫猜想的贡献全世界仅次于陈景润)这书在我们的图书馆里居然 
找得到,说实话,当时那个书上已经长了虫子,别人走到那里都会闪开,但我却格外感兴趣,上下两册看了个遍,我的最大收获并不在于理论的阐述,而是在于他的理论完全的实例化,在生活中去找模型。这也是我为什么比较喜欢具体数学的原因,正如我在上文中提到的,理论脱离了实践就失去了它存在的意义。正因为理论是从实践当中抽象出来的,所以理论的研究才能够更好的指导实践,不用于指导实践的理论可以说是毫无价值的。 

  我在系里最爱做的事情就是给学弟学妹们推荐参考书。没有别的想法,只是希望他们少走弯路。中文的数学分析书,一般都认为以北大张筑生老师的"数学分析新讲"为最好。张筑生先生一生写的书并不太多,但是只要是写出来的每一本都是本领域内的杰作,这本当然更显突出些。这种老书看起来不仅是在传授你知识,而是在让你体会科学的方法与对事物的认识方法。万一你的数学实在太好,那就去看菲赫金哥尔茨?quot;微积分学教程"好了--但我认为没什么必要,毕竟你不想转到数学系去。吉米多维奇的"数学分析习题集"也基本上是计算型的书籍。书的名气很大,倒不见得适合我们,还是那句话,重要的是数学思想的建立,生活在信息社会里我们求的是高效,计算这玩意还是留给计算机吧。不过现在多用的似乎是复旦大学的《数学分析》,高等教育出版社的,也是很好的教材。 

  中国的所谓高等代数,就等于线性代数加上一点多项式理论。我以为这有好的一面,因为可以让学生较早感觉到代数是一种结构,而非一堆矩阵翻来覆去。这里不得不提南京大学林成森,盛松柏两位老师编的"高等代数",感觉相当舒服。此书相当全面地包含了关于多项式和线性代数的基本初等结果,同时还提供了一些有用的又比较深刻的内容,如Sturm序列,Shermon-Morrison公式,广义逆矩阵等等。可以说,作为本科生如能吃透此书,就可以算是高手。国内较好的高等代数教材还有清华计算机系用的那本,清华出版社出版,书店里多多,一看就知道。从抽象代数的观点来看,高等代数里的结果不过是代数系统性质的一些例子而已。莫宗坚先生的《代数学》里,对此进行了深刻的讨论。然而莫先生的书实在深得很,作为本科生恐怕难以接受,不妨等到自己以后成熟了一些再读。 

 正如上面所论述的,计算机系的学生学习高等数学:知其然更要知其所以然。你学习的目的应该是:将抽象的理论再应用于实践,不但要掌握题目的解题方法,更要掌握解题思想,对于定理的学习:不是简单的应用,而是掌握证明过程即掌握定理的由来,训练自己的推理能力。只有这样才达到了学习这门科学的目的,同时也缩小了我们与数学系的同学之间思维上的差距。 

 [2]计算数学基础 

概率论与数理统计这门课很重要,可惜大多数院校讲授这门课都会少些东西。少了的东西现在看至少有随机过程。到毕业还没有听说过Markov过程,此乃计算机系学生的耻辱。没有随机过程,你怎么分析网络和分布式系统?怎么设计随机化算法和协议?据说清华计算机系开有"随机数学",早就是必修课。另外,离散概率论对计算机系学生来说有特殊的重要性。而我们国家工程数学讲的都是连续概率。现在,美国已经有些学校开设了单纯的"离散概率论"课程,干脆把连续概率删去,把离散概率讲深些。我们不一定要这么做,但应该更加强调离散概率是没有疑问的。这个工作我看还是尽早的做为好。 

  计算方法学(有些学校也称为数学分析学)是最后一门由数理学院给我们开的课。一般学生对这门课的重视程度有限,以为没什么用。不就是照套公式嘛!其实,做图形图像可离不开它,密码学搞深了也离不开它。而且,在很多科学工程中的应用计算,都以数值的为主。这门课有两个极端的讲法:一个是古典的"数值分析",完全讲数学原理和算法;另一个是现在日趋流行的"科学与工程计算",干脆教学生用软件包编程。我个人认为,计算机系的学生一定要认识清楚我们计算机系的学生为什么要学这门课,我是很偏向于学好理论后用计算机实现的,最好使用C语言或C++编程实现。向这个方向努力的书籍还是挺多的,这里推荐大家高等教育出版社(CHEP)和施普林格出版社(Springer)联合出版的《计算方法(Computational Methods)》,华中理工大学数学系写的 (现华中科技大学),这方面华科大做的工作在国内应算是比较多的,而个人认为以这本最好,至少程序设计方面涉及了:任意数学函数的求值,方程求根,线性方程组求解,插值方法,数值积分,场微分方程数值求解。李庆扬先生的那本则理论性过强,与实际应用结合得不太紧,可能比较适合纯搞理论的。 

 [3]也谈离散数学   

每个学校本系里都会开一门离散数学,涉及集合论,图论,和抽象代数,数理逻辑。不过,这么多内容挤在离散数学一门课里,是否时间太紧了点?另外,计算机系学生不懂组合和数论,也是巨大的缺陷。要做理论,不懂组合或者数论吃亏可就太大了。从理想的状态来看,最好分开六门课:集合,逻辑,图论,组合,代数,数论。这个当然不现实,因为没那么多课时。也许将来可以开三门课:集合与逻辑,图论与组合,代数与数论。(这方面我们学校已经着手开始做了)不管课怎么开,学生总一样要学。下面分别谈谈上面的三组内容。 

古典集合论,北师大出过一本《基础集合论》不错。 

数理逻辑,中科院软件所陆钟万教授的《面向计算机科学的数理逻辑》就不错。现在可以找到陆钟万教授的讲课录像,http://www.cas.ac.cn/html/Dir/2001/11/06/3391.htm自己去看看吧。总的来说,学集合/逻辑起手不难,普通高中生都能看懂。但越往后越感觉深不可测。学完以上各书之后,如果你还有精力兴趣进一步深究,那么可以试一下GTM系列中的《Introduction to Axiomatic Set Theory》和《A Course of Mathematical Logic》。这两本都有世界图书出版社的引进版。你如果能搞定这两本,可以说在逻辑方面真正入了门,也就不用再浪费时间听我瞎侃了。 

  据说全中国最多只有三十个人懂图论。此言不虚。图论这门科学,技巧性太强,几乎每个问题都有一个独特的方法,让人头痛。不过这也正是它魅力所在:只要你有创造性,它就能给你成就感。我的导师说,图论里面随便找一块东西就可以写篇论文。大家可以体会里面内容之深广了吧!国内的图论书中,王树禾老师的"图论及其算法"非常成功(顺便推荐大家王先生的"数学思想史",个人认为了解科学史会对我们的学习和研究起到很大的推动作用)。一方面,其内容在国内教材里算非常全面的。另一方面,其对算法的强调非常适合计算机系(本来就是科大计算机系教材)。有了这本书为主,再参考几本翻译的,如Bondy \& Murty的《图论及其应用》,人民邮电出版社翻译的《图论和电路网络》等等,就马马虎虎,对本科生绝对足够了。再进一步,世界图书引进有GTM系列的"Modern Graph Theory"。此书确实经典!国内好象还有一家出版了个翻译版。不过,学到这个层次,还是读原版好(说实话,主要是亲身体验翻译版的弊端,这个大家自己体会)。搞定这本书,也标志着图论入了门。 

  离散数学方面我们北京工业大学有个世界级的专家,叫邵学才,复旦大学概率论毕业的,教过高等数学,线性代数,概率论,最后转向离散数学,出版著作无数,论文集新加坡有一本,堪称经典,大家想学离散数学的真谛不妨找来看看。这老师的课我专门去听过,极为经典。不过你要从他的不经意的话中去挖掘精髓。在同他的交谈当中我又深刻地发现一个问题,虽说邵先生写书无数,但依他自己的说法每本都差不多,我实在觉得诧异,他说主要是有大纲的限制,不便多写。这就难怪了,很少听说国外写书还要依据个什么大纲(就算有,内容也宽泛的多),不敢越雷池半步,这样不是看谁的都一样了。外版的书好就好在这里,最新的科技成果里面都有论述,别的先不说,至少?quot;紧跟时代的理论知识"。 

  原先离散数学和数据结构归在一起成为离散数学结构,后来由于数据结构的内容比较多,分出来了,不过最近国外好像有些大学又把它们合并到了一起,道理当然不用说,可能还是考虑到交叉的部分比较多。比较经典的书我看过得应算是《Discrete Mathematical Structures》了,清华大学出版社有个影印版的。 

[4]续谈其他的一些计算数学 

组合数学我看的第一本好像是北大捐给我们学院的,一本外版书。感觉没有太适合的国产书。还是读Graham和Knuth等人合著的经典"具体数学"吧,西安电子科技大学出版社有翻译版。 

  《组合数学》,《空间解析几何》还有那本《拓扑学》,看这三本书的时候是极其费事的,原因有几点,首先是这三本书无一例外,都是用繁体字写的,第二就是书真得实在是太脏了,我在图书馆的座位上看,同学们都离我做得很远。我十分不自然,不愿意影响同学,但是学校不让向外借这种书(呵呵,说起这是也挺有意思,别人都不看这种书,只有我在看,老师就特别的关注我,后来我和他讲了这些书的价值,他居然把他们当作是震馆之宝,老师都不许借,不过后来他们看我真得很喜欢看,就把书借给了我,当然用的是馆长的名义借出去的。)不过收获是非常大的,再后来学习计算机理论时里面的很多东西都是常会用到的。当然如果你没看过这些书绝对理解不到那个层次。拿拓扑学来说,我们学校似乎是美开设这门课程,但是这门课程的重要性是显而易见的,没有想到的是在那本书的很多页中都夹着一些读书笔记,而那个笔记的作者及有些造诣,有些想法可以用到现代网络设计当中。 

  抽象代数,国内经典为莫宗坚先生的《代数学》。此书听说是北大数学系教材,深得好评。然而对本科生来说,此书未免太深。可以先学习一些其它的教材,然后再回头来看"代数学"。国际上的经典可就多了,GTM系列里就有一大堆。推荐一本谈不上经典,但却最简单的,最容易学的:http://www.math.miami.edu/~ec/book/这本"Introduction to Linear and Abstract Algebra"非常通俗易懂,而且把抽象代数和线性代数结合起来,对初学者来说非常理想,我校比较牛的同学都有收藏。 

  数论方面,国内有经典而且以理论性极强著称的潘氏兄弟著作。再追溯一点,还有更加经典(可以算世界级)并且更加困难的"数论导引"(华罗庚先生的名著,科学版,九章书店重印,繁体的看起来可能比较困难)。把基础的几章搞定一个大概,对本科生来讲足够了。但这只是初等数论。本科毕业后要学计算数论,你必须看英文的书,如Bach的"Introduction to Algorithmic Number Theory"。 

 计算机科学理论的根本,在于算法。现在很多系里给本科生开设算法设计与分析,确实非常正确。环顾西方世界,大约没有一个三流以上计算机系不把算法作为必修的。算法教材目前公认以Corman等著的《Introduction to Algorithms》为最优。对入门而言,这一本已经足够,不需要再参考其它书。 深一点的就是大家作为常识都知道的TAOCP了。即是《The Art of Computer Programming》3册内容全世界都能看下来的本身就不多,Gates曾经说过"若是你能把这书上面的东西都看懂,请把你的简历发给我一份"我的学长司徒彦南兄就曾千里迢迢从美国托人买这书回来,别的先不说,可见这书的在我们计算机科学与技术系中的分量。 

 再说说形式语言与自动机。我看过北邮的教材,应该说写的还清楚。有一本通俗易懂的好书,MIT的sipser的 《introduction to theory of computation》。但是,有一点要强调:形式语言和自动机的作用主要在作为计算模型,而不是用来做编译。事实上,编译前端已经是死领域,没有任何open problems,北科大的班晓娟博士也曾经说过,编译的技术已相当成熟。如果为了这个,我们完全没必要去学形式语言--用用yacc什么的就完了。北邮的那本在国内还算比较好,但是在深度上,在跟可计算性的联系上都有较大的局限,现代感也不足。所以建议有兴趣的同学去读英文书,不过国内似乎没引进这方面的教材。可以去互动出版网上看一看。入门以后,把形式语言与自动机中定义的模型,和数理逻辑中用递归函数定义的模型比较一番,可以说非常有趣。现在才知道,什么叫"宫室之美,百官之富"! 

  计算机科学和数学的关系有点奇怪。二三十年以前,计算机科学基本上还是数学的一个分支。而现在,计算机科学拥有广泛的研究领域和众多的研究人员,在很多方面反过来推动数学发展,从某种意义上可以说是孩子长得比妈妈还高了。但不管怎么样,这个孩子身上始终流着母亲的血液。这血液是the mathematical underpinning of computer science(计算机科学的数学基础),也就是理论计算机科学。原来在东方大学城图书馆中曾经看过一本七十年代的译本(书皮都没了,可我就爱关注这种书),大概就叫《计算机数学》。那本书若是放在当时来讲决是一本好书,但现在看来,涵盖的范围还算广,深度则差了许多,不过推荐大一的学生倒可以看一看,至少可以使你的计算数学入入门,也就是说至少可以搞清数学到底在计算机科学什么地方使用。 

  最常和理论计算机科学放在一起的一个词是什么?答:离散数学。这两者的关系是如此密切,以至于它们在不少场合下成为同义词。(这一点在前面的那本书中也有体现)传统上,数学是以分析为中心的。数学系的同学要学习三四个学期的数学分析,然后是复变函数,实变函数,泛函数等等。实变和泛函被很多人认为是现代数学的入门。在物理,化学,工程上应用的,也以分析为主。 

  随着计算机科学的出现,一些以前不太受到重视的数学分支突然重要起来。人们发现,这些分支处理的数学对象与传统的分析有明显的区别:分析研究的问题解决方案是连续的,因而微分,积分成为基本的运算;而这些分支研究的对象是离散的,因而很少有机会进行此类的计算。人们从而称这些分支为"离散数学"。"离散数学"的名字越来越响亮,最后导致以分析为中心的传统数学分支被相对称为"连续数学"。 

  离散数学经过几十年发展,基本上稳定下来。一般认为,离散数学包含以下学科: 

1) 集合论,数理逻辑与元数学。这是整个数学的基础,也是计算机科学的基础。 

2) 图论,算法图论;组合数学,组合算法。计算机科学,尤其是理论计算机科学的核心是算法,而大量的算法建立在图和组合的基础上。 

3) 抽象代数。代数是无所不在的,本来在数学中就非常重要。在计算机科学中,人们惊讶地发现代数竟然有如此之多的应用。 

但是,理论计算机科学仅仅就是在数学的上面加上"离散"的帽子这么简单吗?一直到大约十几年前,终于有一位大师告诉我们:不是。D.E.Knuth(他有多伟大,我想不用我再说了)在Stanford开设了一门全新的课程Concrete Mathematics。 Concrete这个词在这里有两层含义: 

首先:对abstract而言。Knuth认为,传统数学研究的对象过于抽象,导致对具体的问题关心不够。他抱怨说,在研究中他需要的数学往往并不存在,所以他只能自己去创造一些数学。为了直接面向应用的需要,他要提倡"具体"的数学。在这里我做一点简单的解释。例如在集合论中,数学家关心的都是最根本的问题--公理系统的各种性质之类。而一些具体集合的性质,各种常见集合,关系,映射都是什么样的,数学家觉得并不重要。然而,在计算机科学中应用的,恰恰就是这些具体的东西。Knuth能够首先看到这一点,不愧为当世计算机第一人。其次,Concrete是Continuous(连续)加上discrete(离散)。不管连续数学还是离散数学,只要是能与我们研究的内容挂上钩的都是有用的数学! 

2、理论与实际的结合--计算机科学技术研究的范畴与学习方法 

前面主要是从数学角度来看的。从计算机角度来看,理论计算机科学目前主要的研究领域包括:可计算性理论,算法设计与复杂性分析,密码学与信息安全,分布式计算理论,并行计算理论,网络理论,生物信息计算,计算几何学,程序语言理论等等。这些领域互相交叉,而且新的课题在不断提出,所以很难理出一个头绪来。想搞搞这方面的工作,推荐看中国计算机学会的一系列书籍,至少代表了我国的权威。下面随便举一些例子。 

  由于应用需求的推动,密码学现在成为研究的热点。密码学建立在数论(尤其是计算数论),代数,信息论,概率论和随机过程的基础上,有时也用到图论和组合学等。很多人以为密码学就是加密解密,而加密就是用一个函数把数据打乱。这样的理解太浅显了。

现代密码学至少包含以下层次的内容: 

第一,密码学的基础。例如,分解一个大数真的很困难吗?能否有一般的工具证明协议正确? 

第二,密码学的基本课题。例如,比以前更好的单向函数,签名协议等。 

第三,密码学的高级问题。例如,零知识证明的长度,秘密分享的方法。 

第四,密码学的新应用。例如,数字现金,叛徒追踪等。 

密码学方面值得推荐的有一本《应用密码学》还有就是平时多看看年会的论文集,感觉这种材料实用性比较强,会提高很快。 

在分布式系统中,也有很多重要的理论问题。例如,进程之间的同步,互斥协议。一个经典的结果是:在通信信道不可靠时,没有确定型算法能实现进程间协同。所以,改进TCP三次握手几乎没有意义。例如时序问题。常用的一种序是因果序,但因果序直到不久前才有一个理论上的结果....例如,死锁没有实用的方法能完美地对付。例如,......操作系统研究过就自己去举吧! 

  如果计算机只有理论,那么它不过是数学的一个分支,而不成为一门独立的科学。事实上,在理论之外,计算机科学还有更广阔的天空。 
我一直认为,4年根本不够学习计算机的基础知识,因为面太宽了,要是真学的话,我想至少8年的学习能使你具有一定的科学素养...... 
  这方面我想先说说我们系在各校普遍开设的《计算机基础》。在高等学校开设《计算机基础课程》是我国高教司明文规定的各专业必修课程要求。主要内容是使学生初步掌握计算机的发展历史,学会简单的使用操作系统,文字处理,表格处理功能和初步的网络应用功能。但是在计算机科学系教授此门课程的目标决不能与此一致。在计算机系课程中目标应是:让学生较为全面的了解计算机学科的发展,清晰的把握计算机学科研究的方向,发展的前沿即每一个课程在整个学科体系中所处的地位。搞清各学科的学习目的,学习内容,应用领域。使学生在学科学习初期就对整个学科有一个整体的认识,以做到在今后的学习中清楚要学什么,怎么学。计算机基本应用技能的位置应当放在第二位或更靠后,因为这一点对于本系的学生应当有这个摸索能力。这一点很重要。推荐给大家一本书:机械工业出版社的《计算机文化》(New Perspective of Computer Science),看了这本书我才深刻的体会到自己还是个计算机科学初学者,才比较透彻的了解了什么是计算机科学。科学出版社的《计算科学导论》 (赵致琢先生的著作)可以说是在高校计算机教育改革上作了很多的尝试,也是这方面我受益很大的一本书。 

一个一流计算机系的优秀学生决不该仅仅是一个编程高手,但他一定首先是一个编程高手。我上大学的时候,第一门专业课是C语言程序设计,念计算机的人从某种角度讲相当一部分人是靠写程序吃饭的。在我们北京工业大学计算机系里一直有这样的争论(时至今日CSDN上也有),关于第一程序设计语言该用哪一种。我个人认为,用哪种语言属于末节,关键在养成良好的编程习惯。当年老师对我们说,打好基础后学一门新语言只要一个星期。现在我觉得根本不用一个星期,前提是先把基础打好。不要再犹豫了,学了再说,等你抉择好了,别人已经会了几门语言了。 

[1]专谈计算机系统的学习 

汇编语言和微机原理是两门特烦人的课。你的数学/理论基础再好,也占不到什么便宜。这两门课之间的次序也好比先有鸡还是先有蛋,无论你先学哪门,都会牵扯另一门课里的东西。所以,只能静下来慢慢琢磨。这就是典型的工程课,不需要太多的聪明和顿悟,却需要水滴石穿的渐悟。有关这两门课的书,计算机书店里不难找到。弄几本最新的,对照着看吧。组成原理推荐《计算机组成与结构》清华大学王爱英教授写的。汇编语言大家拿8086/8088入个门,之后一定要学80x86汇编语言。实用价值大,不落后,结构又好,写写高效病毒,高级语言里嵌一点汇编,进行底层开发,总也离不开他,推荐清华大学沈美明的《IBM-PC汇编语言程序设计》。有些人说不想了解计算机体系结构,也不想制造计算机,所以诸如计算机原理,汇编语言,接口之类的课觉得没必要学,这样合理吗?显然不合理,这些东西迟早得掌握,肯定得接触,而且,这是计算机专业与其他专业学生相比的少有的几项优势。做项目的时候,了解这些是非常重要的,不可能说,仅仅为了技术而技术,只懂技术的人最多做一个编码工人,而永远不可能全面地了解整个系统的设计,而编码工人是越老越不值钱。关于组成原理还有个讲授的问题,在我学这门课程时老师讲授时把CPU工作原 

模拟电路这个学科,如今不仅计算机系学生搞不定,电子系学生也多半害怕。如果你真想软硬件通吃,那么建议你先看看邱关源的"电路原理",也许此后再看模拟电路底气会足些。教材:康华光的"电子技术基础"(高等教育出版社)还是不错的(我校电子系在用)。有兴趣也可以参考童诗白的书。 

数字电路比模拟电路要好懂得多。推荐大家看一看北京工业大学刘英娴教授写的《数字逻辑》。业绩人士都说这本书很有参考价值 (机械工业出版社)。原因很明了,实用价值高,能听听她讲授的课程更是有一种"享受科学"的感觉。清华大学阎石的书也算一本好教材,遗憾的一点是集成电路讲少了些。真有兴趣,看一看大规模数字系统设计吧(北航那本用的还比较多)。 

计算机系统结构该怎么教,国际上还在争论。国内能找到的较好教材为Stallings的《Computer Organization and Architectureesigning for Performance》(清华影印本)。国际上最流行的则是《Computer architecture: a quantitative approach》, by Patterson \& Hennessy。 

[2]一些其他的专业课程 

操作系统可以选用《操作系统的内核设计与实现》和《现代操作系统》两书之一。这两部都可以算经典。我们当时理论方面学习采用的是清华大学出版社《操作系统》,张尧学教授写的那本。可以说理论涉及的比较全,在有就是他的实验指导书,操作系统这门学科同程序设计结合得很紧密,不自己试着做些什么恐怕很难搞通。我想作为实践类的参考首推的是这本:《4.4BSD操作系统设计与实现》作为开源文化很重要的一个分支的BSD操作系统家族做得非常出色,其中现在若干出色的分支系统(例如FreeBSD,NetBSD,OpenBSD,DragonflyBSD)都与4.4BSD有着难解的渊源。而4.4BSD的开发者亲自撰写的这本理论设计与实现便是一本绝佳的参考。另外在有一些辅助材料的基础上研究*nix的源代码也是深入操作系统设计与实现的一条绝佳之路。(感谢CSDN网友ffgg的建议,我将《Windows操作系统原理》这本书去掉,现在看来这本书的确不能算是一个十分优秀的作品) 

如果先把形式语言学好了,则编译原理中的前端我看只要学四个算法:最容易实现的递归下降;最好的自顶向下算法LL(k);最好的自底向上算法LR(k);LR(1)的简化SLR(也许还有另一简化LALR)。后端完全属于工程性质,自然又是another story。 推荐教材:Kenneth C.Louden写的《Compiler Construction Principles and Practice》即是《编译原理及实践》(机械工业出版社的译本) 

学数据库要提醒大家的是,会用VFP,VB, Power builder不等于懂数据库。(这世界上自以为懂数据库的人太多了!)数据库设计既是科学又是艺术,数据库实现则是典型的工程。所以从某种意义上讲,数据库是最典型的一门计算机课程--理工结合,互相渗透。另外推荐大家学完软件工程学后再翻过来看看数据库技术,又会是一番新感觉。至少对一些基本概念与描述方法会有很深的体会,比如说数据字典,E-R图之类的。推荐教材:Abraham Silberschatz等著的 "Database System Concepts".作为知识的完整性,还推荐大家看一看机械工业出版社的《数据仓库》译本。 

计算机网络的标准教材还是来自Tanenbaum的《Computer Networks》(清华大学有译本)。还有就是推荐谢希仁的《计算机网络教程》(人民邮电出版社)问题讲得比较清楚,参考文献也比较权威。不过,网络也属于Hardcore System,所以光看书是不够的。建议多读RFC,http://www.ietf.org/rfc.html里可以按编号下载RFC文档。从IP的读起。等到能掌握10种左右常用协议,就没有几个人敢小看你了。再做的工作我看放在网络设计上就比较好了。 

数据结构的重要性就不言而喻了,学完数据结构你会对你的编程思想进行一番革命性的洗礼,会对如何建立一个合理高效的算法有一个清楚的认识。对于算法的建立我想大家应当注意以下几点: 

当遇到一个算法问题时,首先要知道自己以前有没有处理过这种问题.如果见过,那么你一般会顺利地做出来;如果没见过,那么考虑以下问题: 

1. 问题是否是建立在某种已知的熟悉的数据结构(例如,二叉树)上?如果不是,则要自己设计数据结构。 

2. 问题所要求编写的算法属于以下哪种类型?(建立数据结构,修改数据结构,遍历,查找,排序...) 

3. 分析问题所要求编写的算法的数学性质.是否具备递归特征?(对于递归程序设计,只要设计出合理的参数表以及递归结束的条件,则基本上大功告成.) 

4. 继续分析问题的数学本质.根据你以前的编程经验,设想一种可能是可行的解决办法,并证明这种解决办法的正确性.如果题目对算法有时空方面的要求,证明你的设想满足其要求.一般的,时间效率和空间效率难以兼得.有时必须通过建立辅助存储的方法来节省时间. 

5. 通过一段时间的分析,你对解决这个问题已经有了自己的一些思路.或者说,你已经可以用自然语言把你的算法简单描述出来.继续验证其正确性,努力发现其中的错误并找出解决办法.在必要的时候(发现了无法解决的矛盾),推翻自己的思路,从头开始构思. 

6. 确认你的思路可行以后,开始编写程序.在编写代码的过程中,尽可能把各种问题考虑得详细,周密.程序应该具有良好的结构,并且在关键的地方配有注释. 

7. 举一个例子,然后在纸上用笔执行你的程序,进一步验证其正确性.当遇到与你的设想不符的情况时,分析问题产生的原因是编程方面的问题还是算法思想本身有问题. 

8. 如果程序通过了上述正确性验证,那么在将其进一步优化或简化。 

9. 撰写思路分析,注释. 

对于具体的算法思路,只能靠你自己通过自己的知识和经验来加以获得,没有什么特定的规律(否则程序员全部可以下岗了,用机器自动生成代码就可以了).要有丰富的想象力,就是说当一条路走不通时,不要钻牛角尖,要敢于推翻自己的想法.我也只不过是初学者,说出上面的一些经验,仅供大家参考和讨论。 

关于人工智能,我觉得的也是非常值得大家仔细研究的,虽然不能算是刚刚兴起的学科了,但是绝对是非常有发展前途的一门学科。我国人工智能创始人之一,北京科技大学涂序彦教授(这老先生是我的导师李小坚博士的导师)对人工智能这样定义:人工智能是模仿、延伸和扩展人与自然的智能的技术科学。在美国人工智能官方教育网站上对人工智能作了如下定义:Artificial Intelligence, or AI for short, is a combination of computer science, physiology, and philosophy. AI is a broad topic, consisting of different fields, from machine vision to expert systems. The element that the fields of AI have in common is the creation of machines that can "think". 

这门学科研究的问题大概说有: 

(1)符号主义: 符号计算与程序设计基础,知识表达方法 :知识与思维,产生式规则,语义网络,一阶谓词逻辑问题求解方法:搜索策略,启发式搜寻,搜寻算法,问题规约方法,谓词演算:归结原理,归结过程专家系统:建立专家系统的方法及工具 

(2)联接主义(神经网络学派):1988年美国权威机构指出:数据库,网络发展呈直线上升,神经网络可能是解决人工智能的唯一途径。关于神经网络学派,现在很多还是在发展阶段。 

我想对于人工智能的学习,大家一定不要像学数学似的及一些现成的结论,要学会分析问题,最好能利用程序设计实现,这里推荐给大家ACM最佳博士论文奖获得者涂晓媛博士的著作《人工鱼-计算机动画的人工生命方法》(清华大学出版社)。搞人工生命的同学不会不知道国际知名的涂氏父女吧。关于人工智能的书当然首选《Artificial Intelligence A New Synthesis》Nils J.Nilsson.鼻祖嘛! 

关于网络安全我也想在这里说两句,随着计算机技术的发展,整个社会的信息化水平突飞猛进,计算机网络技术日新月异,网络成了当即社会各个工作领域不可缺少的组成部分,只要有网络存在,网络安全问题就是一个必须解决好的问题,学习网络安全不是简简单单的收集一些黑客工具黑一黑别人的网站,而是要学习他的数学原理,实现原理,搞清底层工作机制,这样才能解决大部分的现有问题和新出现的安全问题。总的来说信息安全学的研究还是非常深奥的,这方面体会比较深的要算是在最近的微软杯程序设计大赛中利用.NET平台开发的那个项目My E-business Fairy.NET过程中了。 

[3]闲聊软件工程 

关于计算机科学的一些边缘科学我想谈一谈软件工程技术,对于一个企业,推出软件是不是就是几个程序员坐在一起,你写一段程序,我写一段程序呢?显然不是。软件工程是典型的计算机科学和数学,管理科学,心理学,社会学等学科的综合。它使我们这些搞理论和技术的人进入了一个社会。你所要考虑的不仅仅是程序的优劣,更应该考虑程序与软件的区别,软件与软件产品的区别,软件软件产品的市场前景,如何去更好的与人交流。这方面我还在学习阶段,以后这方面再写文章吧,先推荐给大家几本书:畅销20年不衰的《人月神话》(清华大学中文版,中国电力出版社影印版),《软件工程-实践者研究的方法》(机械工业出版社译本),《人件》(据说每一位微软公司的部门经理都读过这本书,推荐老总们和想当老总的同学都看看,了解一下什么是软件企业中的人)以及微软公司的《软件开发的科学与艺术》和《软件企业的管理与文化》(研究软件企业的制胜之道当然要研究微软的成功经验了!) 看完上面的书,结合自己做的一些团队项目,我的一些比较深的体会有这么几点 

1.How important a plan is for a project development. 

2.How to communicate with your team members in a more effective way. 

3.How to solve unexpected situations. 

4.The importance of unification. 

5.The importance of doing what you should do. 

6.The importance of designing before programming. 

7.The importance of management. 

8.The importance of thinking what your teammates think. 

在软件开发过程中我们应当具有以下能力: 

1.Like it if you would like to do it. 

We believe that your attitude toward your work will definitely makes great effect on the project. 

2.The spirit of group working. 

Take myself as an example. I am just a part of the team, just a little part. You must make it clear that you are just a member of the team, but your effort will change your project a lot. 

3.Passion 

With passion, you can do your job in a more effective way. 

4.The ability of solving unexpected problems. 

5.Learning New things in a very short time 

It is the basic requirement for we computer major to learn new technology. 

6.Creativity 

The tools are changing. As for us, what's more important is to use these new tools and technology to enable people and businesses throughout the world to realize their full potential. 

7.The ability to do your work independently. 

Every member has his own business. In a team, your work cannot be replaced by others' so you must do your business well in order to assure the project development process. 

\bibliographystyle{plainnat}
\bibliography{gk}
\clearpage
