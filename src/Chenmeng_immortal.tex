\chapter{记录今天就是记录历史}


\begin{center} 陈~虻\cite{chengmeng_immortal} \end{center}

出处:网络

一个著名的学者曾经说过:历史都是当代史。

从这个意义上来说,记录过去就是记录今天,而今天又是明天的历史,记录今天也就是在记录历史。若干年之后,当我们的后人向我们问起今天的历史时,我们给他们展示的应该是那些有价值、永远不会沉沦的东西,而这些也恰恰是纪录片工作者所应该具有的历史使命感。

那么,对于电视纪录片工作者来说,我们应该怎样记录今天的历史,应该带着一种什么样的心态去记录历史?这是值得我们共同探讨的问题。

以各种人物、各种角度为切入口,记录历史。

在我们以往宣传的人物形象中,小人物、或者普通人的形象占了很小的篇幅,可以说,在当代中国的媒体上,我们并没有由小人物构成的一系列形象,也没有一部完全由小人物构成的历史。

自从《东方时空》以一个栏目、每天一期的播出量开始“讲述老姓自己的故事”以来,从小人物为切入点来记录历史,以小人物来展示中国历史这段流程的工作已经在开始进行了。它除了转变了人们长期以来形成的一种意识之外,还为中国的纪录片作了一些努力。

以前,我们的拍摄对象都是出于一种被拍摄状态,在《生活空间》的节目里,被拍摄对象则处于一种生活状态,这就改变了人们长期以来习惯的一种视觉符号,丰富了他们的视听语汇,同时还教化和培养了一批观众,使他们可以读懂这些语汇,这是用普通人作为突破口来完成的一种教化,一种意识的转变。这种转变之后,一旦人们能够接受真实,读懂真实和识别真实之后,其实更需要用真实的光芒去照亮的领域远比这个更多。因此,从“讲述老百姓自己的故事”开始,我们从普通人的切入口进入纪录片的创作,我们提出:为未来留下一部由小人物构成的历史。然而,对中国的纪录片来说,仅仅局限于小人物是远远不够的,为未来留下一部由小人物构成的历史是必须的,但纪录片决不等于只能由小人物来留历史,还应该从大人物,或者古人,或者是未来的人,各种的角度切进去,只有这样记录的历史才是真正的历史。

\section{纪录片应与其他的艺术门类相通}

%%\textbf{纪录片应与其他的艺术门类相通}

首先,纪录片是一种艺术品,它仅仅是在电视圈内被认知是不够的,纪录片老是由搞片子的人来研究也是不够的,纪录片应该和其他的艺术门类相通。

如果都是做纪录片的来研究纪录片,我们所研究的只能是技术;而如果让一个文学批语家来研究纪录片,他就不会去研究镜头,不会去研究剪接,他会说主题,说节目所包含的内容。因此如果对纪录片的研究仅限于一个窄小的圈子,那么大家也只能去研究技术,因为当你进入一个专业的时候,实际上也就进入了一个狭小的空间。对中国今天的纪录片来说,我们现在需要的是一桶水的来源,就需要打破狭小的专业,需要研究另外的领域,虽然他们也许只是一勺水,但把这一勺水都端过来就给了我们一桶水。

其次,纪录片不应该成为很孤立的、很专业的东西,它应该和社会发生多方面的联系,它最终的结果还是要和这个时代的敏感神经发生联系,和这个时代的重大问题发生联系。如果没有这种联系,就不可能和观众找到联系。举一个极端的例子:如果一个很有意思的历史事件突出发生,有人用非常差的技术,非常业余的水平去把它记录下来,只要记录得完整了,它就可能成为一部经典的纪录片。在这个极端的例子里,历史是非常重要的东西,技术的东西可以消失,因为在这里最中心的东西凸显出来了,这也是最核心的东西。又一个比喻,用来比喻中国纪录片的局部是恰当的:卢浮宫里挂的都是油画,而我们现在都是画素描的,每一位画油画的大师都是从素描开始的,但不是所有画素描的人都可以把油画挂在这里。我们现在整个纪录片的发展状况都是在解决问题,我们群体的状况是还没有解决技术问题,我们缺乏合格的人才。因此,我们应该知道自己的状况,踏踏实实地从一点一滴入手。但是我们肩负着双重责任,一方面是自己的业务需要不断提高,另一方面,这段历史谁来记录?油画的时代,你不能还是画素描,你也得凑凑合合地画油画。

\section{纪录片面临的挑战:对社会关注能力的挑战}

我们和国际纪录片的差距在哪?第一,技术上的差距。比如录音质量和画面影像质量,技术差距中最明显的是声音的差距。第二,艺术上的差距。国外的纪录片大师不单单是做纪录片的,他首先是一个艺术家,他具有一定的艺术修养,对生活的认识以及了解的程度已经达到了一定的水准,因为只有达到了这一水平之后,他才能把自己对生活的理解通过视觉的手段表现出来,这种修养是一种国民素质,需要一种环境,需要一代人一代人积累。而这一点恰恰是我们所欠缺的。

这里有必要谈一下纪录片与故事片的不同收视心理。纪录片最重要的是唤起我们的理性到场,它不是诉诸于人们的非理性,不是让你进入一种很本能的感官享受,相反,它是呼唤你的理性到场,让你去思考这个问题,进入这个问题的核心,因为这些问题就发生在你的身边,发生在你所生活的城市,所以你无法逃避,你必须面对。当你欣赏一部纪录片时,实际上也是和创作者的一种心灵的对话,在这个过程中,你必须要求自己的理性时时刻刻处于十分清醒的状态,这也是纪录片与故事片在观看心理上最本质的不同之处。故事片是靠怀节、从感情上来打动人,靠感官来席卷观众;纪录片除了在情节和情感上打动人之外,还要求观众用另外的东西来参与,这就是清醒的理性精神。

因此,中国纪录片面临的挑战,不光是技术上的挑战,还包括创作者对社会关注能力的挑战,即我们选择什么、关注什么。

中国的纪录片创作者必须回到我们的日常生活、我们现实的土地上来,关注中国现实生活中所出现的种种问题,这可以说是中国纪录片生命和基础。

现在我们有很多纪录片热衷于讲述一个悲欢离合的故事,如果仅仅是这样一个故事,而没有和大的文化背景、时代背景、民族命运相关联的话,其实是背离了纪录片的本原。因为故事片更好看,更能使人动情。现在我们需要解决的一个问题就是:因为我们走得太远,以至于忘了我们为什么要出发。当我们认真地去研究怎样去拍纪录片的时候,或许我们已经淡忘了我们为什么要拍纪录片。也就是说,当你过于进入、过于热衷于一个东西时,你就需要放弃这个东西。你只有出去了才能进来,也只有进来了才能出去。现在中国的纪录片恰恰需要跳出去。不要过于陷入,你才能反过来冷静地加以审视;如果一个人过于热爱,这东西就已经不再是它本身,已经变成了你的一种热爱,强加了你许多个人的东西,而不识事件本身。“太极”里说,练太极的人中太想练成的和三心二意地人都练不成。你必须保持一定状态,得到一种真传。按照西方的美学表述就是,距离产生美,必须有一定的距离,贴得太近反而什么也看不见。

\section{关于纪实的问题}

纪实是纪录片的取材过程,取材就相当于你从山上把石头采回来,但这决不等于说你有了石头就有了宾馆、有了机场和商店。你要用石头去搭建,前提是要把石头取出来。你取出来的是劣质石头,你搭的楼就会塌,纪实就是要你选择最优质的材料。不是说要你去找石头,你就能找到最好的石头。这里面就有艺术,就有思考,就有理性,就有深度。有些人强调楼要如何去造,而中国纪录片还面临着不会采石头的问题。我们不能365天拍摄,24小时开机,那么我们首先遇到的问题是什么时间来开机是最合理的。也就是说,选择拍摄时机是第一位的,第二位的才是我到了现场选择什么角度拍,用什么景别。这就是取材,锤子从哪里下,先采哪块不至于塌方。中国纪录片的发展需要一个从虚假取材到真实取材,从取伪劣产品到取大理石的过程。

这里还要解决一个问题:通常人们把纪实理解得很容易,把纪实手法说得很廉价,其实这是一个误区。纪实不是一句话,不是说你要纪实就纪实,那叫“跟腚”,不叫纪实。纪实本身也是艺术,这里面也有理性到场和深刻的问题。你在拍摄被拍摄对象时,其实无时无刻不在拍摄自己,会看的人一眼就会看得出作者在想些什么,作者的理性是否到场。判断一个镜头的好坏,首要的不是看它运动得是否流畅,而是看它为什么要运动,一个摇镜头重要的不是摇得匀不匀,而是摇的动机是否深刻、准确。

西方纪录片发展100年来,形式上的变化也是五光十色,它以变应万变,实际上是一种自觉的选择。我们现在的问题是,我们是不是把纪实的艺术做好了,是不是把取材的问题解决了。安东尼奥尼在他的《云上的日子》里说:每一种真实后面都还有一种真实,循环往复,以至无穷。这个画面是一个摇镜头,最后这个人物再往后移,往后移,在这句话说完之后,这个人物已经隐入了光的黑区,整个脸是黑的。画面的喻义告诉我们:最后我们看到的是一个黑洞,是一个永远看不到的地方,这就是他对真实的一种思辨。什么叫真实,最后的真实是看不见的。从空间来说,真实就是角度;从时间来说,它是一个无限接近的点。

电视行将摧毁的大部分正是代表着人文精神的书本文化所造成的成果,看电视代替了阅读,人们在电视面前变得思维懒惰,电视降低了整个社会文化水准、精神素质。电视愈是普及,精英文化越是被冷落。他们的忧虑不无道理,但却缺乏一种平民意识。当我们承认目前“电视降低文化水准”这一事实的同时,更应看到:电视吸引了更多本来就没有条件接触印刷文化,尤其是精英文化“引车卖浆者之流”参加的社会文化活动中,包括向《生活空间》这样直接让普通老百姓走上荧屏,再现他们的生活,这不能不说是它的巨大成就。现在的问题是,承载着启蒙任务、人文精神的严肃文化、经营文化与电视文化是否可以再在新的框架中进行联姻,从而使精英文化、大众文化、电视文化提高人们的品位?现在有两条路,一条是直接让精英文化走进电视,如中央电视台开办的《文化视点》、《读书时间》等节目,由此使一些学者成为所谓“电视知识分子”;一条是使电视工作者从精英文化中吸取营养,将人文精神渗透到他们创作的电视节目中去,《生活空间》进行的正是这方面的尝试,就我国的国情和大众文化素质来看,后一条路子似乎更加可行,也更能达到启蒙的效果。

这或许是我们为之而努力的。

\bibliographystyle{plainnat}
\bibliography{gk}
\clearpage

